\documentclass[]{article}
\usepackage{lmodern}
\usepackage{amssymb,amsmath}
\usepackage{ifxetex,ifluatex}
\usepackage{fixltx2e} % provides \textsubscript
\ifnum 0\ifxetex 1\fi\ifluatex 1\fi=0 % if pdftex
  \usepackage[T1]{fontenc}
  \usepackage[utf8]{inputenc}
\else % if luatex or xelatex
  \ifxetex
    \usepackage{mathspec}
  \else
    \usepackage{fontspec}
  \fi
  \defaultfontfeatures{Ligatures=TeX,Scale=MatchLowercase}
\fi
% use upquote if available, for straight quotes in verbatim environments
\IfFileExists{upquote.sty}{\usepackage{upquote}}{}
% use microtype if available
\IfFileExists{microtype.sty}{%
\usepackage{microtype}
\UseMicrotypeSet[protrusion]{basicmath} % disable protrusion for tt fonts
}{}
\usepackage{hyperref}
\hypersetup{unicode=true,
            pdfborder={0 0 0},
            breaklinks=true}
\urlstyle{same}  % don't use monospace font for urls
\IfFileExists{parskip.sty}{%
\usepackage{parskip}
}{% else
\setlength{\parindent}{0pt}
\setlength{\parskip}{6pt plus 2pt minus 1pt}
}
\setlength{\emergencystretch}{3em}  % prevent overfull lines
\providecommand{\tightlist}{%
  \setlength{\itemsep}{0pt}\setlength{\parskip}{0pt}}
\setcounter{secnumdepth}{0}
% Redefines (sub)paragraphs to behave more like sections
\ifx\paragraph\undefined\else
\let\oldparagraph\paragraph
\renewcommand{\paragraph}[1]{\oldparagraph{#1}\mbox{}}
\fi
\ifx\subparagraph\undefined\else
\let\oldsubparagraph\subparagraph
\renewcommand{\subparagraph}[1]{\oldsubparagraph{#1}\mbox{}}
\fi

\date{}

\begin{document}

\section{Donald Knuth Interview 2006}\label{donald-knuth-interview-2006}

This is a transcript of an interview with the guy by Dikran
Karagueuzian, the director of CSLI Publications, videotaped in 2006.
Copyright 2006 Web of Stories Ltd., but the version of the transcript on
their web site is just unreadable; this is my effort to clean it up a
bit. It's about 63~000 words, so it is probably around three or four
hours of reading. Printed, it would be about 110 pages.

Patches to improve the formatting and correct errors are welcome. Please
don't refill the paragraphs, though; it creates unnecessary merge
conflicts.

I know lots of people are going to hate the random boldface. But the
interview, like any record of an extempore speech, is not really
organized into headings, subheadings, and the like. The boldface is my
attempt to pull out the main topics of each paragraph so that it's
possible to skim through the document to find a particular topic.

\subsection{\texorpdfstring{\href{http://webofstories.com/play/17060}{Family
history}}{Family history}}\label{family-history}

If you want to go way back, if you go like, 16 of my
great-great-grandparents, in 1840 they all would have been in Germany,
but by 1870, they all were in America. So if you know, consider all the
different lines. My mother; my father's, the Knuth part of my ancestry
was the most diverse, in a way. He came from Schleswig-Holstein, rather
near the border with Denmark, and he was the last to come over; during
the Schleswig-Holstein crisis in the 1860s; I think it was 1864,
probably, he went AWOL from the army; he didn't want to fight against
the Prussians, and he decided to come to America, and knocked on the
window one night, and told his parents, I'm out of here, and wound up in
Illinois, and then worked, learned to be a blacksmith at that time.

So then his wife was somebody he met in America. She had come --- her
family had come over earlier from the Hannover area, and she lived in
Indiana, all very near Chicago area, and so that part of the family is
from a different part of Germany. My mother's side came; they all
emigrated in 1840s, and they were farmers in the area of what is now
Niedersachsen; it's a small town in Germany called Bad Essen, and they,
her family came and were farmers in Ohio, near Cleveland, Ohio. So they,
yeah, the background then is, my father's side round Chicago, where he
was born and grew up, and my mother's side from Cleveland Ohio, where
she was born.

My dad's first teaching assignment was in Cleveland, and that's where he
met my mom. Then he got a call to Milwaukee, which was way far from any
of their, you know, anybody else in the family, because it was a job
that opened up, so he went and took this job at a school that needed a
teacher.

\subsection{\texorpdfstring{\href{http://webofstories.com/play/17061}{Learning
to read and
school}}{Learning to read and school}}\label{learning-to-read-and-school}

It all started in Milwaukee, Wisconsin, 1938, I was born. I don't
remember anything about the first few years of my life, but I know a
little bit from records that my parents kept that they were kind of
unusual at that time, in introducing me to reading. All of their friends
said that I shouldn't, that I would be bored in school if they would do
much reading for me, before I actually went to school, but I was --- I
guess I was the youngest bookworm in the Milwaukee Public Library.

So that's the first, news I have from the past, because they saved a
newspaper clipping. I guess I was like two and a half years old and I
had become a bookworm at the Milwaukee Library. I start remembering
things more when I get, you know, when I get into school, and I went to
a small school at our church.

My father was a teacher there. His life's work was to be an educator in
the Lutheran School System, and their salaries weren't much to speak of.
I think, I think it was something like ten or \$15 a month. But it was a
very warm, loving community. We were pretty much ignorant of what's
going on in the world, but happy and sort of a stable, nice place to
grow up. And when I was in First Grade my dad was the Second Grade
teacher, but then he moved on so that when I got into the Second Grade
he was the Fourth Grade teacher, and when I got into Fourth Grade, he
was the Sixth Grade teacher. And finally, when I got into Sixth Grade,
he went on to teach in high school, so, fortunately, I never had my dad
as one of my teachers.

In this school we had about oh, I think 20, 25 kids, and our teachers
weren't real strong on Science or Mathematics, they were, but they were
pretty good in English. They- like in Seventh Grade, I remember, that
several of us would spend time after school, diagramming sentences; you
know, take sentences of English that weren't in the book, and figure out
what's the subject and the predicate, and how do the, you know, how do
the phrases go in, so we, our teacher inspired us to know a lot about
the English language, and by the time we got to high school, the
teachers there in high school really didn't know that much about
English, so that was the only time I remember being bored, because we
already knew everything that they were going to teach us in high school.

So the high school I went to, again, was a Lutheran high school. The
people who work in these schools are, like my dad, doing it as kind of a
mission, or, I don't know, not, they consider it their life's calling to
be good teachers, and so we really had people who took a genuine
interest in us, and were not, not in their job just because it was a
job, but because it was something that they felt was an important
service to the world. Some people think that the church schools are
places where they teach intolerance and you know, that you're supposed
to only be, what do you say anyway? Only, only be appreciating of people
who are like you, and so on, but that was absolutely not the case. I
think we had a very good experience and it was, there was, one of my
teachers was a little bit prejudiced against blacks, but he stood out,
and we didn't pay any attention to him.

So it was a, I think a really nice way to grow up, but nothing world
class, in any sense of special, or, you know, special knowledge or
unusual.

I think I was kind of a wise guy, I think, I'd often sit in the back of
the class and crack jokes, and the teachers didn't, didn't really like,
the way, you know, my attitude. But you know, they learned to live with
it, and I didn't get, I didn't get spanked too often.

We had good music also, in the school, for singing, and, but we also had
a lot of freedom, so I remember like, we had a circle of four or five
friends, and when we were in Fifth and Sixth Grade, we started doing
some little projects, like we got hold of a tape recorder, this was in
the 1940s, and we tried to write scripts for fake radio programs, and we
pretended we were on the radio, and we put on these little shows, and
recorded them.

My friends and I started a school newspaper. We called it ``Newsweak'',
spelled W, E, A, K, of course. And in that paper we would tell stories
about things going on in the school, but also we recycled a lot of corny
jokes we had seen in books, and had a puzzle page and things like that.
But that was when I, I had my first experience in writing, and as I say,
we had good English in that school, so from early on, I had a lot of
training in things that had to do with languages, and then a chance to
do some creative work with things like these skits.

\subsection{\texorpdfstring{\href{http://webofstories.com/play/17062}{My
mother}}{My mother}}\label{my-mother}

I should say something about my mom, who was also very important in my
life, of course. She was unusual also at that time, because she had a
good job. My parents were the first in their families, in the whole
history of our family, to have some kind of education. My dad had gone
to a teachers training college in Chicago, and my mother had taken one
or two years of training as a secretary, as a legal secretary.

And in the, during the Second World War, she got a job working for a man
in downtown Milwaukee, who owned and managed several of the skyscrapers
in town, downtown, and she became his personal secretary, and later a
Trustee of the, of these, of some of the organisations that owned the
property, and the time of his death she was, she had been doing rather
well, and was asked to be, asked to be the manager of one of the
buildings, so she spent her life working in real estate in Milwaukee
with the large commercial buildings in the downtown area. And she had
started doing this part time work, when I was five or six years old, but
still doing everything else at home, as well.

My, but my grandfather was, my great-grandfather was a blacksmith, my
grandfather was, my grandfathers on both sides, were in construction
work, and maintenance. There was not a family tradition of education,
and I certainly was the first one ever to go on to higher levels of
education. It's partly the story of America, of course, that more and
more people are going to college, but when I was in high school, at that
time still, it was, I think something like 7\% or 8\% of my classmates
went on to college, and that was considered pretty good at that time.)

\subsection{\texorpdfstring{\href{http://webofstories.com/play/17063}{My
parents' finances}}{My parents' finances}}\label{my-parents-finances}

I guess I can say a little bit more about my parents' finances. Of
course they were married during the Depression, and my dad's first job,
when he came to Milwaukee, they decided after a year, that they should
cut his salary by \$5 a month, and they told him that he would learn
thrift, and he would, you know, this would be good for him, to, and of
course the church was always having trouble with fundraising, but our
family didn't have an automobile till 1951, which is when I was in
Seventh Grade.

We took our first, you know, auto trip at that time, to my mother's
family in Ohio. Before that we had gone once or twice a year by train,
and rarely would relatives from Illinois or Ohio be able to come to
visit us.

We, my dad would always ride his bicycle to work, and my mother, well,
Milwaukee had good public transportation, and in fact it was a very safe
city; completely different from now, when, because of drugs and things
now, but when I was growing up, I could, I would always ride the
streetcar downtown at all hours of the day and night. Now you take your
life in your hands doing this. You know, no parent would let their child
do any of the things that we were doing.

I was, once I took the streetcar; I don't know how old I was, but I
think it was, you know, maybe Fourth or Fifth Grade, and I took the
streetcar downtown, and went to the public library, and I started
reading books, and I didn't know that the library hours, that the
library was closing, and the lights went off, so I went over to a
window, where I could sit, and I kept reading, and finally the people,
you know, and my parents didn't, were wondering what happened to me, why
didn't I come home?

But somebody at the library found me, you know, just in the stacks,
reading the books, and but you know, still they wouldn't worry too much
about letting kids go around the city, and so my mother would take the
streetcar downtown to her work, and we never thought of having an
automobile until they could finally afford something from the little
jobs that my dad would take outside of his teaching experience.

He would; he and his friends got together and they invested in the stock
market, what they called penny stocks. You could buy stocks in various
mining companies. He met a man at the, I think it was at a, what do you
call it? A bathhouse? Or, you know, a public bath, where you can go for
a hot bath, and met a man there who introduced him to some people from
Colorado who were into mining and so then they decided they were going
to make some money this way.

Well, I inherited all these stock certificates now, which make great
wallpaper, but they, you know; they're completely useless. One of the
stocks, however, did well. It was called Silver Bell Mine, and we've,
eventually we went out to Colorado, visited the place, and saw the, you
know, we, the stock was sold to Union Oil eventually, and made a little
bit of money on that. So, and my parents bought stock in Walt Disney,
and that kept doubling and multiplying.

So by working hard and saving money, throughout my mom's life; she died
at almost age 90, and she never retired, she stayed working in real
estate, even in her late 80s they had an office for her downtown, and
she could come in, maybe three days a week, and, but putting money in
the bank, and being a generous person, and contributing to charity and
so on, but she had accumulated an estate of more than \$1 million, by
the time that she died. And this surprised everybody including her.

But they were, my dad was the bookkeeper for the high school, and so he
took some classes in accounting, and so he spent a lot of his time
actually filling out what we'd call spreadsheets now. And he'd stay
keeping track of little transactions with stocks. The, so, I think it's
interesting to see that the way they could, just by being responsible
citizens, make a good life for themselves, the way the times were in our
country at that period. And this idea of personal responsibility was
something I always took for granted, because I got it from my parents.
So it never occurred to me that there was any other way to live.

\subsection{\texorpdfstring{\href{http://webofstories.com/play/17064}{Interests
in high school (Part
1)}}{Interests in high school (Part 1)}}\label{interests-in-high-school-part-1}

Back in high school, my main activities in high school, at, it was
called Lutheran High School, and then during my Senior year it split
into two schools, Milwaukee Lutheran High School, which was the one I'd
went to, and Wisconsin Lutheran High School.

1956 was the year I graduated, and that, well, the place where again I
didn't have special, teachers who were specialists in a world class
sense, but they were always real interested in nurturing us.

I went on to be a mathematician, but the math, but I didn't have any
interest very much in Math at the time in high school, because my, I
would ask questions to my teachers, and they didn't know the answer. So
I could prove that one was equal to zero, and they couldn't find any
mistakes in the proof, and so you know, I couldn't, and so, why should I
go in, you know, go into mathematics? So my main interest at that time
was in music, and also in physics.

My chemistry/physics teacher was a wonderful man, who sort of wrote his
own textbook, he designed the chemical experiments, and I had a great
admiration for him, and he encouraged me to you know, to think some
about physics, and although I spent most of my time in music, outside of
school.

In fact I, well I played the piano with the high school chorus, I sang
in the chorus, I was in the band, I played the saxophone and I played
the tuba in the band, and I played, you know, in the All-City, the
Milwaukee All-City Band, or Symphonic Band. I wrote music, I arranged
music for bands. I had, I had, at the time I took a; I was a big fan of
\emph{Mad Magazine}.

And also of Roger Price, if anybody remembers Roger Price, and he had
written a short story called ``Milton and the Rhinoceros''. And so I
made a take off on Prokofiev's ``Peter and the Wolf'', with the words
from ``Milton and the Rhinoceros''. I was very naïve at the time; I had
no idea about copyrights, or anything, so I took Prokofiev's music, and
I scored it for band, and I took Roger Price's words, ``Milton and the
Rhinoceros'', and I made this, this piece for, I don't know, 20 minutes
or something like this, for our, for our high school band to perform.
And I proudly gave this to our band director, and he lost it, and I've
never seen it again.

So I have no idea whether it's any good, or whatever, but there it was.

\subsection{\texorpdfstring{\href{http://webofstories.com/play/17065}{Interests
in high school (Part
2)}}{Interests in high school (Part 2)}}\label{interests-in-high-school-part-2}

I also started the school newspaper. I was Editor of the paper, and
during my Senior year, I would do an all-nighter every Monday night. I
stayed up the whole night, getting the paper done by seven o'clock in
the morning. We didn't have a professional way to produce it, so I
produced it on a mimeograph machine, which was something that's way
obsolete now, but it uses very greasy ink, and the reason I did that was
because we happened to have one at our house.

My dad did part time work for an architect in Milwaukee, printing up the
specifications, and you know, he could make a little freelance money
this way to supplement his meagre salary, and he also used this machine
to typeset music for local choirs, and so I had this machine at home,
and also some electric typewriter that he had, or maybe my mother had;
anyway we had it at home, so I could produce our school newspaper at
home, without having to worry about fancy typesetting, but we had a lot
of reporters, and I wrote some features for it, like crossword puzzles,
and things like that. So again, I was doing a lot of writing in my spare
time, during high school.

I worked on the Yearbook and other publications as well, and friends of
mine, friends and I wrote plays that were put on by groups in the high
school. So it was a fun time. But I would say at that part of my life I
was pretty much a, like a machine, an autom-, I mean I was just, I would
just learn, absorb stuff, and take tests and you know, get 100\% on the
tests if I could, without really sitting back and taking a look at finer
things in life, or something like this. I was a, I was a dutiful child
who said, okay, you're supposed to go to school Don, so I went to
school, and you're supposed to learn, so I learned, and I had fun on the
side with some of these writing projects, but I really wouldn't read a
book unless it was assigned to me.

Sometimes I would read, I remember, I'm a very slow reader, and I
remember \emph{Bleak House}, by Charles Dickens, with 60, 70 chapters,
and I, it took me so long to read it that I had to use it for two book
reports instead of one, and I didn't; so it wasn't until I was in my 30s
before I actually found some of the great literature of the world and
read it for my own pleasure.

In high school, well I guess I was a fairly successful machine, because
they said that my, okay, in those days they didn't give just letter
grades, like A, B, C, but they gave a number grade, and based on you
know, when you took an exam, they would average these scores, and they
would grade your homework and it was all based on a zero to 100 system,
and they said that my average which was more than 97.5 was a record for
the school that hadn't been achieved before, so I was pretty much a nerd
of nerds at that time.

\subsection{\texorpdfstring{\href{http://webofstories.com/play/17066}{My
sense of humor}}{My sense of humor}}\label{my-sense-of-humor}

Rather than sceptical, cynical, I would prefer satirical or something.

KARAGUEUZIAN: Satirical.

KNUTH: But anyway, that's why I liked \emph{Mad Magazine}, because it
was like crazy type of satire on the sacred cows of the day, and so when
my friends and I discovered in high school, you know, we devoured every
page, and it was a special, you know, mad about \emph{Mad}. We, and
before that, as I said, I had the corny jokes, but my friends and I
tended to be wise-cracking and not to take things too seriously.

Although I said I was a machine, but I did also like to laugh, and so
when I was writing for the paper, we always had sections about jokes,
and we always had you know, in our Yearbook we had fun things in there
that weren't expected. So I've carried that through, as you know, and
many; there are lots of corny jokes in the indexes to my books now that
people probably haven't discovered yet, but somebody will ask me, why do
I have a reference to Bo Derek in \emph{The TeXbook}? And it turns out
that just all the pages which are cited in the index for Bo Derek is
where I used the number ten, so all the way through I've had this silly
streak of some sort that means I don't take everything too seriously.

\subsection{\texorpdfstring{\href{http://webofstories.com/play/17067}{The
Potrzebie System of Weights and Meausures and the Alfred E. Neuman
crossword}}{The Potrzebie System of Weights and Meausures and the Alfred E. Neuman crossword}}\label{the-potrzebie-system-of-weights-and-meausures-and-the-alfred-e.-neuman-crossword}

While I'm on the subject of \emph{Mad Magazine}, I might as well mention
that then I did a project in my senior year, which was submitted to the
Wisconsin Academy of Sciences, well, no, first to the Westinghouse
Science Talent Search, which is now called the Intel Science Talent
Search or something, but it's a nationwide competition, and it was
called the Potrzebie System of Weights and Measures. This is a Polish
word that was very popular in \emph{Mad Magazine}, and I decided to base
a system of weights and measures that was going to be better than the
metric system. It was based on the thickness of \emph{Mad Magazine}
number 26, or something like this, and that was taken as one potrzebie
of length. And then we had, you know, a kilo potrzebie was 1,000 of
those, and a fershlugginer potrzebie was a million --- and a farshimmelt
potrzebie one millionth. We had units of time, weight, everything, in
the system.

And it won honorable mention in this Science Talent Search, and also
was, I did some demos at the Academy of Sciences; it won an award in
Wisconsin. I submitted it them to \emph{Mad Magazine}, and they
published it; they paid me \$25, for it, and it came out in, during the
spring of my Freshman year in college, and this was my first technical
publication, so it is listed on my vita, you know, and publication
number one is the Potrzebie System of Weights and measures, and it was
published later in paperback in one of the \emph{Mad} reprints, and it,
and I used it as a basis of running for student government in college,
and it failed miserably, you know, I didn't get elected.

I also sent a sequel to \emph{Mad}, which was a crossword puzzle. One of
their, the Alfred E. Neuman figure is one of the, was featured on lots
of their articles, and this silly boy's face with the missing tooth, and
when I looked at his teeth, I could see, you know, white squares and
black squares, reminding me of a crossword puzzle. So I added more white
squares and black squares up in his forehead area, and where his hair
is, and I filled, and I made that into a crossword puzzle with
\emph{Mad}-type clues, like, you know, ``blank, me worry?'', or
something like this, and I submitted that, and they rejected that, you
know, they didn't want, but I've still got it, and later on when we get
to writing,

I'm hoping that I'll someday publish a book called \emph{Selected Papers
on Fun and Games}. We will not only reprint my first technical article
on the Potrzebie System, but also the one that was rejected by
\emph{Mad}, and we will see the; people will decide whether they, you
know, will be able to decide now whether they made the right decision in
rejecting it.

So yeah, I like to see things sort of out of the box as well as in the
box.

\subsection{\texorpdfstring{\href{http://webofstories.com/play/17068}{Feeling
the need to prove
myself}}{Feeling the need to prove myself}}\label{feeling-the-need-to-prove-myself}

My parents were asked whether or not they should advance me a grade,
because I was doing well in school. And they; in fact I was born in
January, which meant that I was older than most of the kids in my class,
because if I'd been born in December, I would have gone to school a year
earlier, so I was one of the oldest in my class, and my parents, you
know, a lot of kids at that time were being pushed ahead in school, and
one of my friends, you know, graduated from college when he would
normally have been entering high school.

Well I, but I'm glad that they resisted this, because it gave me this
time for all these extra-curricular activities, so I was always into a
lot of things. I mean sports; I was a terrible athlete. I'm tall, but my
left hand doesn't know what the right hand is doing, so I would only get
in the game after we were already behind by 50 points, and, but I was
the scorekeeper.

I was a terrific scorekeeper, you know, so I became manager of many of
the sports teams in high school, and also in college. I was manager of
the cross-country team, I was manager of the basketball team, I was
manager of the baseball team, so I was a six or seven letter man without
being able to do anything athletics, you know, whatever.

But I'm sitting there at the score table during the basketball
tournament, and- a basketball game, and like, I'm talking a mile a
minute to keep the other scorekeeper confused, so that he wouldn't know,
you know, if somebody said, who's the fouler? And I could tell them, you
know, who the foul was on, based on, you know, who it was most desirable
for us to have the foul be on, rather than the person who really
committed the foul, because the guy's, you know, very mixed up by the
way I'm talking. But anyway, I don't think I won too many games at the
scoring table, but I was the person who went along with the team and
kept the scorebooks.

But all these extra-curricular activities were something that I had time
to do, because my parents hadn't advanced me, and so you know, I could
do well in my classes but also participate in lots of other things.

Now my, but, you know, so here I am, doing rather well at school, but
I've still got an inferiority complex. I'm still always trying to prove
myself. I'm thinking, you know, maybe I'm not getting it well, and in
fact, so I was probably taking, you know, studying hard and getting 100s
on these exams, because I'm trying to prove that I knew it. You know; if
I was really overconfident, so confident, I wouldn't have bothered to
study. And that was, so that was one of the child's eyes looking at it.
I'm always scared that I'm not going to do well, so I was working hard
at schoolwork.

The funny thing was that my; when I was being advised what to do for
college, well, everybody in my school took a whole bunch of tests,
national tests, and one of the tests was vocational, you know, what job
are you cut out for? And I remember that that, according to that test, I
should be an architect, and I should not be a veterinarian. I scored
extremely low on veterinarian skills, but for architect, that seemed to
be the career path that was recommended to me.

But the, and I won scholarships to different colleges, but the people at
my high school, the Vice Principal called me in; he said, ``Don, I think
you're going to be a failure in college''. He said, you know, you've
done well here in high school, but college is completely different, and
you're just, you know, it's just going to be too much for you, you're
not going to, you aren't going to make it. And, well, so he scared me
again, and when I got to college I kept studying.

In fact, just the week before college, the Dean of Students, whoever it
was, told us. I went to Case institute of Technology in Cleveland, which
was at that time not yet connected with Western Reserve University. So I
went to Case, and the Dean of Case says to us, says, it's a all men's
school, says, ``Men, look at, look to the person on your left, and the
person on your right. One of you isn't going to be here next year; one
of you is going to fail.'' So I get to Case, and again I'm studying all
the time, working really hard on my classes, and so for that I had to be
kind of a machine.

I, the calculus book that I had, in high school we --- in high school,
as I said, our math program wasn't much, and I had never heard of
calculus until I got to college. But the calculus book that we had was
great, and in the back of the book there were supplementary problems
that weren't, you know, that weren't assigned by the teacher. The
teacher would assign, so this was a famous calculus text by a man named
George Thomas, and I mention it especially because it was one of the
first books published by Addison-Wesley, and I loved this calculus book
so much that later I chose Addison-Wesley to be the publisher of my own
book.

But Thomas's Calculus would have the text, then would have problems, and
our teacher would assign, say, the even numbered problems, or something
like that. I would also do the odd numbered problems. In the back of
Thomas's book he had supplementary problems, the teacher didn't assign
the supplementary problems; I worked the supplementary problems. I was,
you know, I was scared I wouldn't learn calculus, so I worked hard on
it, and it turned out that of course it took me longer to solve all
these problems than the kids who were only working on what was assigned,
at first. But after a year, I could do all of those problems in the same
time as my classmates were doing the assigned problems, and after that I
could just coast in mathematics, because I'd learned how to solve
problems. So it was good that I was scared, in a way that I, you know,
that made me start strong, and then I could coast afterwards, rather
than always climbing and being on a lower part of the learning curve.

\subsection{\texorpdfstring{\href{http://webofstories.com/play/17069}{Why
I chose to go to Case Institute of
Technology}}{Why I chose to go to Case Institute of Technology}}\label{why-i-chose-to-go-to-case-institute-of-technology}

In high school, as I said, mathematics was just confusing to me. When I
got to Case, though, the teachers, the chemistry teacher was, seemed to
know chemistry okay, but he didn't know anything about physics or math.
My physics teacher knew physics and chemistry, but my math teacher knew
all three subjects, and I had terrific respect for him, because he
seemed really smart, and also he was very hard to please. No matter, you
know, I could show him what I was doing in calculus, but he would, he
never seemed to be impressed, so this was frustrating to me. I'd never
had a teacher that I couldn't impress, you know, so I worked even harder
at my math. But he had a good sense of humor, and his name was Paul
Guenther. And finally, after two years, I was able to impress him a
little bit, and so this was good. But I developed an interest in
Mathematics, but I have to go back a little bit.

I, when I was choosing what college to go to, I had won several
scholarships, and one of the scholarships would have been to Valparaiso
University in Indiana, associated with the Lutheran Church, and it was,
but there I would have been a major in music, and the other scholarship
was to Case in Cleveland, which was where I would major in physics, and
since my mother's family is from Cleveland, she knew that Case had very
high standards, and very few people that she knew were able even to be
admitted to Case, and there was, considered to be one of the hardest
schools to do. I'd never heard of MIT, by the way, until some years
later, or much less Caltech. But Case, to me, was a challenge, where I
would really have to work hard and well, Valparaiso, it would be
something where I could just do some; music would seem to me much
easier. And I decided to go for the challenge, at Case, and I was
admitted to; they had a special section for Freshmen, called the Honors
Section, where 20 of us were given, were taught each of our classes by
the heads of departments. So I really had the, Case's best, you know,
physics teacher, chemistry teacher, math teacher and English teacher.

\subsection{\texorpdfstring{\href{http://webofstories.com/play/17070}{University
life: my basketball management
system}}{University life: my basketball management system}}\label{university-life-my-basketball-management-system}

I believe it was the first year they had tried this, and I'm not sure
how long they continued the experiment, but the teachers, but what they
called us the Honor Section, and this just meant that 20 of us just took
all our classes together, while other students would tend to mix. I
don't think the other students all spent Monday, Tuesday, Wednesday the
same as, the same hours, like we did. And so the classes that I had,
then I think they challenged us a little more too, than they challenged
the other students, perhaps.

Still, I had time to do you know, to work on the school paper and to,
one of the important issues of going to Case at that time was to join a
fraternity. The, most of the interesting action on campus would be
centered around the fraternity system, and so I pledged a fraternity
also in my Freshman year, and I guess I can say more about that.

But I had mentioned that during the spring of my freshman year is when
my \emph{Mad Magazine} article came out, you know, and so that well,
that led later to, later on I was editor of a magazine that we founded
at Case, called \emph{Engineering and Science Review}, which wrote about
topics in science, and I wrote an article about Potrzebie System for
that publication. I was in the music, I was in the basketball; I was
manager of the basketball team, at Case.

And I'll say a few words about that, because after I got into computers
later on, I combined that with my managing the Basketball team. So I
devised a strange formula that I don't believe in much any more, but
anyway, I had it at the time, where you could compute each basketball
player's real contribution to the game.

Not just the points that he made, the baskets that he made, but really
you know, taking everything into account, so for example, if you have
possession of the ball in basketball, this is worth something. In fact,
when you're watching a basketball game, if you sort of add one point to
the score of the team who has the ball that sort of gives a more fair
indication of what the real score is of the game. So possession of the
ball is worth maybe one point. You can work it out after the game; you
can say, really, how many times, when you had possession of the ball did
you really turn over, and fumble it, or you know, how many points did
you really get during that series? And so you can work out that it may
be worth only seven-tenths of a point or something like this. But
anyway, possession of the ball is worth something. So if you fumble,
then you've lost your team one point, or seven-tenths of a point. So
that's a minus something for you. If you steal the ball, if you recover
a fumble, you gain; your team gains possession, so you get credit for
stealing the ball.

If you make a basket, in those days it was only two points, you don't
have the three-point shots in those days, but if you can make a basket
you get two points, but your team also loses possession, the other team
gets possession, so you don't really; you didn't really win two points
for your team when you made the basket, you made two points, but you
have to subtract from that, the fact that you have to get the ball
again.

So at the end, according to my formula, the sum of all the players'
contributions would be the amount by which our team won or lost. But it
would rate, you know, if somebody makes a shot and misses, then
sometimes our team gets the rebound, sometimes the other team gets the
rebound, so I, you lose a little bit for taking a shot and missing. So I
calculated a huge number of statistics for every player, and I had a
spotter, who would call to me, and I could write it down, every little
thing, and after the game I would go and punch cards that recorded all
these statistics, and fed them into a little computer program that
calculated the formula and made a list for every player, what their real
contribution to the game was, not just the field goals and all these
traditional statistics.

So Case's Coach, Nip Heim, loved this system, and you know, he posted
these numbers, and the Case News Service was always good at trying to
plant interesting stories in the local paper, so they sent reporters out
to, and we showed, you know, told about this formula, and IBM heard
about it. So IBM sent out a cameraman, a camera crew, to make a film of
me spotting a game, you know, and there, our Case team playing
basketball of course, but then, how I would punch the cards, you know,
and put it into the IBM computer. Before they took the shot of the IBM
computer, they planted a great, big IBM sign on the machine, so that
nobody could fail to miss it, you know, that I was doing this, and then
I'm turning the buttons on the console, and getting the numbers out, and
it's getting printed out on the IBM printer, and then the Coach is
looking at it and posting this up.

So this was a little movie that I was in, about two or three minutes
long. IBM supplies this movie to CBS, and they put it on the Sunday
Evening News with Walter Cronkite, and all my relatives in Florida can
see me on TV. This was very exciting. Also US News and World Report ran
a story about it, and so this was my connection between computing and
sports, when I was at Case. This was also a clever way for IBM to get
their advertisements in there, rather subtly, but it was a fun.

That was when I first realized how hard it must be to be a movie star,
because I had to walk through these scenes six times everything, you
know, punch those cards over and over again. So I can see, how could
Audrey Hepburn possibly look so beautiful after the sixth take, you
know?

\subsection{\texorpdfstring{\href{http://webofstories.com/play/17071}{University
life: the fraternity
system}}{University life: the fraternity system}}\label{university-life-the-fraternity-system}

At Case, as I said, I pledged a fraternity. That was one of the things
that --- I'd met a few Case graduates in Milwaukee, or Case students,
before, and they said --- oh, Don, fraternities are the big thing at;
you've got to take that very seriously when you go there. Well, anyway,
the fraternity system was maybe a little different then than it is now.
It certainly had its pros and cons, but it was a big; it was certainly a
focus of my life at Case, because after my freshman year I would live at
the fraternity, and with all my fraternity brothers.

The first thing, though, was something we called Hell Week. Raising.
This is, or hazing, I mean. I'm sorry, forget that I said raising. And
for seven days, all of us pledges, before we could become full-fledged
fraternity members, were at the mercy of all the other brothers, who
would have their paddles, and also, you know, we were sleep deprived,
and basically they were, we were also cleaning the fraternity house, so
that's when I learned about ammonia and painting things, and scrubbing
walls, and doing other, you know, fixing the roof, and things like this,
that we would be doing, but then, but meanwhile they would also, they
would play tennis where we were the balls, and they would, you know,
swat us, and it hurt; and do other things to make sure that, you know,
we knew that they had the power and we were just unimportant now.

So this is illegal now, and, but it was my equivalent of going through
boot camp, I guess, which, you know, I never went into military service.
And, you know, I matured an awful lot during that week. After that week,
and I'd gotten through without the, you know, this time, I had a
confidence that I'd never had before, so it's hard for me to say that I
wouldn't want my son to go through the same thing. I don't think he ever
did, but it's a paradoxical thing, in my estimation. Still, that was
part of getting into the fraternity, and I have a picture of myself that
they took at the end of it, unshaven, and looking pretty beat, but still
knowing that I had come through an ordeal, which was something; it's a
question in my mind, how to really give that education to somebody in a
way that is legal, would be legal. Now we'd have people suing.

\subsection{\texorpdfstring{\href{http://webofstories.com/play/17072}{Meeting
my wife Jill}}{Meeting my wife Jill}}\label{meeting-my-wife-jill}

At my fraternity, my best friend, who was also in the Honor Section at
Case, his name is Bill Davis, he now, by the way, is a mathematics
professor at Ohio State, and is very active in internet online education
for teaching mathematics, but anyway, Bill and I were bosom buddies
during my freshman year, and he, and then also in the sophomore year we
both pledged Theta Chi fraternity, and so then we also went through Hell
Week together and had rooms next to each other in the fraternity house.

So I started to date a girl at Western Reserve University; her name was
Betsy, and her roommate was named Jill, and Bill was dating Jill, so I
was --- now, Betsy was Catholic, and I was Lutheran, and this was
something that we thought, well, our families would never be able to
understand, but still I thought that Betsy was pretty terrific, but I
was also dating a girl named Becky, a Jewish girl, whose father owned a
department store in Cleveland and I was dating some nurses too, you
know, there were lots of girls would hang out around the Case area, and,
but I was mostly interested in this Betsy, and I was, I decided to, and
we'd had double dates with Bill Davis and his girl Jill, and as I said,
Betsy and Jill were roommates; Bill and I were roommates.

So I was wondering, you know, how I'd ever be able to make more of a hit
with Betsy, and I decided I would have lunch with Jill, to ask her for
advice. And so this was probably the, my sophomore year, the spring of
my sophomore year, and, you know, she was such a good listener, and gave
such good advice, that I started dating Jill instead, and well, I'm not
sure Bill ever forgave me for that, but she didn't really like him, she
thought, she said, well, anyway.

Well, anyway, Jill is the one that I eventually got pinned to, and
engaged to and married to, and we've been married now for 45 years. And
Jill and I started having dates in the library, where we would study
together, and, you know, I found that I enjoyed kisses for the first
time, when they came from her. In high school I was afraid to kiss
girls, even when they would stand up.

I mean, I was dating a girl in high school who was about four and a half
feet tall, I don't remember her name any more, but I remember that she
was real cute, because, I always liked short girls, because they seemed
to, you know, have a good balance, they didn't fall over as easily as
tall ones, and I, being tall, I was, you know, but she was a
cheerleader, and she stood on the steps of her house, and with her lips
turned up, and I figured I'd got to kiss her, but I'd never done this
before, how am I going to do it you know.

And so I did, but I never realized how good it could be until I kissed
Jill, when I was a sophomore.

\subsection{\texorpdfstring{\href{http://webofstories.com/play/17073}{Bible
study at university and a time of personal
challenge}}{Bible study at university and a time of personal challenge}}\label{bible-study-at-university-and-a-time-of-personal-challenge}

I'm not a real involved with the groovy set, I guess, throughout all my
student time, and I remember then Jill and I started getting to know
each other real well, and we would go to church together on Sundays. She
was a Methodist, I, we studied, we went to Bible Study classes at her
church on Sunday School. I had never gone to Sunday School, because I
went to a Lutheran School as a child. We had something about Bible
during the week, so why should I do something, why should I do anything
more on Sunday? I've had plenty of religion during the week. But she had
gone to public school and she, her tradition was to go to study the
Bible on Sunday mornings, and so I went with her to church, and I
learned something about that.

In college, our teachers were, I was introduced to lots of atheists, and
people from lots of different religious experiences and this was
something totally different, totally foreign to my previous experience,
and I, so I went through a time of personal challenge, where I was
trying to say, well, what do I really believe? Up to that time I had
just, I had just listened to what my teachers said and passed the exams,
and did what was expected of me, but I internalized something about
what, you know, is there a God, and if so, how do you solve various
paradoxes associated with God?

And I went through that kind of experience during the Freshman and
Sophomore years mostly in college, before I became somewhat comfortable
with my, with internalizing what I had been taught as a child. And I'm
sure that if I'd come from another background, I would have turned out
differently, but, and I personally believe now that God is alive in many
ways, but I do believe, you know, I do believe that God is somehow,
mysteriously, involved with our universe, and that underlies a lot of
what I do, and I also know I will never be able to prove it, but I'm
thankful that I could never prove it, because if it was proved, I think
then I would lose interest in the whole subject; there would be no
mystery, and no interest in it.

So I, as they say, the growing up in a religious way took place during
this time, when I'm meeting my future wife.

\subsection{\texorpdfstring{\href{http://webofstories.com/play/17074}{Extracurricular
activities at
Case}}{Extracurricular activities at Case}}\label{extracurricular-activities-at-case}

I was on lots of sports teams, the cross country team, baseball. We
would have a chance to; we would have a chance to travel to lots of
other places then, with the team. That was fun as well. I worked on, I
was the founding editor of this magazine, \emph{The Engineering and
Science Review}, and then, active in various other things, for example I
was vice president of my fraternity.

There was one story about the newspaper and my fraternity I might as
well mention. That we, I was, I would go to downtown Cleveland, where
the newspaper was being typeset, and that's where I first got experience
with Linotype machines, and the way real printing was done. And I would
be the, at first I was the copy editor, so I would check for errors in
the text, before they did the final print run, and I noticed that there
was a story about one of our Theta Chi fraternity parties, before
Christmas, and it said that we had served hot buttered rum, and, well,
it dawned on me that we actually weren't supposed to serve rum at a
fraternity party, and still, it was a linotype machine, and you had to
pay for every correction that you made, and you had to keep your
corrections to one line, if they had to reset several lines --- so I
changed it to hot buttered popcorn. And that worked out okay for the ---
for the story.

Now, so I got more experience in writing, publishing, during that
period, and we had the Chair of the English department as my teacher, as
freshmen, and we had very good teachers also, you know, in my Western
Civilization class, sophomore, junior years, so all the time I was
writing. I was writing stories for, I was, you know, writing term
papers, but I was also writing for campus publications. And I came to
believe that really my education boiled down to 50\% mathematics, 50\%
English; 50\%, you know, writing skills, and somehow combining those two
things, for the rest of my life is what everything else was somehow a
mixture of those two things.

I was in so many extra-curricular activities in fact, that Case has
something called the Honor Key, which is based on points. You get so
many points for being in the band, so many points for being in a
fraternity, not very many for that, but certainly for the newspaper and
for singing in the choir, chorus, and participating in other campus
things, and if you get a certain number of points then you win the Honor
Key, and at, you know, at Graduation Day they'll mention the four or
five students who have won the Case Honor Key. Well, it turned out I had
enough points to win the Honor Key, but after three years, and, you
know, so I think I had more points, more Honor Points than anybody else
had had in the history of Case, so again I was, you know, a machine,
saying, ``Oh? There are points for this? Okay, I'd better do this.'' And
I signed up for these things. So I was involved with lots of
extra-curricular stuff.

I also had a chance to do a little bit of writing music. I wrote a
five-minute musical comedy for our fraternity to perform at the whatever
they call it, the oh, I forget what it was; it was an annual thing,
where each fraternity would put on some kind of a skit. Varsity Day, or
something, I can't remember what we called it. And we would go to a
theatre in downtown Cleveland and perform for whoever wanted to listen
in. I wrote this five-minute musical comedy called ``Nebbish Land'',
based on nebbishes, you know, these were popular in the Greeting Card
Industry at the time. And that I still have the score for, so maybe
I'll, maybe I'll put some of the music for that in my book on fun and
games.

\subsection{\texorpdfstring{\href{http://webofstories.com/play/17075}{Taking
graduate classes at
Case}}{Taking graduate classes at Case}}\label{taking-graduate-classes-at-case}

At Case I put a lot of time into stuff out of class, but in class, I
found a really clever way to, right now, let me brag this way to say, to
avoid having to study too hard, for my classes. In the first place, I
noticed though, that when I was a sophomore, my grades started to go
down, in the first part of my sophomore year. And I ascribed it to too
much ping-pong playing and playing cards too much in the dorm, and so;
no, I'm sorry; this was the second half of my freshman year. I started
having a little problem with my grades, and so I had to give up
ping-pong.

But starting in my sophomore --- junior year, I found out that you could
take graduate courses at Case, and they were easier than the
undergraduate courses. The reason is that Case had really strict
admissions requirements for undergrads, but they were fairly loose about
admitting graduate students. I think they wanted to build up, you know,
admit graduate students, so when you had graduate students, in a class,
they usually didn't know as much as the undergrads did, so if you would
take a graduate course, you didn't have as much competition, you know,
and the teacher would recycle stuff, and all this. So I started taking
graduate classes, and you know, and all these hotshot undergrads would
be taking the other classes.

And as a result, I had accumulated also, by the time I was a senior, I
had accumulated lots and lots of graduate credits. Now, as a result
then, Case did, on Graduation Day, Case did an unprecedented thing that
had never been done before, they awarded me a Master's Degree,
simultaneously with my Bachelor's Degree. And this, the faculty had
gotten together and made a, and voted unanimously that this should
happen, and I remember, you know, that was another thing that got into
the newspapers at the time, that they were awarding a Master's Degree at
the same time as a Bachelor's Degree. So, but the reason was that I had
taken these graduate courses because they were easier. I didn't, I don't
know if I've ever told anybody else this before today, but that was one
of the reasons I could do so many other things.

\subsection{\texorpdfstring{\href{http://webofstories.com/play/17076}{Physics,
welding, astronomy and
mathematics}}{Physics, welding, astronomy and mathematics}}\label{physics-welding-astronomy-and-mathematics}

I started in physics, and I continued in physics; as I said, I liked my
math teacher very much, but I was a physics major, and in sophomore year
I had another very good physics professor, and started learning quantum
mechanics, and had classes in astronomy, and I also had a class called,
I don't know what it was called, but it was a laboratory class.

All physics majors were required to know how to do things like welding.
Now, I was always very bad at lab work, in chemistry lab, I was always
the last to finish experiments, and I would break the beakers and get
chemicals on my hands and burn them and things like this, and start
fires, things; it was bad, but when I got into this welding, this class
where I'm supposed to do welding, it was just dreadful. I mean I
couldn't see, you know, I wear rather heavy glasses, and we had to wear
goggles while we were welding, but the goggles wouldn't fit over my
glasses, so I couldn't, you know, so I couldn't wear my glasses and my
goggles at the same time.

So I'm sitting there with these goggles that I can't see very well out
of, and I've got this electrical thing, which is thousands and thousands
of volts, where I'm supposed to be welding material, but the table is
way lower than I am; I'm kind of tall, and so I, and so here I'm holding
this thing way out of my range, and I'm supposed to, you know, I'm
supposed to get, solder things together, or whatever you call the stuff,
and when I was supposed to put, to get one thing attached to another,
the teacher would, you know, would pick it up and it would fall apart,
by its own weight, so I was failing in welding class. And it was
terrifying too, you see, with all this electrical juice going on in
there, and me not being able to\ldots{}

My astronomy class, I found out that it was very frustrating. I could
pass all the exams, in fact I got a 100 on every exam in my astronomy
class, but secretly I hated the subject, and I decided that I would
continue the classes as self-discipline, because I didn't want the
teacher to know that I hated the class. And I figured, you know, I'm not
going to be, I have to learn how to do stuff that I don't enjoy as well
as stuff that I enjoy. So I studied very carefully for the exams in
astronomy, but I really --- and why didn't I like it?

I'm trying, I believe it's just because I was just, I couldn't imagine
how, they were so different from mathematics, in astronomy you would
never be able to go to the sun and really know what it was like there.
You always had secondary information. If I would be an astronomer, I
would never, I would, I would have to die before I would know anything,
which was really true, because it's all based on speculation. It's all
based on our best guesses about the way the universe is, and nothing
that you can really experience yourself, or prove correct. In physics,
the same way, nature is beyond our grasp, and you don't know.

But mathematics was different. Mathematics had this certainty about it
that, where you could finish a problem, and you could say, you know, I
know I've got the answer. And so I like that. It was easy. It's much
easier; you know, I have to admire the people who, the scientists who
spend their life and never know whether they've solved the problem or
not, they just get supporting evidence for or against it, but to me,
anyway, mathematics got more and more appealing, for the reason that it
gave me some certainty; just the opposite of the reason why religion was
appealing, because it didn't have certainty, I mean I would feel unhappy
of the life where I had nothing certain, and life where I had everything
certain. In either extreme, life, it's hard for me.

\subsection{\texorpdfstring{\href{http://webofstories.com/play/17077}{My
math teacher at Case and a difficult
problem}}{My math teacher at Case and a difficult problem}}\label{my-math-teacher-at-case-and-a-difficult-problem}

I gravitated toward mathematics. And there was another reason. Our
mathematics teacher was very, was a very eccentric guy, and also very
hard, my second year mathematics teacher, also hard to please, and he
had quite a reputation at Case, because, a couple of years earlier, he
had flunked the entire class. He decided that none of them was learning
anything; he gave F to everybody in the course, so Louis Green was a
legend at Case.

And I was taking his class, as a sophomore, called ``Basic
Mathematics'', which he had written the textbook for himself. And that
was a course where you, stuff like, lot, things that are different from
the continuous things that physicists study. And there must, you know,
there must be something in the way I grew up that made integer numbers
more appealing to me. I, I mean it's associated with computing, as
everyone knows now. Of course computers didn't exist, or hardly existed
in those days.

I'll speak more about my first view of a computer later, but the, here
we are, in Louis Green's ``Basic Mathematics'' class, and I'm getting to
a different kind of mathematics than calculus, and one day he gave a
problem to the class. He said, ``Here's a problem that I don't think ---
if anybody solves this problem, I'll give you an automatic A in the
class''. And the problem, now, it turns out, well, I can state the
problem, he said, if you, putting parentheses around into a mathematical
formula, if you have a formula with five variables, A, B, C, D, E. How
many ways can you put parentheses into that formula so that you are
combining two things at a time?

So you could say, A, you could say, parenthesize AB, and then
parenthesize C, and then D or E, or you could start with B and C and
combine that with A, and so different ways to do this. So if you start
out with \emph{n} of these variables, and you put, you combine with
parentheses, what, how many ways are there to do it? And this, by the
way, is something very dear to the hearts of computer scientists now,
because we call it the number of binary trees with \emph{n} leaves. But
Louis Green gave us, as a problem, as a challenge, could we determine
this number? And if anybody could, he said he'd give an A to them.

Well. I don't know to this day whether he knew the answer to the problem
or not, but I have found out subsequently that the answer was published
in the 18th century, and, and had a long history. And so these numbers
are so famous now that one of my friends, Richard Stanley at MIT, has
found 128 different interpretations of these numbers, parentheses is
just one of these 128 ways. And he's collected that many ways. In fact I
had the honor of discovering number 128 last year, when I was with him
in Sweden, but that's; anyway.

My sophomore year, Louis gives us this problem, and we all knew Louis's
reputation, so we figured, why work on the problem? He'll never give out
a problem that we could actually solve, why should we, you know, why
should we waste time on this silly thing? But it turned out that I was
on the football, I was in the band, actually, not the football team, and
our band was going to play in Detroit, at the football game, on
Saturday, but I missed the bus. I got up too late, so I was, so I found
out the bus had just left for Detroit, and I had a whole, and I had
figured I'd spend all, you know, a wasted day, all day in Detroit.

So I figured, okay, I'll work on Louis, I'll spend this day thinking
about Louis Green's impossible problem. And I got lucky, and figured,
and found the answer to it, and so I wrote it up on two sheets of paper,
and turned it in on Monday morning, and he looked it over, and on
Tuesday, he said, ``Okay, you get an A in this class.'' So I'm still a
physics major, but I took his math class, and so I cut class the rest of
the quarter, and he lived up to his agreement, and I got an A on my, on
the score.

Well, I felt a little guilty afterwards, having cut class, so I served
as his grader for his course the following year. But that, what was I
going to say? So, but that summer I switched into mathematics as a
major, because of my experience with the welding in physics, and because
I found that mathematics was something that I would be able to actually
prove, prove correctness, and that, this appealed to me. Still, I didn't
study mathematics that much, because I already had the A, and an A was
what I was looking for in my college grades.

\subsection{\texorpdfstring{\href{http://webofstories.com/play/17078}{My
interest in graphs and my first experience of a
computer}}{My interest in graphs and my first experience of a computer}}\label{my-interest-in-graphs-and-my-first-experience-of-a-computer}

How did I get into computers? I had a scholarship to Case, but it was,
it didn't cover my whole tuition; it just covered part of the tuition,
and so I took a part time job. My parents had no money, and I took a
part time job working in the Statistics Department. Taking, and one of
the things I would do, would run a card sorter, an IBM card sorting
machine, which was kind of a fascinating thing. You put the; take these
punched cards, and you put them in the thing, and it distributes into
different pockets, and then you pull them out in certain orders, and
afterwards look at the results, and you draw graphs. And so I was
drawing graphs for the Statistics Department.

I guess I should say something more about graphs, while it flashes into
my mind. In high school, I had taken time, one summer, working with; I
was fascinated by this idea of graphs in mathematics, where you have,
you know, as a function, as variable X varies, you have Y as a function
of X, then you draw the position that's Y units above the axis, and it
makes a picture. And since I like visual things, I was fascinated by the
idea that I might be able to take, start with the picture that I wanted
to do, and find the equation that would, when you graph the equation,
you would get that picture. And so I played around with graphs. I spent
one summer in high school, I had drawn hundreds and hundreds of graphs,
where I would take, where I would take an equation like the square root
of X? + 5, 5x, minus something else, and I would, and then I would draw
the graph.

And I had, and my dad had a little calculating machine, which was a,
where I, which could calculate square roots. It actually would print it
out. He was an accountant, so it would also print it out on tape, that I
could run this machine and it would do the multiplying and stuff for me,
and then I would have this function of, and then I would say, instead of
X? + 5x, I would maybe change it to X? plus 4x, and draw that graph too,
until I would learn how different graphs looked.

I didn't have calculus, I didn't know calculus in high school, but I did
know how to graph an equation, and that fascinated me, so I had played,
I worked so hard on this, in fact, on this orange graph paper that I
had, I started to get headaches, it was not easy on the eyes, and I
think I started wearing heavier glasses at this time, because I worked
on the graphs, but this had given me some experience with graphs, and
liked, I liked that kind of mathematics, even when I was in high school.

So now, I got my first, I got my first part time job at Case; I'm
supposed to draw graphs for the statisticians. So that's fine, and
downstairs from the sorting machine was a new computer, an electronic
brain, as they called it, in those days, and it was the first, it was
called the IBM Model 650. This was the, historically, the first computer
that was mass-produced; there were more than 1,000 of them. Before that,
computers had; no computer had been made more than a few dozen at the
most.

And this computer arrived, about midway in my Freshman year at Case, and
it was sitting in a room downstairs from the Statistics laboratory,
where I was working. So I could peer through the window at this
computer, and with its flashing lights it looked rather exciting. And
one day a guy saw me looking through the window, and he said, he invited
me to come in, and he explained to me how the machine worked, and so it
was quite fascinating to me that it could do things much different than
this mechanical calculating machine that my dad had shown me.

So I took a look at the operating manuals for the machine, and pretty
soon I, he allowed me to punch cards that would go into that machine, as
what; you know, I knew how to run a sorter, but now I could actually
punch a card that would make a computer program. And I, and so I began
to learn something about the inside of this machine.

\subsection{\texorpdfstring{\href{http://webofstories.com/play/17079}{Learning
how to program on the IBM 650 (Part
1)}}{Learning how to program on the IBM 650 (Part 1)}}\label{learning-how-to-program-on-the-ibm-650-part-1}

I read the manuals that came from IBM, and it had; the manuals had
example programs in there, and I thought of better ways to write those
programs. I thought of, you know, well, okay, this program works, but if
you did it this way, it would be even better. And so that's given me
some confidence that maybe I had a talent for programming. Now, if the
manual hadn't had these bad examples in it, I probably would not have
gotten interested in programming, because I wouldn't have this
confidence, and I would have been scared and say, oh, I would never
think of this.But the fact is, the manuals were pretty stupid, and
that's what gave me the confidence that I should think a little more
about programming, because I might be, you know, I might be good at it.

So, you know, I started to learn a little bit about computers in my
Freshman year, then when I pledge at the fraternity, one of the
fraternity brothers had a problem that he needed to solve, and he was
eval --- finding the roots of fifth degree equations. And I had learned
about a program called the Bell Interpretive System that was, that
looked like it wouldn't be too hard to solve this problem on, and so,
since he was, you know, he didn't really want to do it himself, so for
my fraternity brother I wrote a, I wrote something that would, could do
one of his homework problems for him, and that was, and it seemed to run
correctly the first time. I find it hard to believe, now, but it, as far
as I know, it did.

And so he was happy about that, and I, you know, and I got more involved
with the people at the computing center, where, when I, when they stared
letting me use this IBM machine during the night. And so between my
freshman and sophomore year, I was, I had a part time job at the
computing center, working for them. Really, they allowed us to write
software that other people, other students, would be using.

The first computer program I wrote, and it was the following; I had to
find the prime factors of a number. So you would dial in on the console
of the machine; there were rotary switches, and you could set the switch
to some number, like 100, and then it would punch out 100 is equal to
two times two, times five, times five. So we find the prime factors of
the number you dial on the console switches.

I still have a copy of my program for that, that I wrote, and it started
out something like 60 or 70 instructions in the program originally, and
two weeks later I had the program working correctly, and it had about
130 instructions in it, and I think I had removed more than 200 errors,
during the time. I mean I not only learned about programming, but about
making errors. So what can go wrong when you? Well, anyway, I did, I
didn't really know that much about programming, at the time, but I
learned it a heck of a lot by this exercise of trying to find prime
factors of a number.

\subsection{\texorpdfstring{\href{http://webofstories.com/play/17080}{Learning
how to program on the IBM 650 (Part
2)}}{Learning how to program on the IBM 650 (Part 2)}}\label{learning-how-to-program-on-the-ibm-650-part-2}

So I learned certain things like, what's; this was a decimal computer,
it worked not in the binary system, but in the decimal system, and you
had ten-digit numbers, so I could learn, so it was also very slow, the
division instruction in the machine took four milliseconds, that's a, I
think it's something like four nanoseconds, in other words now machines
go a million times faster. But you know, incredibly slow by today's
standards, but, so you couldn't do that many divisions per second.

And my method of finding prime factors was just to try dividing you
know, you can divide by two? No. Divide by three? No. Divide by four?
No. Until you find a factor. So now you take the largest ten-digit
number that doesn't have any factors. This program would take a long,
long, long time. So one of the things I had to do was make it go a
little faster. I wouldn't divide by two and three and four and five, I
could skip over the even numbers if it's not divisible by two, it's not
divisible by four.

And, you know, I had to do things like that in order to, then I, as soon
as I got up to a five digit divisor, then I could stop; I didn't know
then back then, you know, I didn't realize that at first, but I didn't
have to divide by every possible thing, because if it has a divisor at
all, the smallest divisor has to be less than the square root of the
number you're looking at, less than or equal. I think I first thought
less than, and then I, then I found I had to change my program to less
than or equal.

One of the most subtle bugs was, and it took me, and it took me a hard
time to do it, was the following. What if the number had lots and lots
of prime factors? Well it turned out there are ten-digit numbers that
have, I could only punch eight prime factors on a card, on the answer
card, and so I would have to prepare my program, because I mean you can
have more than 30 prime factors, so I could, so I had to change my
program so that it wouldn't only punch one card as the answer, but it
would also punch up to four cards.

So anyway this was, I'm just trying to explain why this little program
of finding prime factors was so instructive for me at the time, and I
did it near the end of my freshman year, and I was allowed to spend all
night sitting at the machine, turning the dials, turning the buttons,
and Case had an extremely intelligent attitude toward undergraduates.
They allowed us to go and touch the computers, do everything ourselves,
work overnight, sleep in the, you know, in the room, and write programs
that would be used by other people on campus.

And Stanford had a completely different idea. If you used the computer
at Stanford, I learned later, they had a professional staff that had
been sent to them by IBM, of scientists who would do the; you would
submit the job to them, they would put it through the machine, and you
know, you'd get your answers the next day.

Case, all hands on, we are allowed to all that stuff ourselves, and
even, they didn't worry that we were going to break the machine, you
know, we'd learned how to open up the panels, and you know, when paper
jams and cards jam, and things like this, or, we could wire the boards
and all that stuff. So what if we're freshmen? That's okay. And I think
Case and Dartmouth were the only two universities that were so liberal
for allowing undergraduates to play with the machines in those days.)

\subsection{\texorpdfstring{\href{http://webofstories.com/play/17081}{Writing
a tic-tac-toe
program}}{Writing a tic-tac-toe program}}\label{writing-a-tic-tac-toe-program}

So I got my summer job at, in the computing center, so I didn't come
home to Milwaukee, except for a short trip that summer. And this was
before I had met these girls I was telling you about, in my sophomore
year. So I had only the computer to be with, and my second program was
to change the numbers from decimal to other bases, but my, but that was
a fairly quick.

My third program is the one that I spent the most time on at that time,
was to play tic-tac-toe. Now I found out later that a lot of computer
scientists have worked on tic-tac-toe, Charles Babbage, the famous guy
who, you know, was planning a machine that could do tic-tac-toe in
England in the 1800s, Danny Hillis built a tic-tac-toe playing machine
out of Tinker toys, that went into the computer museum at Boston. But
anyway, I and, I decided I'd write a computer program to play this
children's game, and it was a bit of a challenge.

I wasn't using Tinker toys, but this IBM 650 had another interesting
feature; not only was it decimal, with ten-digit numbers, but it only
had 2,000 altogether. So the total memory of that computer was, oh well,
let me see, 2,000 times, well what have we got? Five bytes or so? So
that's ten K bytes of memory. Now, right now, if you don't have ten
gigabytes, oh well, ten megabytes, you know, you're dead. You can't load
Microsoft Windows unless you hundreds of megabytes, but here we had
10,000 bytes, total.

And so I, but I was wanting it to play tic-tac-toe against me. I wrote
three versions of tic-tac-toe. One was an expert version, which you
know, I pre-planned a strategy that I knew was going to be right. What's
the second version? I can't remember. But the third version was the most
interesting. This was the learning version, where the machine starts out
knowing nothing. And it learns by experience. So it, so it remembers all
the possible positions on the tic-tac-toe board, just barely enough to
fit inside of ten K bytes, and every time it plays a game, if it lost a
game, it said, oh, these moves that I made were bad. The moves that the
other guy made were good. If it won the game, it said, oh, my moves were
good; the other guy's were not. So it would adjust; it kept; it had one
digit for every position, so it would start out like at four, and if
you'd win a game it might go five. If you'd lose, it'd go to three. And
so then if you had different moves, you'd it would try the ones that
looked good in its ranking, and that took me a month to write the
program, and I learned a lot playing around with that.

And afterwards I, then I tried the learning program against the expert
program, so I would use the expert program to play and train the
learning program, so how many times would the expert play against, you
know, like I think it was like 120 games or something, then the learning
program learned not to lose against the expert.

Tic-tac-toe is a kind of boring game, because if you really know how to
play it, every game comes out as a cat's game; it's a tie. Nobody wins.
But before that it's interesting. When you make bad moves, it gets
really exciting. So then I said, okay, now I'm going to have two
learning. I'm going to have the learning game play against the learning
game, so both of them start out knowing nothing about the game, and each
of them, so originally all their moves are completely blindfolded, and
you know, they're blundering along, but then at the end somebody happens
to win the game, or somebody happens to lose the game, so then their
strategy changes a little bit.

Well now it turned out that after about 350 games, they would learn how
to play conservatively, to draw against each other. It was a very dull
game that they finally wound up doing; they played safe moves instead of
brilliant moves, but it was anyway, a good learning experience for me,
writing this tic-tac-toe program.

\subsection{\texorpdfstring{\href{http://webofstories.com/play/17082}{Learning
about the Symbolic Optimum Assembly Program and the Internal Translator
(Part
1)}}{Learning about the Symbolic Optimum Assembly Program and the Internal Translator (Part 1)}}\label{learning-about-the-symbolic-optimum-assembly-program-and-the-internal-translator-part-1}

My family went to a summer camp in, on the shore of Lake Erie for a
week, that summer, and, Linwood, it's called. I revisited it a couple of
years ago, to remember. Anyway, it's on the beach, two weeks we were
there, and I would play tennis with my uncles and so on, but I also had
brought with me the, a couple of examples of computer programs that I
had heard about, and so I had some spare time while I'm at the summer
place, to look at these programs. And this made a big influence on my
future life, actually.

One of the programs was what they call an assembler, which was, meant
that instead of writing in the machine language, you could write in a
more symbolic language that would make it easier for, I mean I, when I
started my programming I was just programming everything in numbers. And
so, if I wanted to add, add a number in location one to the number in
location two, I would have to say something like, 20, oh, oh, one,
something, anyway, I would have to write all these numbers out, and
punch them into cards, and get it to work, and this was all I knew, when
I started programming.

But then the symbolic, the assembly program was a new-fangled thing at
the time, and allowed me to do it in a way that's more easier to
understand. Instead of choosing a number for every place in my program,
I could give it a name, and then the machine would figure out the, you
know, what number to associate with that name. And in those days
computers couldn't deal with letters very well, the computers were more
set up for numbers, but we were allowed to use capital letters, so we
had, so, and in this assembler we could use five capital letters, for a
word, and I remember specifically when I used this for my tic-tac-toe
program, and I, and I would have to think, what five-letter word would I
use to indicate a current part of the program, and I remember, you know,
with delight, when I want in my tic-tac-toe program, when it wins, I
would go to location BINGO. So I could use the five-letter word B, I, N,
G, O, for that part of the program.

Okay, so I'm learning about a symbolic assembler at this time, and so I
got the code, all right? So I'm using the assembler, but I was wondering
how it did it? How did you know, how, what went on, behind the scenes
that would take my symbolic thing that said, BINGO, and put it into
numbers that the machine would understand directly? And so I had the
listing for this program called SOAP Two, by Stan Poley of IBM.

Then I had also brought with me another listing, from a program called
IT, or Internal Translator, and this was a new program; had come from
Carnegie Tech, later Carnegie Mellon University, written by four people
at Carnegie, and IT took an algebraic language, instead of a machine
language. So you're writing here; it's a; this is early days, this was
before FORTRAN or other high-level languages were known. I'm talking
here of 1956, 1957. The idea here then, you can write X = A + B. Ah,
well, you couldn't really write that because we didn't have plus signs,
so we would really write, X one, Z, X two, S, X three, where S stood for
plus, and Z stood for equals, and, you know, and every variable was X
one, X two, X three or something like this, but anyway, you could put in
algebraic formula on your card, and then the machine would figure out,
from that, how to compute A + B and store it in X, or whatever you
wanted it to do.

So instead of a numeric thing; instead of a symbolic version of a
numeric thing, it was algebraic, and you would put your program from the
IT language into the computer, and it would flash the lights for a
while, and then, punch, punch, punch, punch, and give you a machine
language, in the computer's language. Magic. I couldn't; I hadn't a clue
as to how this was possible at all, so I got a copy of the program that
they had used, to write this IT, translator, and I had a copy of the
SOAP program, the Symbolic Assembly, the Symbolic Optimum Assembly
Program.)

\subsection{\texorpdfstring{\href{http://webofstories.com/play/17083}{Learning
about the Symbolic Optimum Assembly Program and the Internal Translator
(Part
2)}}{Learning about the Symbolic Optimum Assembly Program and the Internal Translator (Part 2)}}\label{learning-about-the-symbolic-optimum-assembly-program-and-the-internal-translator-part-2}

I brought this to the summer place, and I spent nights looking at it and
psyching out how this program worked. And, wow, I found; first I figured
out how this IT worked, and so, oh yeah, that's how they can look in the
algebraic formula and convert it into instructions, but then, it was
terrible style. The program was kludgy. You would read it, and every
time they would do something, they did it the hard way. I looked at the
SOAP program, the one that came by Stan Poley, it was elegant; it was
beautiful; it was like hearing a symphony. It was, whenever it did an
instruction, the instruction was sort of accomplishing two things at
once. Everything fit together harmoniously; it was elegant code.

I said, boy, I'd love to write programs like this guy can do. And
conversely, you know, this clumsy, clunky code that came from the other
one, wow. You know, I can do better than that. So a couple of friends
and I wrote an improvement of IT, which we called RUNCIBLE; every
program had to have an acronym in those days, and RUNCIBLE was Revised
Unified New Compiler IT Basic Language Extended or something. We had
some reason for the word R U N C I B L E. But we, but mainly we had, we
wanted to redo that algebraic language in a way that was more elegant
and had more features, and so improve it in, you know, in lots of ways
and still stay within our ten K bytes.

And so that was how I spent my first summer at the Computer Center. It
turned out that after RUNCIBLE was done, we also, I wrote a user's
manual, for how to write, you know, how to use this program, and
curiously, this user's manual was then used as a text book for students
the next year. And so I was in the unusual position of taking a class
for which I had written the textbook, when I was a sophomore, one of my
classes in computing.

Now, RUNCIBLE, we revised it the next summer, and made it really, lots
of bells and whistles, and still with our ten K bytes, but we got a
floating point attachment, and we had some other things, so it was, so
my friends and I came up with better versions.

So here, this, and I also wrote a SOAP Three, I loved the SOAP Two
program we had from IBM, I wrote SOAP Three, which, which improved on
SOAP Two, and this was used as the assembler program for other software
development that we were doing. So here are Case is allowing about a
dozen of us undergrads to write software that's being used by the other
students and faculty of Case, and I, the, Fred Way, who is the director
of this program, was very fore-sighted, able to trust the students, and
allow, and you know, and to, and we had a, we had lots of fun talking to
each other about all these issues, and we did ---

New literature of journals --- the \emph{Communications of the ACM}
started up in 1958. All of a sudden we saw that there was, there were
people publishing ideas from other parts of the world, about how to
write programs, and we knew that we had already discovered a lot of
these things too, and we had some better science out there. So my second
technical publication, after Potrzebie System for \emph{Mad Magazine},
my second technical publication was about this RUNCIBLE; the method we'd
used in RUNCIBLE, to do the conversion of formulas into machine code.
And here I sent it to this magazine, the \emph{Communications of the
ACM}, which was, which we had just, began, just, had just begun to come
out, and I was totally naïve, not understanding anything about
scientific journals or publication conventions.

I had been seeing magazines; I knew what magazines were, and so on, but
no people, but there was no idea of credit for something like this. It
was just the story that was important. And so, when I wrote this up, I
was, I considered myself to be a spokesperson, spokesman for these guys
at Case Computer Center, who had been working together to create
RUNCIBLE, and I, you know, so I, so I wrote an article about the methods
that are used in RUNCIBLE. Nowhere in the article did I mention the
names of these other guys, who had been working on it, and I didn't know
that I was going to be getting credit for any of these ideas, I just
wanted to describe the ideas, and so this was a, you know, a, I learned
later more about, about the scientific conventions of publication, but I
was just, you know, I was just a journalist, and I was treating the
story as that. So we set the story right when the article was published
as part of my collected works a couple of years ago.

\subsection{\texorpdfstring{\href{http://webofstories.com/play/17084}{Adding
more features to
RUNCIBLE}}{Adding more features to RUNCIBLE}}\label{adding-more-features-to-runcible}

In those days, computing was so different from what it is now, not only
was the memory tiny, only ten K, and the speed was slow, but also we,
the way we wrote programs, well maybe as a result of that, the way we
wrote programs was something I would never do today, because in order to
pack it into this small memory, we had to do, resort to some trickery
that was almost impossible to understand, and therefore highly likely to
cause errors.

But one of the things we learned as, early in trying to write software,
is that the users of your software always suggest new features, and
they're never satisfied. You give them ten things, and they'll want ten
more, and so we kept adding features and features. That's called
creeping featurism now. But we had only this tiny machine, so how were
we going to pack more and more features, when there's no space for it?
And the answer is, we use more and more tricks.

So in this compiler RUNCIBLE, there were four versions of the compiler.
You could ask it to produce computer instructions for a machine that
did, that had the so-called floating-point attachment, or maybe your
computer didn't have the floating-point attachment, so you either had
floating point or not. If you didn't have the attachment, then you would
have to go through a slower routine that would simulate, and pretend
that the attachment's there. Then you could also ask it to compile
directly into machine language, or you could ask it to compile into the
symbolic assembler language. The symbolic assembler would be able to
produce slightly better final product, but it would take longer, because
after you got the symbolic, you'd have to take those cards and run them
through another program before you could run them. So we had, so either
floating-point or not, either symbolic or not, it was four different
possibilities.

So we wrote it, we wrote the program in such a way that there was a
floating-point set of instructions, and a non-floating-point set of
instructions, each of these we had exactly, let's say, 731 positions of
memory for, so you'd swap out those 731 for another, and similarly for
the symbolic and non symbolic, you could swap out two parts of the
program; each of those parts of the program had to be the exactly the
same size, in order to pack it all in.

Somebody asks for a new feature, or we want to extend the language a
little more, then we think of a way to do it for the floating point, but
then we have to think of a way to shorten the non floating point part of
our program, so we kept on revising this program until it was really
inscrutable. I mean every, all kinds of tricks were used for that. When
I'd have a constant that was used in one routine, I would also make sure
that it could be used in some other routine, for some other, completely
different purpose. Almost everything in the program had many uses, and
therefore a few months would go by, and we'd forget about these tricks,
and we would try to change something else, and something would go wrong.

So it was a very shaky, bad way to do software, but it was the way that,
the only way we knew how to do at the time, and it was, and people
couldn't believe that we were able to do with such a small computer, to
do as many things as it did.

\subsection{\texorpdfstring{\href{http://webofstories.com/play/17085}{Wanting
to be a teacher and why I chose to go to
Caltech}}{Wanting to be a teacher and why I chose to go to Caltech}}\label{wanting-to-be-a-teacher-and-why-i-chose-to-go-to-caltech}

I never thought of that as a career. That was my summer job at Case, but
there was no such thing as a career in computers, and computers were
just this weird thing on the periphery of society. What did I really
want to do, as my own career? When I was in grade school, I wanted to be
a grade school teacher. I wanted to, you know, if I was in sixth grade I
was thinking of myself as a future sixth grade teacher, seventh grade
and so on. And in high school, I wanted to be a high school teacher. I
always viewed myself as having, as being a teacher later on, and also
maybe with a part time job as a musician or something. So I got to
college, I wanted to be a college teacher, I got to graduate school, I
thought, okay, graduate school teacher.

But I never thought of computer science as a, as part of my career, it
was something that I could do in order to make money, to prepare myself
for a career that I'd heard about before. It was like, so I was going to
be a physics teacher, then I was going to be math teacher, when I
switched to math. And I went to, okay, so, when I finally graduated from
Case, I had been recommended that I should go to the West Coast, well,
no, I chose the West Coast for graduate school, because my family, we
would go; finally we had enough money to take vacations in the summer
time, and we could drive round, and so we had driven to the different
parts of the United States, and I fell in love with California, and so
when I applied to graduate school, I applied to Caltech, Stanford and
Berkeley. Basically, where should I go to, to do my mathematics?

And I was accepted to all three, and I had scholarships to all three,
and, but the, I was specially recommended to Caltech, because of
Professor Marshall Hall, who later became my advisor, since at Case we
had a visitor, Professor Bose, a great Indian mathematician, who had
just, who had been, introduced me to research, and he spoke very highly
of Marshall Hall.

Bose, at the time, was doing his famous work about disproving Euler's
conjecture about Latin Squares; it's an interesting subject, a several
hundred year old problem, where people hadn't been able to construct
patterns that are, that are useful in many parts of statistics and
combinatorics and Euler and this greatest mathematician of all time,
probably, or at least in everybody's Top Ten, in the 1700s, had
conjectured that there was no solution to this problem, and three or
four people had even proved that he was right, but there were mistakes
in the proof, and so finally, my professor, my Professor Bose and two of
his co-workers found that Euler was wrong, and that there were really
patterns of that kind. And so Bose got me interested in research,
because I was at the Computer Center, and he had a computing problem
that he couldn't solve, and he put me to work on Latin Squares of order
12, and I came back the next morning with five mutually orthogonal Latin
Squares of order 12, and this excited him very much, and that was my
third technical paper, I guess, was this paper with Bose.

But he recommended Marshall Hall, who was one of the leading
mathematicians in the combinatorial mathematics area, and so I liked the
idea of going to Caltech.

\subsection{\texorpdfstring{\href{http://webofstories.com/play/17086}{Writing
a compiler for the Burroughs
Corporation}}{Writing a compiler for the Burroughs Corporation}}\label{writing-a-compiler-for-the-burroughs-corporation}

During the summer, between Case in Cleveland and Caltech in Pasadena, I
had a summer job, of writing compilers, this algebraic, this software
that converts algebraic language into machine language. I had a job
writing a compiler for Burroughs Corporation. Burroughs was
headquartered in Pasadena, I mean the Burroughs Division that was
dealing with software was headquartered in Pasadena, and Case had
recently installed a Burroughs computer that I liked very much, during
my senior year.

Some people from Thompson Ramo Wooldridge approached me and said, Don,
we understand that you can write compilers. We're going to put in a
proposal to Burroughs to write a compiler for their machine, the
Burroughs 205, for a language called ALGOL, which was just being
invented at that time, the Algorithmic Language an International
Standard, that was supposed to replace FORTRAN, the most popular
language at the time. And so Thompson Ramo Wooldridge was, made a
proposal to Burroughs for I believe it was \$70,000, to create an ALGOL
compiler for the Burroughs 205. The people who made this proposal
really, though, were clueless about how to write compilers, so they
hired me, a Case senior, to do the job, because they had heard that I
knew how to do it. I had done this RUNCIBLE and a couple of versions of
other software at Case, so spring of my Senior year, they are showing
me, they're showing me about this machine that Burroughs had, and I
started playing around with the computer, and I had learned, got
interested in the project, but Burroughs turned them down; they didn't
give them the contract.

So after Burroughs said, no, they weren't going to do it with Thompson
Ramo Wooldridge, well I said, well, wait a minute, I've, maybe I'll
write to Burroughs on my own, and I sent them a letter and said, I'll,
you know, for \$5,000 I can write you an ALGOL compiler. Now \$5,000 was
a huge amount of money in 1959, 1960. I mean a college professor was
making \$8,000 a year or something like this. So I thought \$5,000 was
an incredibly rich thing, you know, to do. I said to Burroughs, no, I,
except that, I can do it in the summer, except I won't have time to
implement all of this ALGOL, I'll be able to do everything except
procedures. I'll be unable to, you know, which is actually the hardest
thing to do in a compiler, which is of subroutines, the ability to
extend the language.

Well, they wrote back, you, we can't have ALGOL without procedures,
you've got to put procedures in too. And so I thought about it and said,
figured out, oh yeah, okay, I see how I can do it, but you'll have to
pay me \$5,500, instead of \$5,000. So they said, okay. So I spent the
summer of 1960, that's after my, after graduating from Case, I spent the
summer writing this compiler for ALGOL, for the Burroughs 205, and I
wrote it, but really I had it only in pencil and paper, I didn't have it
ready to go, and then I would take it with me on, as I went out to
Pasadena, and work on it there when I was at the Burroughs plant in
Pasadena.

So on my way out West; I had a little Volkswagen, that I had gotten from
my parents, and I drove 100 miles a day, and got a motel, and sat down,
and wrote code, wrote software, and took 30 days to drive from Milwaukee
to Pasadena, every day writing a little bit of this software. Then I got
to Pasadena, and had all my code, had all my notes, and started putting
it on to punched cards, and you know, debugging it and by Christmas time
I had their compiler for them, and it was a machine that didn't sell
very well, so there weren't too many places in the world, but I heard
for the next ten years that people in Brazil were still using it a lot.

And so it was a, it was an interesting experience for me, but the most
important, from my point of view, was I had \$5,500, which was enough to
get married. And so in the summer of `61 Jill and I got married, and it
paid for our honeymoon trip in Europe, which was our first time seeing
the world.

\subsection{\texorpdfstring{\href{http://webofstories.com/play/17087}{Working
for the Burroughs
Corporation}}{Working for the Burroughs Corporation}}\label{working-for-the-burroughs-corporation}

I got to Caltech as a student, a grad student in mathematics. Again I
was scared stiff that I wouldn't be able to succeed, because the
students at Caltech were really, really selective they, at Caltech, the
undergrads, one out of three of them was first in their high school
class, and the other two out of three were very near the top. And so it
was awesome, the people there, but they only admitted 200 students a
year in Caltech as undergrads. Graduate students, we had 12 of us in the
math program, and they were excellent as well. It wasn't like at Case
where anybody could get into the Graduate School at the time.

So I, so I knew I had my work cut out for me, and I, again I went back
into a mode where I did more than I needed to for my classes, but I also
liked the people at Burroughs, so I took, and so I took a consulting job
at Burroughs, working with their software group, which I had great
admiration for. Their software group had, for another computer, the
Burroughs 220, had written one of the best pieces of software ever, an
ALGOL compiler for the 220, and I got to know the people who did that,
and learned a lot from them.

So I enjoyed joining their group, but I wasn't, but according to the
rules, I couldn't have my, I couldn't have a fellowship from the
National Science Foundation or from the Woodrow Wilson Foundation; I
couldn't have those, that financial support and not be a full time
student. Having, you know, it was a no-no to also be a consultant, also
be working, working outside Caltech. Wow, your education couldn't
possibly be any good that way. So I renounced those fellowships, and I
did become a consultant, and I would spend time over at Burroughs
several days a week, and for the next six or seven, seven or eight
years, all the while I was down there in Pasadena.

The people at Burroughs were splendid to work with. I was in a group
called Product Planning, which designed, had designed, or was, early on
was in the process of designing a completely new kind of computer, one
that would do in hardware what, what we had been doing in software
before. That kind of computer is, is not in, it's been realized later
that it doesn't really provide a real good cost benefit trade off, it's
better to build cheap hardware and good, and then make software than to
build expensive hardware, but at that time it was you know, it was not
at all clear what the future would say about that issue. And so we had
the most complicated machine ever built, the Burroughs 5000, then the
5500, 6500, that came out later.

And the Product Planning group where I was consulting was the group in
charge of specifying the machine. Another group, the Engineers, had to
build the machine. It was completely different than any machine done
before. And my role as a consultant was to talk to the engineers who
were building it and make sure they understood what the designers on the
other floor knew.

And so I got, really, to talk to almost everybody in the company, and my
role was more of a communications role than anything else. These people,
if they went through channels, they would have to go several levels up
the hierarchy, and down again, and they couldn't talk to each other, but
they could talk to me, and then I could talk to the other people in the
other group, and so I, and so I could provide a valuable service to
Burroughs, and they, they would pay me, I think, you know, I think I was
getting \$5 an hour, or \$7 an hour; considered high at the time, but it
had no medical benefits or anything associated with it, so from their
point of view they were getting a good deal too.

And one of my roles was to check out the designs of the engineers, and
see if there were any mistakes in them before they would build the
machine. They would first take their designs and put them through a
computer program to see if they would simulate, and apparently work
correctly, but then, after they thought it was working correctly, then
they would show it to me, and I still found several hundred errors in
the design that I was able to catch, well you know, very weird cases
that could come up, but hadn't arisen during the simulations, so that
the expensive corrections to the hardware weren't necessary.

You know, really my role in that time, well, communication and trying to
find errors, you know, if I could look at somebody else's program, or
design, and couldn't find any mistakes in it, it was a rare event, and I
would feel bad. I mean finding errors was my, was a big treat for me in
those days, so far as my work in computers was concerned.

\subsection{\texorpdfstring{\href{http://webofstories.com/play/17088}{Burroughs
Corporation}}{Burroughs Corporation}}\label{burroughs-corporation}

Burroughs was a company established in the 19th century, and was quite
pre-eminent in the banking industry, almost everybody used equipment
from Burroughs to do its accounting, and in the early days of computing,
Burroughs acquired the Electric Data Corporation, in Pasadena, and they
were one of the many companies putting out computers. And, and as I say,
not only that, they had very innovative computers, and I liked the
spirit of all the employees working there as well.

The --- later on, they were acquired by Sperry Rand, this was in the
'70s, I think, and became Unisys, combined with Sperry Rand. So half of
Unisys was from the Burroughs Corporation, and half of it was the
Sperry, which, the Rand, the Remington Rand, the people who had built
the UNIVAC computer. That happened after I was no longer in Pasadena.
And they continued to have banks as their primary customer base, but
then, in more recent years, the, you know, other companies survived, and
there were many, many changes in the computer industry, to that time.

So Burroughs was, you know, Edsger Dijkstra was also a consultant to
Burroughs during the 60s, and he would come out to Pasadena, visit
periodically, and it was a great company to work for in those days, I
think.

\subsection{\texorpdfstring{\href{http://webofstories.com/play/17089}{My
interest in context-free
languages}}{My interest in context-free languages}}\label{my-interest-in-context-free-languages}

I had no idea that that work connected with mathematics in any way
whatsoever. I wear one cap in the computer --- when I'm consulting for
Burroughs, and I wear another cap when I'm at Caltech as a student
learning mathematics. Mathematics was something where we proved things
correct; we knew what we were, you know, we knew what the rules of the
game were. In the computer field, we just fiddled with something until
it seemed to work and we couldn't find any errors any more. But we never
had this idea that it could be mathematically correct.

The only small exception to that was the area of syntax of programming
languages. This means the grammar of languages. I mentioned that when I
was in Seventh Grade my friends and I loved grammar, and we learned
about English grammar. Well, now I was seeing the same kind of things,
they were not nouns and verbs, but similar things in the algebraic
languages like ALGOL, that I was supposed to write software for, and a
theory had been developing called the theory of context-free languages,
that was appealing to me, because here was something that I could be, I
could use my mathematical cap and my and my computer cap at sort of the
same time. You know, my computer science intuition, and my love of
grammar and language was; would suggested interesting problems, the
mathematics that I knew suggested how to solve those problems. So I,
that was one thing where the two worlds, the computing and the math
world, were coming together for me.

And I have to say that on my honeymoon, when Jill and I sailed to
Europe, I brought along with me a book by Noam Chomsky, which was one of
the pioneering things about context-free languages, and I read that in
odd moments, you know, when Jill was seasick or something, and I would
try to solve the problem about context-free languages. The problem I
tried to solve was, is there a way to test whether a context-free
grammar is ambiguous or not? Ambiguous means that you could write a
sentence that had two different ways to be understood. And I thought I
might have a way to resolve this, and I reduced it to several other
problems

But I couldn't solve the problem, in general, and little beknown, you
know, I learned several years later that in fact three Israeli
mathematicians had already proved that the problem had no solution; that
there was no way to solve this ambiguity problem, in finite time. But I
didn't know that during my honeymoon, and I just wanted to mention that,
although I do love Jill, there were also other things that I love too,
and one of them was the mystery of context-free languages. Another time
I guess I should, I have to mention is the time when I forgot about one
of our dates, and, when I was playing with the computer at Case Tech,
but we won't talk about that. But she doesn't let me live that one down.

\subsection{\texorpdfstring{\href{http://webofstories.com/play/17090}{Getting
my PhD and the problem of symmetric block designs with lambda equals
two}}{Getting my PhD and the problem of symmetric block designs with lambda equals two}}\label{getting-my-phd-and-the-problem-of-symmetric-block-designs-with-lambda-equals-two}

Here I am then, earning money as a computer person, and enjoying, and
then studying mathematics so that I can be a math teacher as a career.
And the idea that there could be any connection with this, or there
could be, ever be an academic field of computer science or something,
never entered my mind whatsoever as a possibility. Very few people in
the world had thought of it at that time either.

Now, at Caltech, I --- somebody had told me early on in life that you go
to grade school for eight years, and you go to high school for four
years, then you go to college for four years, and then you might want to
go to graduate school for another three years, and get your, and get a
PhD. I don't know who had told me that early on, but somebody had said
that it takes three years to get a Ph.D.. And I had believed that and so
when I got to Caltech in 1960, I'm thinking, oh, okay, in 1963 I'll
graduate. And I've just sort of set my sights that way. Now if somebody
had told me it was going to take five years, I'm sure I would have
graduated in 1965, but I had always been thinking I would graduate in
63, so I, so I just sort of planned ahead.

It was, it didn't occur to me, until the day of graduation, that none of
my other colleagues were there. You know, Al Hales was there, but just
two of our 12 had done it in three years, and I think it was partly
that, just that we had set ourselves up that way, to do it.

My thesis, I was working with Marshall Hall, who I hadn't met before
coming to Caltech, but I got to enjoy. His lectures were very
disorganized, but he really knew the subject, so he would be teaching us
about permutations, what actually I found was that I would take notes of
what he said during class, and the problem was to figure out a
permutation of what he had said that made sense. What, how to re-order;
you know, I mean every important idea had been presented some time
during that hour, but they didn't come to him in the right order, just
like now, I'm not thinking of everything in the right order, well he,
you know, so his classes were like that, but I learned a heck of a lot
this way, by trying to unscramble the notes of what he said in class, so
I learned about the things that he was good at.

And I decided that you know, one of the unsolved problems that he posed
to us, technical term is symmetric block designs with $\lambda=2$. It's a
technical term that really means; it's something like geometry, except
instead of having two, instead of having one line through every two
points, you have two lines through every two points. And you know,
instead of saying that every two lines intersect in one point, every two
lines intersect in two points.

So this is a system called the symmetric block designs wit$\lambda=2$, and at
the time I started working on this unsolved problem I was going to try
to find infinitely many such designs, such, symmetric designs, and six
were known altogether. And so I thought, okay, but I had a new way to
approach the subject, maybe I'd be able to find infinitely many. Well,
it, it's a good thing that I didn't take that as my final project
though, because I believe up to this day only one more has ever been
found, so nobody knows that there is only finitely many, but it seems to
be extremely hard to construct these designs.

\subsection{\texorpdfstring{\href{http://webofstories.com/play/17091}{Finding
a solution to an open problem about projective
planes}}{Finding a solution to an open problem about projective planes}}\label{finding-a-solution-to-an-open-problem-about-projective-planes}

Another open problem had been mentioned in one of my classes, having to
do with projective planes, a finite kind of geometry, and the projective
planes; one of the things Marshall Hall was good at was studying
projective planes, and he had developed some of the major theories of it
during the Second World War. These have lots of applications in
code-breaking as well, and other, and kinds of code making from, you
know, transmitting reliable codes.

And I took his class on projective planes, and one of the things he
mentioned is that no, that only one projective plane of order 32 is
known, and only one is known of order 64, and so, for whenever you had a
power of two; two times two times two times two times two, and you'd
find, one of the projective plane of that size, basically, that, it was
a question whether it was only; that projective plane was unique, or
were there other kind of geometries with this size? And I, the first
open case was of size 32. And I took a look at it, and I received in the
mail, as a result of a computer program that had been written, by R.J.
Walker, I think he was in Princeton, and he had found all of the --- he
had actually found new ones of size 32, by computer, and he, he had a
list on this, of all of this, all of the projective planes of a certain
kind, which I later called I think a semi-field.

And he had two lists, and both of the lists had 16 solutions on it. And
the thing was, so this gave him 32 different projective planes of a size
32, 16 of them were of one kind and 16 of them were of another kind, and
that's all the computer, his computer found. But one of the 16 in the
second list was the one that had been known for years and years. So I
thought to myself, oh, all I have to do is find a pattern, a rule that
would take every one on the first list, and find its corresponding mate
on the second list, and then I will have a rule that I can take on and
solve the next problem, 64, the next problem 128, and so on, because my
rule that worked in the case of 32 would then be a general rule if I
could find this pattern.

And so I had gotten this list from him, it was in my mail at nine
o'clock in the morning, and I remember riding up in the elevator with
Professor Olga Todd, who was one of my professors, and I said,
Mrs.~Todd, I think I'm going to have a theorem in two hours. I'm going
to find a way to match these 16 with these 16. And well, I just had a
hunch that it was possible, and sure enough, you know, staring at it a
little bit, a little bit, I found a connection, and by noon I had a
theorem that had solved many, many cases of this open problem about
projective planes.

I showed it to Marshall Hall, and he said, well Don, this is your
thesis. Write this up and get out of here. You know, you don't have to
wait and do, you know, do this other hard problem, just do this for your
thesis. So I felt a little guilty of solving my Ph.D.~thesis in two
hours and so I had, you know, I spent another few months refining the
result and adding on some related theory.

But basically I could write a thesis of about 70 pages, and then, and
that solved the problem that was considered by people in this very small
sub group of the world, who were projective planologists, the finite
projective plane people, that this was one of the problems that they had
thought might never be solved. So I had, so that was nice. And it gave
me a thesis.

Now, I, in order to graduate, I also studied --- you know, Caltech had
other requirements that I had to fill, fulfill too, but one of them,
interestingly, had, was that you choose some other classic problem of
mathematics, and you see, and you prepare for a month, read up on it,
everything you can, and see if you can say something new about the
problem, and the problem that I studied was what's called the
Thue-Siegel-Roth Theorem, in which Freeman Dyson had made one of the
main contributions, and I mention that just because I know Freeman Dyson
is interviewed in this series of People's Archive, and he's become a
good friend since then, but I read his papers on the subject, when I was
a grad student preparing for; I think he had recently, I think he
graduated about ten years before I did.

\subsection{\texorpdfstring{\href{http://webofstories.com/play/17092}{Teaching
a computer course at Caltech and being asked to write a book about
compilers}}{Teaching a computer course at Caltech and being asked to write a book about compilers}}\label{teaching-a-computer-course-at-caltech-and-being-asked-to-write-a-book-about-compilers}

In my third year at Caltech I was also asked to teach some classes about
computers. A group of people said, you know, Caltech doesn't teach
anything about computers, and we know that you're consulting to
Burroughs, why don't you, why don't you think of giving a course, just
to, to offer to Caltech? So I had also, then been giving a once only
course, even before I graduated at Caltech, and they made the very
unusual decision to hire me as an Assistant Professor after graduation.

Usually a university won't hire its own graduates, except MIT. But
usually, you know, it's considered bad to have inbreeding, because a
department can get bogged down in one philosophy, and you usually want
to bring in new blood. Well, Caltech, I guess, felt that I was
sufficiently strange of blood that they, that it was okay to hire me
too.

Now, meanwhile, in the January of 1962, I'm in my second year of
Caltech, and in my first year of marriage; we got married in the summer
of '61, so Jill and I had six months of wedded bliss, we started with
our honeymoon, and then, and then we had a time before I was approached
by Addison-Wesley to write a book about computer science, about
computers. And in the January of `62, an editor from Addison-Wesley took
me out to lunch, and said, Don, we want, we need, we would like to
invite you to write a book about compilers.

You know, compilers, this is this system, thing that I had done for
Burroughs the previous year, and I had, I just finished, and you know,
you have been recommended to us as somebody who knows how to write
compilers, and would you think about writing a book like that? So I'm
one and a half years into graduate school at the time, and doing
consulting for Burroughs, but I, but, boy, I couldn't get the thought
out of my mind. Wow, I love writing a book? I just, you know, I'd been
working previously on newspapers, magazines, you know, writing a few
articles. I enjoyed writing all the time, and I, and here was the
publisher of my favorite textbooks, Addison-Wesley, was asking me to
write a book for them.

And so right away I went home and I jotted out the titles of 12 chapters
I thought would be good for a book. And then, well, our marriage was
still happy, but it was different, because I started concentrating on
this book, for the next 40 years.

I thought I could finish the book, you know, rather quickly. I have
letters that I wrote in 1964 or five. No, '64, I wrote a letter to
somebody saying, ``I'm sorry I can't visit Stanford University this year
because I have to finish my book before my son is born,'' you know. And
now he's 40 years old, and I still haven't finished the book, but that's
just; we'll get to that. But I thought I, I thought I would finish; I
had no idea how long it would take me to, you know, to write this book.

They asked me to write a book about compilers, but I, I thought, well,
wait a minute, there's a lot of other stuff goes on in computer
programming that you also need to know before you finish your compiler,
so I said, would you mind if I put in chapters about these other aspects
of computer programming? And they said, no, go right ahead. Okay, so
this book, we, I, we decided to call it ``The Art of Computer
Programming''. They liked that title.

My original motivation for writing it was not only that I liked the idea
of writing books, but because I could see a big need for such a book.
There was nothing like it.

In fact, although I had written several compilers, and I knew a lot
about compilers, I hadn't invented any of the ideas in those compilers.
I had just applied ideas that I had learned from other people. And so
everybody else I could think of, who was able to write a book about
compilers, I also, as far as I could see, they were pretty biased and
slanted. They would mention their own method, and they wouldn't mention
anybody else's method. But I was the only person I knew who didn't have
this axe to grind. I had never invented anything myself, I was just a
writer, I could present everybody else's idea, in a way that was
consistent, and wouldn't distort the picture the way they would, if they
wrote it. You know, anyway, this is in the back of my mind, when I'm
saying, yes, I want to write this book. I wrote the book because I
didn't have any; because I felt that I was, that I was fairly good at
writing, and that I would be able to balance the accounts of other
people who had had more of a stake in it would.

And of course as soon as I got started writing it, I naturally would
discover a few things too, and I had my, and I developed my own biases
and distortions. But, and so I didn't succeed in my goal, of making the
unbiased presentation, but I have to say that quite frankly, I did
believe originally that that was my main reason for writing the book, it
was needed, such a book was necessary, and I couldn't think of anybody
else who would be able to present the story fairly. Not that I did it in
fact fairly myself later, but at least I couldn't think of anybody else
who would do it.

And I began to, so I began writing drafts of the material, starting then
in the summer of '62. I had classes of, during the beginning of '62, but
I started drafting material for ``The Art of Computer Programming'', and
the course that I wound up teaching at Caltech, during my third year of
graduate school, was based on these notes that I had made, preliminary
to ``The Art of Computer Programming''.

\subsection{\texorpdfstring{\href{http://webofstories.com/play/17093}{1967:
a turbulent year (Part
1)}}{1967: a turbulent year (Part 1)}}\label{a-turbulent-year-part-1}

My son was a year, my daughter was born in December of 66, and my son
was born in the summer of 65, so, you know, he's one and a half years
old, and she's one month old, my wife, Jill needed a gall bladder
surgery, I was scheduled to be a lecturer, a national lecturer, to go
around the country for three weeks, in February, and I was also
scheduled to do other things like lecture for three weeks in Copenhagen
in summertime, in France in May, et cetera, et cetera. I had sent in the
manuscript for Volume One, it was, you know, being processed by
Addison-Wesley, and the galley proofs were starting to arrive early in
1967, but I was in the middle of writing Volume Two, and I was actually
almost exactly in dead center of it, I mean if it has 900 pages, I was
on page 450, or whatever it was, anyway, I was almost exactly halfway
done the Volume Two, and it was taking much longer than, you know, I
thought it should, and I was getting, it was getting very hard to, some
of the Mathematics was at the limits of my ability, but I've got this
deadline I've got to finish Volume Two, because I had already promised
that it would be done years ago. And that's only Volume Two, there were
supposed to be seven volumes. Okay, so I'm going around and lecturing,
my parents and Jill's parents came to help out around here with the
kids, but I had to leave for, I had to leave, and this lecture tour
meant that I was in a different city every day. And so in one way it was
terribly boring, because the small talk is the same in every city. What
can you say in the first, when you first meet somebody? So you know, you
get up in the morning, get in the airplane, go, somebody else, somebody
greets you, the faculty takes you out to lunch, they ask you, where are
you going next, Don? And so on, and how is your book? And then you give
your lecture, and then you sleep, and the next day, you're in Atlanta or
something. So, but on this trip, maybe it was my first time? No, one of
my stops was in Stanford, one of my stops, well, there were, yeah, one
of the important stops was in Cornell, and so, yeah, different places, I
was in Rhode Island, where there was a college, where nobody understood
Computer Science, at all, and the audience just sat there dutifully
listening to me and, you know, and not smiling, you know, and then I
left, and then I went to University of Pennsylvania, and the students,
and the overflow crowd, and everybody is with me and everything, and
it's an exciting lecture to give, and then I went to Cornell, and I had
two days there. It was a weekend, and Peter Wegner was there. As a
professor, I had met him before, in England, or no, he had come to
Caltech, anyway, so Peter and I went hiking around Ithaca, on the
weekend, and one of the big questions, issues, in Computer Science at
that time was how, it was called the Semantics of programming Language.
How do you describe the meaning of things in the language? We knew about
the grammar, but how do you get the meaning of the language? And I had
got this idea that a good way to define a meaning is, if you know the
meaning of little things, then you can use that to make, build up, the
meaning of larger things, and so on, and continue on, but there was
another approach, where you could start with the meaning; you could
start at the top, and sort of propagate that downwards, into the meaning
of the, into how the context affects the words of the sentence.)

\subsection{\texorpdfstring{\href{http://webofstories.com/play/17094}{1967:
a turbulent year (Part
2)}}{1967: a turbulent year (Part 2)}}\label{a-turbulent-year-part-2}

And so there was this combination between the inherited part of the
meaning and the synthesised part of the meaning. The part that comes out
of the, you know, from the small to large, and the part that comes from
the large to the small, and I said, you know, I can't decide which of
these two is better. And so Peter says to me, as we're walking in this
park, he says: well why don't you do both? And I said, well, this is
obviously ridiculous. You can't, you know, it; will be circular, if
you're trying to go down, and describe the meaning of the bottom in
terms of the top, and you're trying to describe the meaning of the top
in terms of the bottom, you get into a loop. It makes no sense. But
after I was arguing to him about this, for about ten minutes, I realized
I was shouting, because it occurred to me that he was absolutely right,
you know, that you could do both, as long as you were careful that the
way, that the aspects of the meaning you were defining from the top,
wouldn't depend on the ones that were coming up. So the meaning has
different parts to it. And so this led to a research, a sub-field of
Computer Science, called Attribute Grammar. And the idea came while I
was on this lecture tour in February. But I had- I get back home, I have
absolutely no time to work on Attribute Grammars, because I'm supposed
to write ``The Art of Computer Programming'' in all my spare time, and I
have students and classes to teach, and everything else, and kids to
take care of, et cetera, now I'm a father. Okay. Now, meanwhile, I had
also been thinking of another thing in Mathematics, called the Word
Problem, and here the question is, if you had two mathematical
expressions, can you prove that there's a way to transform one into the
other? And I had stumbled across a way to solve this problem that's now
known as the Knuth-Bendix Algorithm. Peter Bendix was a student in my
class, at Caltech, who worked out the computer program for it for his
term paper. And because of some work I had been doing, these ideas came
together and so I also, in 1967, besides these obligations of stuff to
do, I had these brand new ideas that were just waiting to be explored.
The Knuth-Bendix Algorithm, type of work, and the Attribute Grammar type
of work. But no time to work on, so it's the busiest year of my life.
And I'm editing journals. I was editor of 12 technical journals at the
time. I was getting papers to referee, and, you know, I was taking that
job conscientiously. The way I was operating, when I was at Caltech
would be, you know, well, okay, the kids, if, I'd take care of the kids,
you know, change the diapers, and so on, then they go to bed, if they
wake up and cry, I put in my ear plugs, this is my time to do my ``Art
of Computer programming'' writing. I watch TV, old movies on television,
while I'm writing chapters for the ``Art of Computer programming''. I
get to school, do my editorial work, send out papers to be reviewed,
write to the authors of papers. Every morning, I would figure out, what
am I going to accomplish this day? And I'd stay up until I finished it.
You know, I was used to all-nighters from high school, well I started
to, you know, to work every day until I had finished what I had set
myself to do that day.)

\subsection{\texorpdfstring{\href{http://webofstories.com/play/17095}{1967:
a turbulent year (Part
3)}}{1967: a turbulent year (Part 3)}}\label{a-turbulent-year-part-3}

So not too surprisingly, I suddenly came down with a bleeding ulcer, and
I was, you know, I was in the hospital. So I had all these other things
going on, no time to do the research that was coming up, and then it
turns out that my body rebelled. And the doctor showed me a book of his
that described what you might call the Type A Personality, which
described me to a T, and said, you know what? Don, you just can't live
this way, it's not sustainable. He was a wonderful doctor, he let me
read his own textbooks, didn't, you know, didn't just say, doctor knows
best, listen to me. He let me understand my condition, and I got shots.
He gave me shots to restore the iron that I'd lost from bleeding, some
more medication, told me to change my lifestyle, only do this much. So I
resigned from all my editorial duties. You know, I wrote to my publisher
with a black frame around the letter, saying, I'm sorry, I know you
wanted me to finish Volume Two this year, but I'm in the hospital; I
can't finish it, you know, I'm doing my best, and I'll continue working.
The day that this ulcer occurred, I can tell you exactly where I was in
Volume Two, because if you look in the index to Volume Two, there's an
index entry called Brute Force, and I was trying to solve an exercise by
Brute Force, at the time, and I just put that there as a reminder, so I
can know exactly where I was at the time when this low point occurred.
But then when I realized that I was doing too much, and I started, and I
could still, you know, change my lifestyle and learn about the concept
of mañana, and so on, Doctor allowed me to go to Copenhagen and give my
lectures there. These were lectures about yet another idea, about syntax
and semantics of programming languages, and it was to an international
summer school being held in Copenhagen. I hadn't prepared the lectures
in advance, but I sort of knew what I wanted to talk about. And so I had
a week, I got there a week early, so that I could prepare my lectures to
be given the next week. And in Copenhagen they have these wonderful
forests, and so I went and sat under a tree, and started planning my
lecture. Well, then it turned out that the other lecturer for the first
week, Niklaus Wirth, had called up, and he had been on a round-the-world
trip, and he had caught dysentery in India, and wasn't able to lecture,
so they moved up my lectures to the first week, from the second week, so
I had only, I can only be sort of a day ahead of my lecture. On Monday I
could prepare Tuesday's lecture, and so on, and go back to the forest
and figure out what I'm going to say on Wednesday. And those lectures
were presenting original material, that I had, you know, that hadn't
been done before. And I was recovering from ulcers, but I was taking it
easy, I was trying to, you know, do this all relaxing and so on, and I
actually enjoyed being in the forest under the trees, because it was a
nice way to do research. And that all worked out, the students were
helpful, and helped me solve some of the problems that came up in that
day, and so that worked out. Then I went to Oxford, for a conference,
where I presented my theorems about the Knuth-Bendix Algorithm, and I
was able to write that paper on the plane, for, during the second week
of when I was at Copenhagen, and on a trip to France, I had a couple of
days to think about attribute grammars. So this was 1967, and it was the
most creative time in my life, in the sense that three ideas that I had
that turned out to be important in the field of Computer Science, the,
certainly the attribute grammars and the Knuth-Bendix Algorithm, plus
the third idea about top-down syntax analysis, which is less important,
but still not bad, all came out that year, and I had to do them all by
stealing a few minutes of time here and there, from the problems that I
was really supposed to be working on, which is my family, my book, my
teaching. And so I've always wondered whether or not this is, you know,
would I have been so creative if I hadn't been under such strain? If I'm
designing a Research Institute, would the ideal design be something
where you have babies screaming, and people are sleep-deprived, and you
know, and are bombarded with responsibilities, and then they would
produce better research? Or where they, you know, have a luxurious setup
with comfortable surroundings and so on? You know, you read about like
Stravinsky's conditions, when he was composing his great music, and, you
know, he was in some garret in terrible circumstances. Why is it that
the year 1967 was the year when I had so many ideas? Maybe it was just
that the time was right for them. You know, you can't go back and change
history.)

\subsection{\texorpdfstring{\href{http://webofstories.com/play/17096}{A
new field: analysis of
algorithms}}{A new field: analysis of algorithms}}\label{a-new-field-analysis-of-algorithms}

In the Fall of '67 there was a conference in Santa Barbara of
mathematicians, and, and I, that's how I remember meeting, I think also
that, wasn't that the year when there was a conference in Chapel Hill in
the Spring too, of mathematicians. So, I, I was meeting a lot of people
that stimulated me and we had interesting problems to talk about between
each other, but when I got to the conference at Santa Barbara, I
realized that this was going to be my only chance to do research, so I
sat out the conference. I didn't go to any of the lectures. I just sat
on the beach and wrote my paper about attribute grammars at that
conference. But I went to the meals, and I particularly remember
somebody at that conference asking me what do I do, and I was just
deciding, you know, I'm going to become a computer scientist instead of
being in the math department. And so I said, well, I'm going to- I think
I'm going to be a computer scientist. And so he said- well what, are you
a numerical analyst? And I said no, not really, so he, he says, ah,
artificial intelligence, and I said- no, not artificial intelligence
either. Then he said- well, then you must be in programming languages,
and in other words, compiler is what I had been asked to write a book
about, and in fact, at that time computer science was considered to
consist of three things, numerical analysis plus artificial intelligence
plus programming languages. And, and Stanford's department also was sort
of organized along those three lines. There were three qualifying exams
and things like that. But I said, well, you know, I've done a lot of
research in programming languages, and I've been editor of the
programming language department for journals, but, but my main interest
is different. And so I realized I didn't have a name for what my main
interest was. And what is, what was my main interest? Well, see, as I'm,
by this time, I'd written 3,000 pages of ``The Art of Computer
Programming'' and typed part of it, and published, and almost ready to
publish, you know, reading the galley proofs or part of it. And, and it
turned out that what I, what I found was that we wanted the mathematical
basis for understanding the quality of computer methods. We wanted to
know how good a method was, whether it's twice as good as another one or
whether, you know, some quantity of things, you know, not just
qualitative of, yes, better, but how much better, some, some way to add
quantity to this. And so I, so I used that as the underlying unifying
theme of my books, is to find these quantitative, descriptive ways of
judging the merits of a computer method, and, but I didn't have a name
for it, and, and at this conference at Santa Barbara I realized that if
I'm going to, if I have to explain to somebody else what is, what is my
field, I'm going to have to have a name for the field, so I, so I made
up a name for the field. I decided to call it analysis of algorithms,
and I figured, okay, you know, I'm writing a book. I could use that to
explain what an analysis of an algorithm is. I, I talked to my publisher
and I said, let's change the title of my book. Instead of calling it
``The Art of Computer Programming'', let's call it ``The Analysis of
Algorithms''. Well, that didn't sell with their, with their focus group,
but, but I decided anyway that would be, that would really be my career,
to be, to focus mostly on the analysis of algorithms, meaning the
quantitative study of how good an algorithm or computer method is. And I
used that as when I was asked to give a lecture at the International
Congress of, of, what's it called, of Information Processing Societies
in, in 1971. My title was the Analysis of Algorithms, and I also was
asked to speak at a mathematical, International Mathematical Conference
in 1970 in France, and my title there was Mathematical Analysis of
Algorithms. So, I was trying to promote this term as a, as a buzzword so
that somebody, so that somebody would understand what it is that I do,
or that I like, that I hope to do. And, and I'm glad to say that after,
after ten years or so, I mean in the late '70s, it started to show up as
a category in the reviewing journals and there were books coming out
called ``Analysis of Algorithms''. In fact, even though I, you know, my
publishers didn't like that title, there's, there are quite a few books
now that, that have that title in them. And, and so, but I got to name
the, the area that I work in, but if anybody asked me what does it
really mean, I would just say, well, it means whatever I'm interested
in, you know, so I could change it for the first few years to, to suit
myself.)

\subsection{\texorpdfstring{\href{http://webofstories.com/play/17097}{The
Art of Computer Programming: underestimating the size of the
book}}{The Art of Computer Programming: underestimating the size of the book}}\label{the-art-of-computer-programming-underestimating-the-size-of-the-book}

I had the Table of Contents sketched out, the 12 chapters, from the
first day, and I started filling in, chapter by chapter, and writing,
and writing more material, and Computer Science is growing, and I, it
turned out I'm very bad, not only at estimating how long it's going to
take to do a job, but also I wasn't very, at all good at estimating how
large a book I was writing. I looked at my handwritten; I, my manuscript
was all written out in hand, and I looked at my handwritten- you know,
and my letters seemed a lot bigger than the letters in books, and I'd
looked at books, and I've certainly read a lot of books, and so it
seemed to me that what I had written would be a fairly reasonable sized
book. In fact I had, I got to the end of chapter 12, and after chapter
12, I think it was 64, '65, of this handwritten draft, and I had 3,000
pages. I had accumulated 3,000 handwritten pages. Which I still have the
manuscript. And while I was working on these 3,000 pages, I had written
to Addison-Wesley saying, saying, they said, how's it going? I said, oh
yeah, I'm writing, stuff is flowing, do you mind if I, you know, if I
make it fairly complete, I find these other materials? They said, go
right ahead. So I accumulated 3,000 pages, and then I took it to a type-
to my typewriter, and started typing. Now, I typed up chapter one, which
was 400 pages of typescript, or something like that, manuscript, that's
double-spaced. Incidentally, I got myself a, an IBM Selectric
typewriter. This was very new at the time. I think they said, they told
me later that I was the first private individual in California to buy
one, instead of a company buying one. This was the typewriter that had a
little ball, that would rotate and strike the page, and, but the
important thing to me was the touch was much better than any other
typewriter I had ever felt. When you hit, when you struck a key on the
Selectric typewriter, it would transmit a signal, saying to the ball, to
go this way, but you could strike several keys ahead, and the machine,
internally, would buffer this, and remember where the keys that you had
done, and then they would be sent to the ball, you know, you type t h e
real fast, and you can get to the e even before the t has finished
printing, but the, it's like, it was designed so that you could do that.
So I could type really, I mean like the first time I saw a Selectric
typewriter in an IBM exhibit, I typed, you know- now is the time for all
men, good men, to come to the aid of the party, faster than I had ever
typed anything before. I looked at it; it came out perfect. And so I
said, I've got to have one of these typewriters, and so I bought myself
a Selectric typewriter, and I had a, and I used it to type my thesis, at
Caltech, and I had, I had been a good, I had been, a, you know, a
keyboard person, I'd been playing piano for a long time, and I learned
machine shorthand the way court reporters do, and so I was, I had played
a lot with, you know, a saxophone player and things like this, so this
was just another, typewriter. I could use this very well, and so I
started typing it, and I typed up my chapter one of the 12 chapters, and
sent it to Addison-Wesley, saying, here is, you know, here is the first
chapter of my book. And then I got a letter from a person; he was the
person who, he was the person who actually had been the first editor who
came to talk to me in 1962, but this was 1966, I think, by the time I
had gotten to this point 3,000 pages, plus typing the chapter. And so
now I heard again from this same guy, but he had been promoted three
levels, in the company meanwhile, so now he was way up, and he was
saying, what's going on here? You've got this book. Don, do you realize
that your book is going to be more than 2,000 pages? You know, and,
what? I thought I had a six or 700-page book. I said, you know, I
thought, you know, I know, I've read books for years. How can you tell
me that this book, that this book is going to be so long? So I went back
to ``Thomas's Calculus'', the original book that I had loved in Freshman
in college, and I typed out, you know, they said, Don, you know, I had
felt five pages of my typing would go into one page of book, but they
said, no, no, it was one and a half to one. And I couldn't believe it,
so I took ``Thomas's Calculus'', and I typed out two pages, of
``Thomas's Calculus'' on my typewriter. Sure enough, three pages of
typescript went into two. So here I had a book that was more than three
times as big as I, you know, so I thought. No wonder it had taken me
this long to finish the darn thing. But then you know, they said, well
we, nobody will be able to afford this book, you know, all publishers
have their horror stories about the professor who sends them a
manuscript, you know, ten volumes about the history of the egg, or
something like this, you know, and it just lays a big egg. And so how
are they going to get around this problem?)

\subsection{\texorpdfstring{\href{http://webofstories.com/play/17098}{Deciding
what to do with the book and the success of the first
edition}}{Deciding what to do with the book and the success of the first edition}}\label{deciding-what-to-do-with-the-book-and-the-success-of-the-first-edition}

I fly to Massachusetts, and we have discussions about what are we going
to do? And so they say, well, maybe we can think of something.
Meanwhile, they showed the chapter to a few people, and the people liked
it, so they weren't so wary about it, but they said, you know, but I, my
editor at the time, Norm Stanton, I saw, I happened to notice at lunch,
the notes that he had written to himself, and said, terrific cost bind,
and things like this. You know, I mean he was trying to break the news
to me gently, but what was I going to do? And you know, he was
suggesting, they were suggesting, okay, you leave out this, you leave
out this, you know, don't give the answers to the exercises. Instead of
having professional illustrations, they'll just use the illustrations
that I had put in my manuscript. They said they were charming, and
things like this. Well, I thought they stunk. And I, you know, and I
said, no, I like Addison-Wesley because the quality of the books has
been so superb; the illustrations have been top of the line. This is the
reason I signed with you guys. And my editor said to me afterwards, boy,
you were courageous in there, standing up to the president of the
company, and all this. So they decided, well, maybe we'll publish it as
three volumes, and then they changed their mind again, they decided to
publish it in seven volumes. So we set up a plan to publish ``The Art of
Computer Programming'' in seven volumes. And that plan is still
officially there, but three volumes have been in print now for more than
30 years, and I'm working on volume four. Now, the first, but the reason
is because the subject- Computer Science didn't stop there, so the 3,000
pages that I had written described the state of Computer Science in
1965. Well, a few things have been learned since 1965, so we've got to
include those too. Well, the book went through several other stages.
First they said, you shouldn't put in the answers to the exercises,
we'll publish them separately, as a, you know, some people can order
them or get them as a paperback. And we'll reproduce them just from the
typescript. But after reviewers started reading the book, they said, no,
it's actually be better to typeset those answers and put, include them
in the book, and so when, in 1968, when the first edition really came
out, it was an expensive book, more than all the other books in Computer
Science, about twice, I mean it cost, I think \$32 or something for
Volume One. I forget what the price was; I could look it up, while other
books were selling for \$10, or something like that. And yet, in the
first year, it was adopted as a textbook by more than 50 universities,
and you know, so we came out with another printing shortly after, and it
became unbelievably successful, although it was not an easy read, it was
still, you know, it proved that there was a need for a book of this
kind, in the area. So that's the beginning of ``The Art of Computer
Programming''. And in 1968, in January, is when I got my first copies of
that book, the, since then, I believe it's something like three or
400,000 copies of Volume One have sold in English, and more than that in
other languages, so I couldn't believe how successful that was going to
be. But if I had, you know, if I had known in 1962 that I was writing
such a big book, and that I would still be working on it when I'm 68
years old, I would, I would certainly have not said that I would go
ahead with this. I thought that I was going to finish it before my son
was born in 1965.)

\subsection{\texorpdfstring{\href{http://webofstories.com/play/17099}{Writing
fiction: Surreal Numbers (Part
1)}}{Writing fiction: Surreal Numbers (Part 1)}}\label{writing-fiction-surreal-numbers-part-1}

This is, is a unique event in my life. It took place in early '70s. I
was, I had met John Conway, probably, certainly one of the greatest
living mathematicians. I had met him on a trip to the University of
Calgary, in '71, and we had lunch, and he scribbled on a napkin a new
theory that he came up with, which I thought was really terrific, and
well, it's a purely mathematical theory about a new way to define
numbers, not only the integer numbers and the fractional numbers, but
also infinite numbers and square root of infinity, and infinity to the
infinity, and infinity to the square root of infinity, and one over
infinity, and makes sense of all these numbers. So a year later I was on
sabbatical in Norway, and a thought came to me in the middle of the
night, wow, this theory is so beautiful it would be interesting to tell
the story, to write a book that where the characters in the book
discover Conway's theory, they find his rules, on a stone tablet, and
they decipher the tablet, and they develop all the consequences of this
tablet, so that they can prove things about infinity, and so on, by
themselves. The point being that this would be a way to teach how to do
research, that the students could, you know, not only learn what other
people had done, but how to do new things in Mathematics themselves. And
it could be taught, and it could be presented in the form of a story,
with characters discovering these things, by themselves. And so I
thought, this would really be a cool book to have, and it would
supplement, it could be used as a supplement; I was thinking of it as
actually high school teachers could recommend it to some of their
students, so that students could see the way Mathematics is discovered.
So, I got the idea of calling these numbers surreal numbers. You have to
know that real numbers had, is what we call our numbers that have
infinitely many decimal places, and so surreal numbers are even more so,
because they, there are surreal numbers between the real numbers. And
this was Conway's system, and I thought, so I had this title, ``Surreal
Numbers'', and I had the idea that a theory could be developed by
characters in a book. And I woke Jill up, and said- Jill, you know how
I've been working on ``The Art of Computer Programming'' for many years,
and I'm still not anywhere near being done with it? Well, I just thought
of another book that has to be written, what do you think? And I said I
thought I could write this one in a week, and really a week, because it
would be short. And so, you know, she said- okay, Don, go for it. This
is your year on sabbatical, we're in Norway, why not actually take your
week, and go do this project, you know, concentrate on that, and then,
you know, then you'll be happy, and we'll live happily ever after.)

\subsection{\texorpdfstring{\href{http://webofstories.com/play/17100}{Writing
fiction: Surreal Numbers (Part
2)}}{Writing fiction: Surreal Numbers (Part 2)}}\label{writing-fiction-surreal-numbers-part-2}

So we made the plans. I decided I would rent a hotel room for a week in
downtown Oslo, near where Ibsen lived, so that I could maybe get some of
the vibes from his soul. And we planned that she would come and visit me
a couple of times during that week, and we could have a little affair in
a hotel room. We'd never; we'd always wondered what that would be like.
And so I could be an author for a week, and then, you know, and we could
have this thing, fool the people in the hotel that here was this woman
visiting me to, you know, she's sneaking. So this was what happened, and
during the next few weeks I started, you know, I sort of wrote the first
page of this novel in my mind, several times, and I had, whenever I was
riding a bus or something, I would think of another sentence, so I check
into the hotel in January, and it was probably the greatest week in my
life. As I said, the time, there was kind of like a muse there,
dictating this book to me, and this was, I stayed in a mission hotel,
it's a cheap hotel run by strict, by volunteers of the Norwegian Church,
they probably were, but anyway we had our affair. And students from
America were staying at this hotel, from St Olaf University, in
Minnesota, and every morning, in Norway, at the hotels, we have a
Norwegian breakfast, which is a huge spread, where you have, it's not
like kippers in England, but it's, but we have certainly herring, and
many other, you know, eggs, and all kinds of other things, and so I
spent an hour every morning eating a very leisurely Norwegian breakfast,
which was enough to last me till dinner time, and listening to all the
American students talking to each other, and what they're saying to each
other, and picking up the, especially, you know, I hadn't heard American
English spoken for a long time. We'd been living in Norway for several
months, for six or seven months by this time. And so I could remember
some of the dialect, and some of the expressions, and my novelette is
told in dialog form, so it was important for me to, you know, to have a
sort of authentic accent to these characters. So I listened to what the
students were saying, and I, you know, ate a lot, and then I would go to
my room, and I would start writing the book. And the book is only
really, it's something like opera, where opera, you know, has a good
music, with a little bit of a plot, and my book was good Mathematics
with a little bit of a plot. And so I was developing the Mathematics
myself. I had remembered that Conway had written down his stuff on a
napkin; I had lost the napkin, so I tried to remember what he'd told me
a year ago, and I tried to reconstruct it as the characters in my book
were discovering the mathematics for themselves. I was discovering the
Mathematics from the same ground rules as they were under, and if I
would make a mistake, you know, I would think of something that didn't
work, the characters in my book would also then make the same mistake,
and then they would recover from it the way I had done. So I, so the
book was also an authentic presentation of mathematical research, in
this way, as I'm writing it. And so I would start up, and I would end
up, in the morning I would, well I usually had something to go on from
the previous day. The first day I had been thinking about the first page
of the book for some months, so I was ready to start the first page, and
I only had to pause when I got to page two, I sort of could dictate page
one. And then I would, you know, so I'd start writing, and then I would
try to do some more Math, and then I would get stuck. And I would run
into a block. There was nothing to, you know, which way to go next? What
to do next? And so then I would go outside, and walk around in the
streets of Oslo, and it was a wonderful snow free month at that time,
and so it wasn't too slippery to walk on the streets in January, and you
know, so about an hour, an hour and a half, just meandering around my
hotel in downtown, the solution to the problem would occur to me, and
I'd get back to my room, write it up, and you know, then have a
leisurely supper in the evening, Norwegian style, and then back to my
room and do some, do a little bit more writing, you know, get ready to,
and then as I said, I couldn't turn the light out because I had to
write; I knew what the next day's first few sentences would be, before I
could go to sleep. I was really into this book at the time, as you can
imagine.)

\subsection{\texorpdfstring{\href{http://webofstories.com/play/17101}{Writing
fiction: Surreal Numbers (Part
3)}}{Writing fiction: Surreal Numbers (Part 3)}}\label{writing-fiction-surreal-numbers-part-3}

I predicted that it would take me a week to do this book, well, after
six days I finished the last chapter, and on the seventh day I rested.
And it was very interesting. On the sixth day, actually, was one of the
days my wife came to visit me, and we went out to see a movie, it was
called ``Butterflies are Free'', with Goldie Hawn, I remember, and then
afterwards we walked around the King's Palace in Oslo, and there was a
wonderful hoar frost on the trees, and there's these magnificent trees
in the Palace grounds in Oslo, and I was just entranced, looking up and
seeing the snow-covered tree branches, against the dark blue sky. I'm
just telling you this was such a week for me, that, you know, having
this inspiration all the time, of writing the book, and then getting; I
was in the second last chapter at this time, and going back to my room,
I would finish the last chapter, but I'm looking up at the sky, and just
seeing these patterns of these ice-covered branches, and it's one of the
great moments of my whole life, seeing this, and Jill was on her way
back home to our apartment. Then I went back and finished the last
chapter, and that was the sixth day, and on the seventh day I rested,
but on the seventh day, I tried to write a letter, to my secretary,
Phyllis, who was going to type up the manuscript, telling her, you know,
how to do it. And I would start a sentence, and I would get to the
middle of the sentence, and I couldn't finish the sentence. I'd been
writing fluently for six days, and then I got to the end, and the book
was done, and my writing was done. I couldn't figure out how to write a
letter to Phyllis. Finally I got the letter done, and we sent the
manuscript off to her, you know, it's all handwritten. She typed it up
and sent it back to me a few months later, and I sent the manuscript to
John Conway in Cambridge, England, where he was at the college there.
Well, he politely pointed out that I had gotten one of his two rules
wrong, that the characters in my book had actually been working on a
different system, not the one, you know, his system was actually nicer
than the one that I had done. Well, okay, not so bad. We came and
visited him in England. We went to Cambridge; stayed at his house a few
days, at Easter time, and I learned about what I should have said, you
know, and when I went back then to Norway, I took another week, and this
time, not in Oslo, but in a remote valley one of the valleys that had
actually, during the time of the Black Death, had lost all of its
population, and there was a little rest home there, called Solheimen,
where a lot of people came to spend a week in summertime, and every one
spoke Norwegian there. We ate at common tables, and I would do my best,
by that time, to speak to them in Norwegian, but then, most of the time
I went through the whole book, and rewrote it with the rules that Conway
said I should have used from the beginning, and it came out better. And
I also had some other comments from a few other readers, who had
suggested, you know, like my wife was a, there were some romantic
scenes, and my wife was a consultant on those. And so on. So I took some
readers' comments in, and then another week's work, and then I had this
book finished, ``Surreal Numbers''. It's a completely different part of
my personality from the computer books. This turned out to be kind of a
litmus test for whether or not somebody is a Mathematician or a Computer
Scientist. Some people think they're the same, and some people, you
know, anyway, I know they're different, because the Mathematicians will
look at ``Surreal Numbers'' and think that it's really interesting, and
the Computer Scientists will look at it and say, this is my greatest
mistake, you know, why did I waste time on this project? Well, I just
got word from the publisher that they want to do the 17th printing of
``Surreal Numbers''. I'm supposed to send them the corrections, you
know, by next week. And it's been translated into eight or nine
languages, so there are some Mathematicians in the world who appreciated
this little book. It's about 80, 90 pages, I think, of, small pages, but
it turned out to be a little thing that you know, I felt was dictated to
me. I will never be able to do anything else like it, but it was a great
experience while it lasted.

Have you ever had the urge to do another novel?

No, but you know, while I was writing it, you know, I was, like my eyes
were open, and like everything; I'm walking on the streets of Oslo, and
I'm seeing the birds, and I'm hearing sounds, and I'm, you know, aware
of the world, in a way, because I'm writing something that means I'm
going to be writing, you know, about what my characters are saying, and
so I'm, that automatically makes me much more receptive to the world, so
that was another part of this experience, and I can see why a writer,
you know, can get a thrill from that, because it sharpens your own
perceptions as you're facing the task of trying to put something into
prose. Yeah. But no, I mean I, you know, maybe I'll wake up in the
middle of the night some time with, you know, but I think, I've always
thought of it as sort of a once in a lifetime thing, and as I say, some
people say it's my great mistake too.)

\subsection{\texorpdfstring{\href{http://webofstories.com/play/17102}{The
emergence of computer science as an academic subject (Part
1)}}{The emergence of computer science as an academic subject (Part 1)}}\label{the-emergence-of-computer-science-as-an-academic-subject-part-1}

When I was writing the first volume, I was at Caltech, and I was
advancing in the Math Department, I was still, you know, I was a
Mathematician, and in the early 60s, as I said, I never would have
believed that computer science, you know, under any name, that there
would be an academic discipline of this sort. However, at Stanford
George Forsythe had come over in the late 50s to Stanford, and he had,
and a very few other people had a vision, that there was something here
that was really ought to be a part of every university's curriculum, and
it became known as computer science, and he, I consider him the Father
of computer science, and a wonderful man, he figured out a way to bring
people to Stanford, to build the department, and he started, he
scrounged around the whole world. Everybody that he felt would be the
best in what he saw as this emerging field, he would invite them to
Stanford, and say, please come and join us here. And so he recruited me.
And operated on a shoestring. He was able to; he had the support of
Provost Al Bowker, who later became famous for many other things,
including Chancellor of Berkeley, but Al Bowker was in Stanford
Administration and very supportive of Forsythe's ideas and giving him
strategy on how to proceed. And so he was able to, even though there was
no funding for this, he was able to get somebody to be part time, here,
in Math, somebody is part time in Electrical Engineering, somebody part
time in SLAC, and more and more people, John McCarthy coming, so that he
had the largest concentration of good Computer Scientists anywhere
during the 1960s. And next week our Department is celebrating its 40th
anniversary, and we're going to be reminiscing about this time, it's
actually the 41st anniversary, because the Department began in 1965. And
it was, they had, I don't know, half a dozen students, or ten students
at the very beginning, and then quite a few later on, but it was, it
was, you know, a pioneer in this area of doing computing. Now, I learned
about the connection between computing and Mathematics. I said first I
studied grammar, and I liked the idea that the grammar of a computer
language could be studied with mathematical tools, and I got into
correspondence with Bob Floyd, who at the time was in Massachusetts, and
I met him in the summer of 62, after I had written another compiler. I
wrote a compiler for Univac Corporation, and met Bob in the fall of
1962, actually, and he introduced me to a brand new thing- that you
could actually prove things about computer programs. You could not only
fiddle with a program until it worked, you could develop a theory so
that you would know that the program would work. This was a revelation.
I'd never brought my mathematical side and my computer side together.
Computer side, I just was an Engineer, fiddling. So Bob said, no Don,
look at this. You can actually be sure that your computer program is
working, if you look at it the right way. This was a great revelation to
me, so I got very, I got friendly with Bob, I visited him in
Massachusetts, he visited me in California, and I began to see that
maybe there was a possibility for an academic subject of Computing.
Forsythe invited me up to Stanford; my sister-in-law was living in San
Francisco at the time, so it was a convenient time for us to come up and
take a look at the campus, which was of course beautiful, and George
said- you know, okay, Don, join us. Oh, I said- no, no, no. I'm writing
this book, and I'm going to finish and my, you know, my son's going to
be born. We've got to have some stability. But I knew that at some point
in my life I'd have to make a decision, where was going to be my
permanent home, instead of just, you know- at this time I'm an Assistant
Professor. Caltech actually promoted me to Associate Professor after two
or three years, and I should say something about tenure, I guess. About
that time there was a story in the ``Los Angeles Times'', that students
at UCLA were unhappy because a professor at UCLA hadn't gotten tenure,
and I'd never heard of the word tenure before, so I want to Marshall
Hall, I said, Marshall, what is this word, tenure? What does it mean,
that she didn't get tenure? And he said- well Don, remember when you got
your appointment form from Caltech last year, where it said you're
appointed Associate professor, you know? And I said- yeah? Well there's
a line on there that says, the ending date for your appointment,
remember that that line was blank? And I said- yes. Well, he says,
that's tenure. You know, your job goes on. Well I'd never heard of the
concept before, so I had tenure before I knew it existed, and I'm glad,
because a lot of students now are so worried about not getting tenure,
that it interferes with their professional development, you know,
they're spending more time strategizing about how to get tenure than
about how to do good science.)

\subsection{\texorpdfstring{\href{http://webofstories.com/play/17103}{The
emergence of computer science as an academic subject (Part
2)}}{The emergence of computer science as an academic subject (Part 2)}}\label{the-emergence-of-computer-science-as-an-academic-subject-part-2}

There I was at Caltech, but I knew that sometime, that I had to make a
decision, what shall I do for a permanent position? Because I had never
expected I would stay all my life at one place, but I didn't like to
move, so I knew there was going to be a time when I should move, and to
the place where I would stay the rest of my life. So where should this
be? My first offer for full professorship was to Norman, Oklahoma, and
then I was offered, that was at '65, or so, and then I got an offer to
go to Chapel Hill, North Carolina, but these were unsolicited offers. I
started to wonder where I should go, and of course I knew about
Forsythe's enterprise here, so I wrote to Bob Floyd, who was my best
friend in Computer Science, and said- Bob, what's your assessment about
the different parts of the world? And wherever I go, I'd like to be your
colleague, so the two of us could go the same place. You know, where
would you like to go, and I'll see if it suits me too? So he wrote me
this nice, long letter, you know, telling about all the places in the
world the way he ranked them, and he put Stanford up at the top of his
list, just as I had tentatively put Stanford at the top of my list. And
so he said, you know, if I wanted to go to Stanford, the chances are he
would like to do that too. So this turned out to be kind of a package
deal that when we were negotiating with Stanford later, we'd say, well,
you know, there should be room for two of us, if you're going to make an
offer for one. But I started to search for a permanent home, and I got,
and the main four places that made, where we went through interviews and
made offers and so on, the main four places were Harvard, Berkeley,
Stanford and Caltech. And this was, as I say, taking place in the middle
60s, in 1967, was when the main interviewing was going on, well '66,
yeah, about that time. The firm offers didn't come through until '68.
But during 1967 was when there was a lot of correspondence happening.
And so I could be a full professor at any of those four places, and it
was quite; each one was different. If I was a full professor at Caltech,
I would remain in the Math Department. I would be a full professor of
Mathematics there, and that would be my career. If I went to Stanford, I
would be a full professor of Computer Science; also at Harvard and
Berkeley. At Harvard they had almost nobody in Computer Science, and you
know, a very few students, very few faculty. My role at Harvard would be
to help them build it up. And it was a case where well Caltech and
Harvard, you see I knew that at Caltech they wouldn't have a Department
of Computer Science for a long time, it was a very much uphill battle
because this faculty at Caltech is so strong in Physics and in Chemistry
and Biology. The Physicists don't see Computer Science as important,
except for how it helps Physics. Biologists the same way, how does it
help Biology? They didn't see that the computer scientists might have
problems of their own that, you know, that are interesting, not just as
a service to other disciplines. And for this reason it looked to me
really unlikely that Caltech would ever want to have a Department of
Computer Science the way Stanford did. Harvard, the same story; at
Harvard they said they wanted me to be a, you know, Professor of Applied
Science, and help them build up Computer Science, but I knew that if I
went there, I'd be spending most of my time arguing for Computer Science
rather then doing Computer Science. And I hated that. I'm not a
politician; I don't like confrontation. I don't like trying to persuade
people of things, but I like them to come to the conclusions themselves,
so that although Harvard was tempting, because I knew that it would be
great to have a spokesperson for Computer Science at, where you have
some clout there, and people listen to you, I wanted to do Computer
Science instead of arguing for it. I had seen a lot of cases where other
fields, the Operations Research was a special case in point, where more
than half of the papers in the early days of Operations Research were
arguing for Operations Research than just doing Operations Research,
just presenting the beauties of Operations Research, but saying why we
need Operations Research. Well, I wanted to have none of that, so I
didn't, you know, sneeze at the offer from Harvard, but I decided,
really that that wasn't going to be the life I wanted. I also, then I
had Berkeley and Stanford, and I felt all around University of
California, Berkeley was the best university in the world from a total
standpoint of all fields, and very attracted to the idea of going there,
but they didn't have much in Computer Science, like Stanford did. Again,
my role would have been to build, more to build things up, and I could
live on campus at Stanford. This was a strong attraction to me, to be
able to build a house like I have here now, where I can walk or bicycle
to the pool every day and not have to worry about commuting. So I, you
know, Stanford had already so many good students, and so many good
faculty, I could come here and be one of the boys, and, well, men and
women; one of the people, and so I took Stanford as my top choice, just
as my, my, now, on the other hand, Stanford has standards, and I was
pretty young, so why should they appoint a full professor at this age?
And fortunately Volume One came out in print, just in the week when the
faculty had to approve my appointment. I mean the people in other
departments on campus; the Trustees and the other people who had to
approve of an unusual appointment to a young person for full professor,
and the book came out just in that week, when they needed, when that
helped. So they were all smiles, and I got my letter from Stanford
saying, yes, you can come here.)

\subsection{\texorpdfstring{\href{http://webofstories.com/play/17104}{A
year doing National Service in
Princeton}}{A year doing National Service in Princeton}}\label{a-year-doing-national-service-in-princeton}

But I didn't come to Stanford immediately. My first year at Stanford was
actually spent on leave of absence in Princeton, New Jersey. Now this
was the time of the Vietnam War, some of you might remember, and great
ferment on campuses, and I had been deferred from the draft because of
being a graduate student, at first, and because of being married, and
having bad eyesight, and so I was 4F for a while, and then I, anyway,
but I felt that I owed something to the United States for providing me
with the freedom to be able to study, and not have to carry guns and
defend my territory, so I thought it would be good to give at least a
year of National Service, so I said, well what could a person with my
skills do for the country? And the answer was, you know, I was recruited
into working for code-breaking, cryptanalysis, where people like
Marshall Hall, my advisor, had, and other people at Caltech were
prominent, because the Combinatorial Math that I was doing was useful in
making and breaking codes. So then I, so I went there, and I worked not
at the Princeton university, but at the Institute for Defense Analyses,
which had a building on the Princeton campus, called von Neumann Hall,
actually. And we, and I worked there on classified work with some
tremendously excellent colleagues, and the way we did it there was that
half of our time was supposed to be doing pure Mathematical Research,
and half of it was supposed to be doing this classified work that was
relevant to the code breaking, and in our case we were, most of us were
also especially good at computer programming, so we would try to make
computer programs that could help in code breaking. The work there was
secret; I still haven't been able to tell my wife what I was doing that
year. And although I met a lot of great people that year, I found out
that the secret life is not for me. I'm the kind of person that likes to
talk about what I'm doing, and explain and teach, so I worked hard at
helping that agency for a year, but then I knew that I should really be
an academic, and no more do classified work after that time. And so I
can't tell you much about that period, except that the people that I
worked with were super, and I published also some papers that people
that I worked with, and my roommate there, Ed Bender, and I solved a
famous problem in combinatorics where we were able to take what's called
plane partitions and find a reason why the number of plane partitions of
n has an amazingly simple form. The plane partition is; you can
understand it like this. Suppose you have 100 sugar cubes, and you want
to put them into a box, and you pack it solid against one of the corners
of the box, how many ways are there to do that? And a man named Percy
McMahon, M C M A H O N had spent many years of his life solving this
problem, and the answer that he got was that there was a fairly simple
formula explaining how many ways there are to put n sugar cubes into the
corner of a box. But nobody knew a reason why it should be so simple,
and Ed Bender and I were roommates, and we spent a lot of our free
research time thinking about this problem, and lo and behold we came up
with a solution. And Ed and I, Ed is the person who sort of most
perfectly blends his research skills with mine, in the sense that the
two of us together, I never worked together with any other person on my
life where we were sort of such an ideal team. That is, his strong
points would complement my weak points, and my weak points would
complement, you know, his strong points, my strong points would
complement his weak points and vice versa, and we could talk to each
other at the same rates and understand each other at the same rates, and
so here we are in the same room, solving the problem together, and we
each are going to take the next step, and we're getting somewhere. I
never met anybody else who, where we were so perfectly attuned this way.
In fact, it's so- such a good match that we're afraid to meet each other
now, because we know that if we get together we have a responsibility to
the world to invent some new idea, and that's too much of a
responsibility, you know, it exhausts us, so we haven't seen each other
for years. But when we were doing it, we were really hitting it off
well. And so we solved this problem about plane partitions during the
time I was in Princeton.)

\subsection{\texorpdfstring{\href{http://webofstories.com/play/17105}{Moving
to Stanford and wondering whether I'd made the right
choice}}{Moving to Stanford and wondering whether I'd made the right choice}}\label{moving-to-stanford-and-wondering-whether-id-made-the-right-choice}

People would ask me; I'm in Princeton, and people would say, well Don,
where are you from? What should I say? I'm on leave from Stanford; my
employment at Stanford had officially started, although I was in
Princeton. The reason was it gives me legal ability to acquire the lot,
where this house is, before somebody else snatched the lot, so, you
know, I have, otherwise I don't get a desirable lot, so I'm officially a
Stanford Professor, but not getting any salary, because I'm away at
Princeton. And so they asked me where am I from? So I say, I'm from
Stanford. And they say, oh, isn't that where the students just burned
down the such-and-such a building? And I say, well, that might be true,
but I actually have never lived in Stanford, you know, so I spent the
year in Princeton, being away from Stanford, then I finally came back
here and oh, it's scary, because, you know, I knew that I didn't, and I
only wanted to move once in my life, and I thought I had picked the best
place to move, because, you know, good library, good climate, you know,
I could live on campus, and you know, good students, you know, close to
San Francisco. Sounds great, but here I am, and I actually, you know,
wonder, now I've put all my eggs in one basket, so I remember you know,
driving, the first day I'm here, and we stop at a stop light over on
Stanford Avenue, and I said- oh my goodness, how many times in my life
am I going to have to wait for this stoplight? And I'm realising now
that I don't have any choices anymore. When you have options, when you
have many different places you could go, you can always say, oh, how
great that's going to be. But then you make a choice, and you've given
up all the other choices, and then you say, now, anything that goes
wrong can be an aggravation from this point on. And there were other
things about Stanford that I didn't know the answer to, but fortunately,
you know, all the answers that I really did learn, turned out to be
nice, so I've been very happy here. I came to Stanford then in the
summer or the late part of 1969, although officially I started in 1968,
and I think California made me pay taxes for that year, even though I
was in New Jersey.)

\subsection{\texorpdfstring{\href{http://webofstories.com/play/17106}{Designing
the house in
Stanford}}{Designing the house in Stanford}}\label{designing-the-house-in-stanford}

While we were in Princeton, every Sunday night, Jill and I would talk
about what was our dream house. In fact, our first date, when we, at the
library in Cleveland, we each, we started talking about our dream house,
and it turned out we both had the same dream house, so we both, I mean
that's one of the things that we were meant for each other, we had
thought of a bean-shaped house. Never got a bean-shaped house, but both
of us had been thinking about it, so anyway, we decided that we'd spend
every Sunday night, hour, two hours, making plans up for what kind of a
house we'd like to have when we move to Stanford. The kids were in bed;
we had two children by that time, John was born in '65 and Jenny in '66,
so we made notes, and we finally had it was, I think about every room,
what we wanted to do, and we came back here, and there are a lot of good
architects in the area, and we thought- we'd met a man named Jim
O'Neill, who we liked very much, and he listened, and he looked at our
notes, and he didn't laugh at them, but he gently pointed out that all
of our plans were completely impractical, and you couldn't build these
things, and all these triangular corners would just not work, that we
had planned. And he had very good sense of lighting, and how houses go
together, so we, so he was a very good listener, so he knew what we
really wanted, rather than what we had written in our plans. He was good
at boiling these down into reality. And so we started making plans. We
were, you know, just renting at the time, a house until we could finally
build here, but we had the lot already. Stanford has this wonderful idea
that you can get an 80-year lease from the university and live on
campus, and the only downside is that all your neighbours are Stanford
professors, you know, who are, tend to be a little kooky, but there we
are. Now, one of the big ideas we had is that we should have rooms that
are not all the same. On our honeymoon we had gone to visit baroque
palaces, and we had seen places where, you know, there are grand, there
are rooms that are large in scale, and rooms that are small, and so why
should you have all the rooms about the same size? And, you know, we
went to the, we went to Rome, and we saw Roman baths, and we said- oh
we've got to have a big bathroom, you know, bathrooms should be big. We
figured a bedroom should be small, intimate. So we decided, you know, we
said, let's have one big room for Jill to do her artwork, one big room
for me, to have a pipe organ. I can talk a little about pipe organs
later, if you ask me. But I was, by that time I had been playing organ.
I took organ lessons in Princeton, at Westminster Choir College, while I
was there. So it would be nice to, you know, have two big rooms, and
then small rooms, somehow, but mostly a variety, so that you wouldn't
have everything the same. Lots of other ideas about how we could make
sure that, you know, we could keep an eye on the kids. We didn't want to
have rooms where the kids could run around in circles, because we knew
that that's, you know, a danger, and so various other plans. So we're
talking to the architect, and he, and one of our key ideas that we
thought we had to have was a winding staircase, a circular staircase.
It's very romantic. You need a house with a circular staircase. Well, as
you'll notice being in this house, we don't have a circular staircase,
but that's one of the things that, when Jim drew up various plans, and
plans one through 12 all had circular staircases, but he couldn't make
the thing work. Finally we decided, okay, he's got this other one, plan
number 13, which is basically what we have now, where we have the one,
Jill's room on one side of the house, where it gets the north light, my
room on the other side of the house. In between, where everybody's, you
know, it's space for everybody, an expandable place for children. We
were hoping to have more children, and a place where students could
stay, various things all put together, in the other part of the house.
With the, now he wasn't as good at the small spaces though, as the big.
We didn't get the small rooms that we wanted, so later on we actually
built a partition in our bedroom, so that it, to make it into two,
visually into two parts that are half as big. Still this is the basic
idea of the house. We knew it was going to be a house that we would live
in the rest of our lives, and we wanted it to be adaptable, for the
future, so for example it was a two-storey house; we didn't know if one
of us might become crippled for some reason, so we made sure that there
would be a place for an elevator, in case we wanted to install an
elevator if we needed it to get upstairs. And we could use it as a
dumbwaiter meanwhile, for clothing or something. And then the people,
you know, the builders came, and at the time the house cost almost
\$100,000, which was considered way out of line. Now, of course, if you
can get a house in Palo Alto for less than \$1 million, you consider
yourself very lucky, even a very small house. But at that time this was
considered a rather large house, and we had to, we didn't have enough
money in the bank to pay for everything, but we, so we didn't put in the
wood panelling until a year later, and by that time the price of wood
had doubled. And we didn't put in the organ until four or five years
later. If you look in my book, in the index to Volume Three, you'll see,
Royalties, comma, Use of, and that refers you to a page where there's a
picture of organ pipes. And, you know, the fact that my book had been
selling well, meant that I got about a dollar for every copy that people
bought, and, you know, that book has been amazing. It still sells more
than ten copies a day, every day of the year.)

\subsection{\texorpdfstring{\href{http://webofstories.com/play/17107}{Volume
Three of The Art of Computer
Programming}}{Volume Three of The Art of Computer Programming}}\label{volume-three-of-the-art-of-computer-programming}

Volume Three is this, is this book about, about sorting and searching, a
field that was, that was changing greatly at the time I was making the
final version of that. So, I'm thinking about, I'm thinking about
sorting and searching as I'm in Princeton. You know, I've finished
Volume Two and I'm ready to come to Stanford, and, and, but, but I'm
getting new ideas about sorting all the time because the field is
changing. So, after I arrive at Stanford, I have to teach courses and
start brand new things that I've never done before, so it took a while,
and I wasn't able to finish Volume Three until I was in Norway in 1972.
That was my sabbatical year. It wasn't really a sabbatical year. I
called it a sabbatical year, but it's, it's a leave of absence. I'd been
three years at Stanford, and I, I thought, oh, it would be great,
professors are supposed to get six years and then a sabbatical, and then
six years and a sabbatical. I thought, maybe I'll go on a four year
cycle. I'll go three years, a leave of absence, then three years and a
sabbatical. That was okay with Stanford's rules. And so, I got an offer
from the University of Norway to spend a year there as a guest professor
and I didn't need financial support from Stanford, so I could, so I
could do that, run a leave of absence without the, without, you know,
the problem of, of who's going to pay for it, and I loved Norway. We had
visited there in 1967, believe it or not. That was another thing that
happened in 1967, and so I fell in love with that country. In fact, the
Norwegian National Anthem, vi elsker dette landet, we love this country.
And so, I, I went to Norway and that was where I, I completed Volume
Three and began to work on Volume Four. The, during that year I gave
lectures at the university every Wednesday, and stayed at home and did
my, my book writing and reading and taking care of my kids the rest of
the time. So, finally then Volume Three came out and it was, it was, it
added a new dimension to this whole idea of analysis of algorithms,
because when you start talking about the, about the topics that are in
Volume Three, it turned out that, that there were many even though the,
it's talking about particular things, sorting, searching - the lessons
that you're learning apply to many, many other things, and so, and so
that was why Volume Three was, was especially important for me, that it,
the paradigms, the methods that we use to study algorithms of all kinds
seemed all arise in the context of sorting, so I could make a unified
story about it, but also, but from teaching general principles while
being able to illustrate them around the special theme of sorting and
searching.)

\subsection{\texorpdfstring{\href{http://webofstories.com/play/17108}{Working
on Volume Four of The Art of Computer
Programming}}{Working on Volume Four of The Art of Computer Programming}}\label{working-on-volume-four-of-the-art-of-computer-programming}

Then I come back home and I'm thinking, okay, I'm ready, I'm going to do
Volume Four, and I, Volume Four is about Combinatorial Algorithms. Now,
combinatorial means, deals with zillions of combinations of ways to do
things, and, and as a result there's, there were many, many problems
where people, people had been wanting computers to solve, because they
couldn't solve them. So many cases had to be done. You needed a computer
to do it, yet nobody could figure out how to do it efficiently. And so,
a good idea could, could make a method run more than a million times
faster, and I was collecting all these good ideas but people kept having
more and more good ideas. And so, there's something in the field called
Combinatorial Explosion, which means to most people that the size of a
problem is growing, is growing huge, very rapidly. To me, Combinatorial
Explosion meant the research on combinatorial methods was growing
explosively. In 1974, five and six, when I'm working on Volume Four,
more than 50\% of all the papers in all the technical journals were
about things that belonged in Volume Four. So, in other words, it's like
sitting on top of a kettle that's boiling, you can't control it. If I'd
write something one week, it would be obsolete the next week. It's like,
you know, trying to write a book about the internet today, or something,
you know. So, it looked impossible to finish Volume Four. I was doing my
best, you know, and gathering material for it and reading section,
making lots of notes, and upstairs I have hundreds of folders of, of
these notes. I started and I had 30 folders and they were
well-organized, and then I made folders called X1, X2, X3 until I got up
to X15, not well-organized but just extensions to the system, and then
as new material came in I started throwing it into a pile, saying I hope
to have time to read this some day. So this field is growing very fast.)

\subsection{\texorpdfstring{\href{http://webofstories.com/play/17109}{Poor
quality typesetting on the second edition of my
book}}{Poor quality typesetting on the second edition of my book}}\label{poor-quality-typesetting-on-the-second-edition-of-my-book}

The publishers of my book are happy with the sales and so on. They say,
now, let's come out with a new edition. Let's come out with a new
edition of Volume Two, and they, and they, and I had, I had thoroughly
revised it. Another one of my projects in Norway was to go through page
by page and they had it on tear sheets, and you know, I tore up the
other book and then I said, okay, now add all this copy. So I rewrote
more than half of it for the second edition. And so then they, they sent
me back the galley proofs for the second edition and I was shocked, I, I
couldn't believe it. It looked awful. The quality of typesetting was,
was abominable. It was a pain to read. You couldn't look at this because
they had changed the printing technology in the meanwhile. The, the
books in the first edition were typeset on a Monotype system, a lot of
handwork involved, and the books in the second edition were printed
with, with a photographic method instead of, instead of lead type, and,
and the, the letters weren't positioned very accurately on the films and
some of them were darker than others. There was no quality control. The
subscripts were in a different style than the, than the letters in other
parts of the formula, so the math looked especially atrocious. Anyway, I
didn't want, I didn't want my book to look like this. I couldn't stand
to have, to have such a thing coming out as my work, when I, you know,
when the first edition had been the quality that I loved from my
calculus book in college. And so I was extremely upset. I didn't know,
you know, what I was going to do and I went and took another trip to
Boston and I talked to the, to all the people there and they sympathized
with me and they said, well, maybe they could go through some undercover
people in Poland who might be able to make a better copy of the old
typefaces and, and do a better job, but the proofs that I got from that
were, were awful too. And they were contracting out to the people who
were the best in the world at this new technology. The old technology
was dying out. You could still get some places in Hungary that would,
that would use it and maybe in India, would do Monotype setting of
mathematics, but the people had died who had this skill. And the, the
printing industry had gone to the new technology because it worked fine
for newspapers and magazines.

For the typesetting?

For the typesetting; and for mathematics, nobody cared. Mathematics had
always been considered penalty copy that, you know, that, that you do.
It had gotten, Addison-Wesley actually had the best mathematics they,
they had a man named Hans Wolf, who had learned his typography in
Germany, set up his type shop to do all Addison-Wesley's books, right
next door to Addison-Wesley headquarters in Massachusetts. And that's
why I had liked his books, because he had, he had, you know, taken the
old technology and, and perfected it, and, and he was retired, you know,
and all the people and the, you know, the apprentices, they, they
disappeared from the scene. So, I was at quite a loss, and meanwhile, I
was at Stanford. I was Chair of our committee to design comprehensive
examinations. The, the students are supposed to take two examinations
before they do their Ph.D, one is comprehensive and one is specialized.
So the comprehensive, they're supposed to know a little bit of
everything, and the specialized is in their own field, whatever it is,
artificial intelligence or whatever. And so, every year one of my
committee's duties is to read the new books in the fields and to, and to
see which ones to recommend for the reading list for the students the
following year, and a book came out by Patrick Winston, which was only
in galley proof form, but this, but it was a book on artificial
intelligence that was typeset in a new way, and the, the typesetting had
been done in Southern California on an, on a machine that was completely
digital. It wasn't lead type, it wasn't photographs, but it was, it was
pixels. It was bits like we now have in all of our printing equipment
now, but at that time it was, you know, the digital printing at that
time was not considered good quality, but this, at this machine made by
Triple I Corporation-Internat- I can't remember what III stood for, but
its headquarters was in Southern California. This machine was used to
typeset the book by Pat Winston, and they sent galley proofs for my
committee to look at, and I looked at this and this was produced by a
computer and it looked gorgeous.)

\subsection{\texorpdfstring{\href{http://webofstories.com/play/17110}{Deciding
to make my own typesetting
program}}{Deciding to make my own typesetting program}}\label{deciding-to-make-my-own-typesetting-program}

I was about to do my first real sabbatical, by going, and I was planning
to go to Chile in 1978 and learn Spanish, and, so, but in February of
1977 I see these proofs from Winston's book and I know that it's done
with pixels, with bits, and that is a computer science solution. That
means these pixels are zeroes and ones, these are in every part of the,
in every tiny little part of the page you say, ink, yes or no, and yes
or no, zeroes and ones, that's a computer. So, so printing had suddenly
been reduced from a problem for metallurgists or from a problem for
optic, people in optics, you know, photography and lenses to a problem
in computer science, a problem about zeroes and ones. All of a sudden
printing, high quality printing was just a matter of writing a computer
program. And now, if anybody in the world can deal with zeroes and ones,
it was me, right? Anyway, I'd been studying, you know, looking at galley
proofs for ages, and my father had printing presses in the basement, I
didn't mention that. So, I'm interested in type and I also, you know,
see now that it's a computer science problem. So I changed my whole life
plan. I wrote to the people in Chile saying sorry, I have to stay in
California because we have special equipment here that I'm going to need
to do my work with during the next year. I'm going to try to figure out
a way to print my book by making it myself out of zeroes and ones
instead of using any photographic or metal methods, and then, then I'd
have perfect control over it, and if technology changes again, I'll be
able to survive the change because I'll have the thing in digital form.
And that was within a week of seeing the proofs of Pat Winston's book in
February that I, that, I made this decision to stay at Stanford for my
sabbatical year and work on typography.)

\subsection{\texorpdfstring{\href{http://webofstories.com/play/17111}{Working
on my typesetting program (Part
1)}}{Working on my typesetting program (Part 1)}}\label{working-on-my-typesetting-program-part-1}

In the beginning of 1977 I spent the first three months writing part of
Volume Four. I, you know, I completed all my reading for Volume Four and
I was ready to finally type some pages, and I typed up a hundred and
some pages, but in the meanwhile that was that overlap of the period
when I had discovered about digital typography, that you could get good
quality from computer output. And so, my plan had been to go to Chile
and spend a year working on Volume Four, my, my first real sabbatical,
and I cancelled that to stay at Stanford and I stopped working on Volume
Four, put that on hold starting in April of '77 and began to say, well,
let me spend a year writing computer programs that will make my book
look okay again. And so, first I thought, I had seen some people at
Xerox Palo Alto Research Center playing with letter forms, Butler
Lampson especially, I remember sitting in front of a computer monitor
and he had a big capital B on the screen and he was looking at spline
curves along the edge that would trace the edges of the B, and I
thought, okay, that would be pretty easy, I'll, I'll just capture all of
the letter forms from my book, I'll go over to Xerox and use their
equipment and, and I'll write a program that reproduces the format of
``The Art of Computer Programming'', and I have a year to kill, so I'll
do this. Well, the first thing that went wrong was Xerox said, oh yes,
it will be fine if you come over and use our equipment, but then we own
all of the fonts that you produce, and I said no, no, I wanted these,
you know, these are mathematics that- I mean I come out of here with a
bunch of numbers that say where the edges of these letters are, but
these are just numbers. How can you own these numbers? You know, the
numbers belong to the world. But so I, anyway, I decided I would work at
Stanford instead. Stanford didn't have very good equipment. The, we had
some old TV cameras, and if you changed the lighting by a little bit in
the room, the letters would get 50\% darker. So, there was absolutely no
way to get consistency between one day's work and another day's work,
and consistency is really important for, to get good looking letters.)

\subsection{\texorpdfstring{\href{http://webofstories.com/play/17112}{Working
on my typesetting program (Part
2)}}{Working on my typesetting program (Part 2)}}\label{working-on-my-typesetting-program-part-2}

I decided to go to Stanford instead and there, at that time, Stanford
had a really great artificial intelligence laboratory up in the
mountains presided, presided over by John McCarthy. He welcomed me to
come up there and do anything I wanted and it was a great place to work
for the year that I spent, but we didn't have very good, very good
equipment compared to what they had at Xerox. We had TV cameras that,
that I could use, but if they changed the lighting in the room just a
little bit, the letters would change drastically from thin to thick or
vice versa, so there was no way to get consistency from one day to the
next. I should, I should mention that my publishers had sent me
beautifully printed copies of, of previous editions of my book. What I
mean by beautifully printed is, the way the books are actually produced,
are kind of interesting. With monotype, they set in lead and they print
one copy, and then they take that one great copy and photograph it and
print offset off of the, off of the film that they've taken, and that
was getting great results, but they had saved the original copy that
came off of the lead, so they sent me 50 pages of the, of the real,
absolutely definitive versions of what had been used to do the previous
edition of my book. So I took these to the, to our TV cameras at the AI
lab, but I couldn't get any consistency at all. So I, so I didn't know
what to do about this. I was talking to Jill and she said- well, she
could make some good slides, she could, she could photograph these pages
and we could make 35mm slides and they would be sharp, and we could see
what we could do with, with that instead of working with a TV image. So,
we tried that and then I got my home projector, and I, and I projected
the 35mm slide from the end of the corridor onto a wall and so that it
would be large enough, and I could trace on a piece of paper what the
letter forms were. But the images were very fuzzy. When enlarged that
much, you know, we didn't have any way to magnify things now, like film
scanners or anything that didn't exist. Certainly we had no such
equipment. So I'm sitting there looking at these, trying to trace the
letters and figure out what are the shapes and, and suddenly it occurred
to me, you know, these letters were actually designed by a human being
who had something in mind, and if I could only psyche out what that
person had in mind, I could, I could write mathematical formulas that
would match that person's intentions. This was a weird, a weird dream,
but anyway, this was what, I was naïve and I thought this would be
fine, and as I had said, when I was in high school, I played a lot with
graphs and the relationship between shapes and, and mathematical
formulas that make those shapes. So, so as I'm sitting there with my
pencil and paper trying to, trying to trace the outlines off of these
blurry slides being projected on the wall, you know, I noticed a lot of
regularities. For example, I noticed that the letter n was exactly twice
the width of the letter i, and the letter m was exactly three times the
width of the letter i, and okay, great, you know, there was consistency
there, that if you have a word like minimum, all of those vertical
strokes are coming exactly at the same spacing. Also the u was, you
know, the same width as the n. So it occurred to me then that if I could
figure out, you know, what all these principles were that the type
designer had done, it wouldn't be too much of a problem to capture that
in mathematics. Okay, so instead of using- so then my whole focus
shifted so that I wasn't going to just copy the output of a previous
design, I was going to try to, to discover the intelligence behind it,
and program the intelligence into the computer, so that you could tell a
computer, draw a letter m, and it would draw a letter m, and I, but I
could say, now make it a little bolder, now make it a little taller, a
little narrower, and it would adapt everything correctly. Now this was a
weird concept to type designers. Type designers always, you know,
approached it with a completely visual approach and no mathematics,
probably thank goodness, because they came up with beautiful design, but
at any rate, I naïvely thought that this was going to be easy, and of
course my whole, my goal, it was to have Volume Two ready for the
printer after my sabbatical year. I had one year to write these programs
and to, and to capture the letter forms and figure out how to typeset
them.)

\subsection{\texorpdfstring{\href{http://webofstories.com/play/17113}{Research
into the history of
typography}}{Research into the history of typography}}\label{research-into-the-history-of-typography}

I wasn't going to go into this whole project blindly, I wanted to, I
wanted to avoid the mistake that almost every engineer makes when they
approach something with technology and that is to say, well, forget the
past, it's all a new world, throw, throw the old stuff out and redo
everything. So I didn't want to, in any way detach myself away from the
old traditions. I had these ideas about using mathematics to capture the
past, but I didn't, but I didn't want to throw away the past so I did
lots and lots of reading in the library, and I studied. And Stanford has
one of the best collections of books on fine printing, called the Gunst
Collection in its, in its department of rare books. And so I spent lots
and lots of time looking at all the books I could get my hands on, on
typography. Also, my wife and I were members of the Associates of the
Stanford Libraries, and we went on a trip to Sacramento where we could
see the Gutenberg Bible and various other typographic specimens. There
was a show put on there at the Sacramento Bee Office about history of
typography and beautiful things, and we visited people who lived in the
mountains and had their own printing presses and, and collected books on
typography. So this was a two day trip into Gold Country, and we, and
I'm staring at all of the letters that I see, and saying, how could I
capture this in, in a computer? This was all taking place in the Spring
of 1977.)

\subsection{\texorpdfstring{\href{http://webofstories.com/play/17114}{Working
on my letters and problems with the
S}}{Working on my letters and problems with the S}}\label{working-on-my-letters-and-problems-with-the-s}

In June of 1977 my sabbatical begins, classes are over, I'm ready to go
on this and I, and I had sketched out ways to do lower case alphabet, A
to Z, and with mathematical specifications and I thought great, this
will, this will solve my problem because, you know, I can make the
letters all different sizes with slight changes in the specifications. A
good type design doesn't just make nine-point letters, nine tenths of a
tenth size of a ten-point letter, but they also make them a little bit
wider, subtle changes in the, in the greyness of the characters and, but
I could design my program with these parameters to do the variations and
draw each letter to each size properly. This is a very natural thing for
a computer scientist to, to do because we're used to specifying things
with parameters all the time. I mean we, we have a job to perform and
we, we have certain things variable, and we twiddle those dials and that
will give us a different output. What, but I found out to my surprise
that when I was talking later to people in the graphics industry, I,
they hadn't heard the word parameter before, and they thought it meant
perimeter and, and I mean, it was something that, you know, but computer
people think it's just as natural as anything. Well, so I came up with
25 of the 26 letters in the middle of June and they weren't beautiful
but they were pretty close to, to being okay to my eyes at the time, but
then there was the letter S, and I couldn't figure out how to draw a
blasted S. And, I- it has a very peculiar shape where it changes over
from curves, sort of, curves left, then right a little bit, and then
back and forth, and what's going on in this shape. None of my
mathematical formulas would, would handle it, and I spent several days
without sleep up at the lab, you know, trying different things and every
time it would just look very ugly. And after, finally, I had to come
home and, you know, and, and go to bed and I showed my results to Jill
and she said to me, well Don- why don't you make it s-shaped? That was
her solution to my problems. Well, oh man, what was I going to do? So,
but, but finally after, I don't remember how many days it was, counting
the nights, anyways, it seemed like an eternity, I, I got the idea that
if, that I could go back to a geometry problem that the Greeks might
have enjoyed about ellipses, where if you have an ellipse in it, and you
try to make it, the ellipse just big enough so that the tangent, as it's
coming down, meets things in a proper way, it, we- become a problem that
would be interesting to Euclid or somebody, although I don't think it
was, would appear in his work. And that would be a way to actually
figure out, the ellipse would be for part of the S, and when it comes to
this tangent, then we go off on that tangent and go to the other part of
the S, and so I had finally a mathematical formula that I could, that
looked like an S, and I could change some of the parameters and it would
look like a slightly different S and so on. But before I came up with
that formula, I was thinking, well, maybe I could change my book, you
know, and try to avoid using the letter S entirely, but then I, you
know, I'd have to leave Stanford. So it was, I would go, you know, visit
Safeway and they had a funny looking S, you know, and I would see all
the different S's everywhere, but to finally solve the problem of the S,
right, but incidentally, I never did solve the problem of the dollar
sign, though. The dollar sign that I have is still pretty ugly and, and
I'm waiting for some new idea to come along, but I never was too much
interested in dollars really, so I didn't bother to spend too many days
on that.)

\subsection{\texorpdfstring{\href{http://webofstories.com/play/17115}{Figuring
out how to typeset and the problem with
specifications}}{Figuring out how to typeset and the problem with specifications}}\label{figuring-out-how-to-typeset-and-the-problem-with-specifications}

So then I had 26 letters, but I had to go on and figure out how to
typeset, how, how to put them on the page next, so I stopped working on
fonts and I, and I started to just figure out well, how am I now going
to specify my book. This was what, you know, I, this was very early
days, now there's all kinds of software that does this, but nobody had
it in those times, except there were some good programs written for the
newspaper industry, especially in fact for ``US News Report''and for
``News Day'', and, and so the people who had written those programs were
students of friends of mine, and, and so I went and talked to them. I
looked at the programs that were available. I looked at the programs
that had been used to typeset the Volume Two that I'd thrown away. That
typesetting was done in Belfast by, by a company that was top of the
line as far as commercial mathematics setting was concerned, and they,
they knew their business but they just didn't have good fonts to use for
the output of their system, so I, so I looked at that system, and I
looked at a system that was developed at Bell Laboratories and, you
know, every, every other system for this, and I, then I sat down and I
took ten sample pages of my book and I said, you know, as soon as I get
those, these pages working then that'll handle all of the formats that I
need, you know. In fact, these were just fragments of pages, but I have,
have lots of different kinds of things going on in my book, from
ordinary copy to, to math formulas and algorithms, computer programs,
table, tables and figures and all these different typographic features,
but it, it boiled down to four pages of stuff, with changing a lot in it
and I, and I sat down and I said, okay, here's how I would like to
specify those four pages of, of sample text for a typesetting system
that would work. And I stayed up all night one night typing this trial
language for, for capturing my book, and, and then the only problem was
to imagine a computer program that would go from this input to zeroes
and ones on the page. That would, that would do it. And so, then I
stayed up another night writing up the rules of this language and showed
them to some friends a couple of weeks later, revised those rules. Then
it was the middle of July and time for me to go to China. I had, with
some other people, my, my wife and kids and, and three other people, we
took a trip to China. This was kind of early days for China. Mao had
just, had died less than a year before, but we were invited to spend
three weeks at, in various parts of China and see the country, and, and
it was kind of a unique opportunity, and also for, for my kids, they
were some of the first children from the USA to be allowed into the,
into mainland China at the time and so we had, so that was my summer
plans. So I had two student assistants that summer, and I presented them
with my specifications for, for my typesetting language, which at the
time I called TEX, now we pronounce it Tech, but you know, first it was
called TEX, you know, and I, and I assumed that when I got back from
China they would have the computer program working. Then I, you know,
then I can go on and do the next part of my work. Well, I had a great
time in China and I came back, but to my surprise, they hadn't even
gotten one page out yet, one test page done, and, well, another student
had made some progress on the question of hyphenating words, but I was a
little bit surprised that, that it turned out to be so difficult, when I
thought my specifications were so clear. Well, this was my introduction
to the reason why so many software projects fail, because people think
that specifications are clear, but, but really, as soon as you try to
translate some specifications that seem clear to a human being and, and
explain those to a computer, you know, there's all kinds of questions
that come up that you never anticipated. So, here the students are, and
every time they, they come to, to a new part of my specifications, a
question arises and they can't ask me because I'm, I'm off in China. So
then they have to meet with each other, and say, what did he mean by
that, you know? What did he think? And so on, a get-together. And so
they solved that, and they, they'd work for another half an hour and
they'd come up with another such problem, you know, over and over again.
So, it was miraculous they got done as much as they did, that, you know,
while I was gone. Anyway, they did show me the program that they wrote
and that, that gave me an idea for architecture or structure that I
could use when I had to write the real program myself later. So then
they go back to school, and I come back and I start, and I start
programming in earnest.)

\subsection{\texorpdfstring{\href{http://webofstories.com/play/17116}{Working
on the programming for TeX and using a structured program for the first
time}}{Working on the programming for TeX and using a structured program for the first time}}\label{working-on-the-programming-for-tex-and-using-a-structured-program-for-the-first-time}

I started programming in a, in a way that I hadn't done before. During
the '70s, a style of programming was developed called Structured
programming, which made it easier to, to produce reliable programs,
programs that you could understand as you were writing them, and, and I
had done some experiments with it on a lower, on a smaller scale, but
this was going to be my first large program written with this
methodology of Structured programming. And the new thing was that as I'm
writing the program, I didn't feel the need to test it, to test each
part of it as I was writing it. I could just write and write and write
and assumed that after it's all written, it'll be pretty close to
working. So what actually happened is that I, I started writing in
September or October, and I didn't test, I, I didn't actually enter any
of the program into the computer or test it or debug it until the next
March, so all these months are going by, and, and I didn't feel the need
for a prototype or any, any, I mean, when most programmers write
something is they'll put, they'll put something together and then that's
only part of the program so then they'll mock up the missing parts to
have something that pretends that these other missing parts are there
before they write them, but then, then later on when those parts are
there, then they can continue, and they, and they build up more
confidence all the time. Well, with structured programming, I already
had pretty much confidence, so I didn't feel the need to dummy up these
extra parts, and that saved me a lot of time because all the time that I
would take to write, to write the dummies, I no longer needed. But,
sometimes I have to admit a few twinges of doubt, you know, you had five
months going by between the time I wrote the program to the time I'd get
to really try it out, but you see, I couldn't. One of the things I would
have had to dummy up was the whole, the whole question of fonts. I
couldn't do typesetting without the fonts, and I had to write those, I
had to do all those fonts as well, and that was, you know, three, three
months of that five was to get the fonts going.)

\subsection{\texorpdfstring{\href{http://webofstories.com/play/17117}{Why
the designer and the implementor of a program should be the same
person}}{Why the designer and the implementor of a program should be the same person}}\label{why-the-designer-and-the-implementor-of-a-program-should-be-the-same-person}

Getting the first version of TeX to work, I also had to get the fonts
ready, the letter forms, and after, after five months I had something I
could finally test, and I, but after, after two days I realized why my
students hadn't gotten any further than they did, because I had to keep
changing the language constantly, every, several times an hour as I was
beginning to write the code, and if the students, several of my friends
have tried to, to direct software projects where they set the
specifications and they say to the students, write the code, and it
never works, and now I know why, because the students would have to
schedule an appointment. Every few minutes they have the students
working on it, they have to schedule an appointment with the professor
to, to discuss what the issue is, and then the professor would have to,
you know, take 15 minutes to get in the frame of mind to understand what
the problem is, or he'll say, oh yes, do this, and then the student can
go back and work for another five minutes before another question comes
up. So it just takes forever. When you do enough, working on a first
generation software like this I think it has to be done, the designer
and the implementer have to be the same person.)

\subsection{\texorpdfstring{\href{http://webofstories.com/play/17118}{Converting
Volume Two to TeX and how an implementor/designer should be the first
user of his
programs}}{Converting Volume Two to TeX and how an implementor/designer should be the first user of his programs}}\label{converting-volume-two-to-tex-and-how-an-implementordesigner-should-be-the-first-user-of-his-programs}

By the time, by Easter time then at near the end of my sabbatical year,
I finally was able to typeset a few, a few sample pages of, and got my,
my test program going, but I didn't have Volume Two done by any, by any
stretch of the imagination. Still, it was- enough was in place that I
could see that it was going to be possible. I guess I should mention
that I had no idea that I was designing a typesetting system that
anybody else in the world would ever use, besides me and my secretary.
My idea was that, that I would teach Phyllis, whom I, who was my
secretary for almost 20 years, and I dedicated one of my books to her,
that's a long story there, but anyway, so Phyllis and I were going to be
the users of this TeX system, and I tried to make it so that she, you
know, I knew she'd be able to learn it, but I tried to design it just,
just good enough to do my books and no other books. So, so the next
phase was however, for me to go through the entire manuscript, the
Volume Two, and I wrote a program that converted from the Belfast system
to my new system and that would be, that would be a close approximation,
and I started in. Well, it was several months of work, it's a 700 some,
700 or 800 page book, but another, I learnt another important lesson
during this time. Not only should the designer of a system be the first,
that would be for the implementer, also has to be the first user,
because I'm actually using the system, I'm always getting ideas about
how to make it better, and if, and if I'm not a heavy user, but just a
designer, I don't have the experience to know why those features are
useful. So I kept track of all of the changes that I made to the system
at this time, and I also knew how many pages I had been typesetting of
Volume Two. And I went through and, and made a graph of it afterwards,
and it's almost a perfect straight line. Every four pages I typeset, I
got a new idea, you know, I'd do, I'd do 200 pages, what is that, 200
divided by four, have a computer, 50 new ideas, right. So, and it's a
straight line until I got up to about 500 pages, and then it's flat.
After that it was all boring. The last, you know, I mean after, after a
certain point there was no learning curve anymore. I knew all the things
that needed to be improved, but, but those many changes that I made at
the beginning, were, were certainly important for all future users of
the system, although I didn't know that there would be any users of the
system at the time.)

\subsection{\texorpdfstring{\href{http://webofstories.com/play/17119}{Writing
a user manual for
TeX}}{Writing a user manual for TeX}}\label{writing-a-user-manual-for-tex}

However, the AI lab is a very open place, there's, there were dozens of
people and lots of visitors, and no doors closed, and so people are
looking and saying, what are you doing, Don, and they, they see this
interesting stuff coming out of the printing machine, and so, pretty
soon it became known that I had a system that would typeset something
that looked almost like a real book. And, and one of the visitors was
Guy Steele from MIT, and he said, Don, I want to port, you know, I want
to port this system, I want to modify it so we can use, we can use it
also at MIT. I thought, oh my goodness, should I have other users, you
know, but then, by this time I could see that other, you know, that
there were enough people expressing interest in it, that, that I had
better write a manual, and expand that. It wasn't going to be just
Phyllis and me using, using it. So, then during the summer of `78, I
wrote a Users' Manual and it was a little bit interesting that the, our
computer was extremely unreliable up there. The people, the people, we
had four really brilliant computer scientists improving the operating
system every day, but, but because they were improving it every day, it
would, every day it was also crashing because they would make mistakes,
and so, so I wrote the entire User Manual during computer downtime. I'd
go up, you know, I'd go up there and the computer, you know, at nine
o'clock in the morning and the computer would crash at 09:30, so I, I
opened my tablet and I'd write another chapter, the computer comes up
ten or eleven o'clock, and I can, you know, do a little more work until,
until it's ready to write another chapter. So, I learnt another lesson
there and that is, if you have a system that's always improving, nobody
gets to use it. And I and I'll talk a little more about stability of
software later on. The, the budget for, for operating system work ran
out the next year, and so three of these four people went to Livermore
to work on another system, another project, and the fourth guy, Marty
Frost stayed and his job was not to make any more improvements, but just
to make sure that it was stable, that the system was stable and
reliable. And so, so that was the greatest thing that the system wasn't
improving any more after, after '78. And I dedicated a book to Marty
also.)

\subsection{\texorpdfstring{\href{http://webofstories.com/play/17120}{Working
with The American Mathematical Society and giving the Gibbs
Lecture}}{Working with The American Mathematical Society and giving the Gibbs Lecture}}\label{working-with-the-american-mathematical-society-and-giving-the-gibbs-lecture}

I got my Users' Manual ready and had visitors from the American
Mathematical Society. They were, they were, they had horrible
typesetting problems. They're the largest publisher of mathematics in
the world, and they, they tried to do their best with their journals,
but their journals were looking terrible, just like, you know, just like
my books. They were, they were upset about that situation, so they had
heard about rumours of my system. Even their, their most prestigious
publication, ``The Mathematical Reviews'', which they had typeset by a
really expensive company, that company couldn't do all of the operations
that they wanted, so they came, visited me for two weeks and took a look
at my system and I, and I showed them some mock-ups of how we could
maybe typeset journals as well as ``The Art of Computer Programming''. I
was invited then to give, to give, something called The Gibbs Lecture of
the American Math Society. Every year it's one of the main invited
lectures to the main annual meeting, and the lectures are usually given
by a pure mathematician, but every four years they take somebody who's
more applied, so Einstein had given one of these lectures, you know, all
the various glorious mathematically oriented, mathematically oriented
physicists had done, so it's quite a thing to get, to be invited to give
a Gibbs Lecture. I decided my Gibbs Lecture would not be about what they
expected in computer science, but I was going to talk about typography,
so my Gibbs Lecture was about mathematical typography. And I still, I
still didn't have my fonts all done though, for Volume Two, I just had
enough done that I could test my system, but there was still a lot of
work to do to get the, to fine tune and to make it look like, like a
high quality thing. Still, I could give the Gibbs Lecture in, at the end
of, well, it was the beginning of 1979, and it was a big hit with the,
with the audience, so I knew that there was also a hunger out there for,
you know, for, for people being able to do better typesetting of the
mathematics.)

\subsection{\texorpdfstring{\href{http://webofstories.com/play/17121}{Developing
Metafont and
TeX}}{Developing Metafont and TeX}}\label{developing-metafont-and-tex}

Somebody suggested that I call it Metafont, because it wasn't my, yes,
it wasn't something I, I made up myself and I can't, I think it was
maybe Bob Filman, but I can't remember for sure, but as we were looking
at these things up at the lab, there was lots of visitors and people
talking to me all the time, and that, that was a nice name because of
the parameters. I'm not designing just one font of type, I'm designing a
font that varies with the specifications, you know, it varies
substantially with different specifications, and the, and so there's
lots of like words, like metamathematics means going a little beyond
mathematics or metaphysics, metaphysics really meant after physics in
Greek, but the word Metafont turned out to be kind of appropriate. Now,
it's something like morphing, in a way that you can go from one shape to
another, that's why I'm thinking metamorphosis here, and, but the, my
first programs for fonts were, each, each letter was drawn by a specific
computer program written in computer language. I had to change that so,
so that I could devise a special language just for fonts, so that it was
easier to make changes and, and to make the design. Instead of writing
computer code, you know, the language like ALGOL, I could write now in
this Metafont language, and then I would have an interpreter that would
read the Metafont language and draw the shapes instead of the original
way, which was calling subroutines that would, that would draw the
shapes. So, it was a completely different kind of a language than had
existed before and, but I came up with the first design of a Metafont
language in 1979. Now, as the number of users grew from one person to
ten people, I needed to change the system because they had different,
you know, they'd discovered different bugs, they needed different,
different facilities. Then it grew from ten to 100, and again, I had to
add more stuff to the, to the language. By this time it was called TeX
instead of Tex because there was already a system called Tex, and the,
Honeywell refused to give us permission to use that name, even though
their system was an operating system, not a typesetting system. We
couldn't do that. So, we said, well, this isn't Tex, this is TeX, this
is Greek, Tau Epsilon Chi, it's not, it's not an English word at all, it
just happen to look very much like the English word Tex. And well, we
told all the users to pronounce it TeX, and, but their lawyers didn't
think that wasn't much, enough of a distinction, so we left it on paper
that, you know, if they want to sue us, you know, we wouldn't, we didn't
get an official registered copyright, we just have it as a copyright,
not a registered copyright. Metafont on the other hand, is a registered
copyright. Nobody was complaining about that name. Well, I was going to
say, as we get, as we get ten times as many users, then we need, then I
have to go back and retool the system. This can, this can be too long a
story, so I'd better start summarizing. The first version of TeX, a
prototype version, was called TeX 78, because it, it saw the light of
day in 1978, then there was Metafont in '79, but these are quite
different from the Tex and Metafont systems that we have now. These were
the first versions. I have another policy that I recommend, and that is
if you're doing a first generation piece of software, get it working, do
your best, try to get it so that it's perfect and then get a lot of
experience with it, then throw it away and start over. Scrap it and
don't worry about being compatible with the old, because there are going
to be many, many more users in the future than you ever had in the past,
even though there might have been thousands of users in the past, there
will be more, you know, tens of thousands in the future and they'll all
thank you for having a better, a better system. I don't do it- the
second one I keep though. And now there's so much invested in it that,
that I'm sure that I would never want to use a system that, you know,
that I never want to change again to a system that's incompatible with
the one that I've been using for so many years now. Okay, this is, this
takes us then to the point where suddenly we have users from around the
world, and even now a meeting of people coming to Stanford to take a
look at this system and to try to get it running and not- at first it
was running at MIT but, but then it was, it was running at hundreds of
different kinds of computers all over, all over the world.)

\subsection{\texorpdfstring{\href{http://webofstories.com/play/17122}{Why
I chose not to retain any rights to TeX and getting it transcribed from
SAIL to
Pascal}}{Why I chose not to retain any rights to TeX and getting it transcribed from SAIL to Pascal}}\label{why-i-chose-not-to-retain-any-rights-to-tex-and-getting-it-transcribed-from-sail-to-pascal}

From the beginning I decided that I wasn't going to retain any rights to
this system, except the right of, of nobody should, should, should
diddle with it. They, anybody could use my system free of charge as long
as they didn't make any changes whatsoever. It had to be entirely
compatible with the system that I had, and, and I wanted, so I didn't
want to charge for anything. If I had, if this had been the only thing I
had worked on in my life, I would have probably, I would certainly had a
different idea, but I, but I had seen the way proprietary things had
been holding back the printing industry for years and years. There were
five or six different commercial systems for describing pages of text,
and they were, everybody thought they were the systems that everybody in
the world would use, and they, and they were totally incompatible with
each other and they, and you couldn't use different fonts with different
systems. It was all, it was all a mess. So, I was thinking that in the
earlier days of computer programming languages, IBM did not keep FORTRAN
as a, as something that was an IBM-only language, they allowed, they
allowed dozens of other manufacturers to make their own FORTRAN
versions, and as, and that was, that was a big boon to the, to, to use
of computers. I, I could see that the same would happen in, in the
printing industry if, if I didn't retain proprietary rights to, to the
use of this system, and I also made all my code available so people,
people could read it and see what I had done and find mistakes for it,
and I paid people if they found an error. So, thousands of volunteers
sprung up all over the world, helping me get this, get this system
better.

Is this when you invented the Weave and Tangle?

So, the literate programming idea, the Weave and Tangle came about
shortly afterwards in 1982. First of all, I had written my program in a,
in a Stanford-only language. It was called SAIL, Stanford Artificial
Intelligence Language. It's something that not many people could do, but
I, you know, so, so in order to run at first on other machines, it was
difficult unless they'd happen to have a PDP10 Computer which is, with a
SAIL compiler. So, in the early days that's why MIT was, were one of the
first users, and there were, there were quite a few installations of
this computer and system, but not, but it was, it wasn't on very many
different machines, so I don't think on IBM mainframe you could run, you
could run it. So then a student transcribed my code. This was Iñaki
Zabala, transcribed my code into Pascal, I think David Fuchs and Art
Samuel also worked with him on this. Anyway, lots of people were working
on this, I didn't do it myself, but they, they transcribed my, my
program from the SAIL language into Pascal language and then that would
run on, on every computer and that's, that's where we had hundreds of,
of, of different users and then we started having a, having annual
meetings of the people who were from around the world to trade
information.)

\subsection{\texorpdfstring{\href{http://webofstories.com/play/17123}{Tuning
up my fonts and getting funding for
TeX}}{Tuning up my fonts and getting funding for TeX}}\label{tuning-up-my-fonts-and-getting-funding-for-tex}

After I had so many users, I realized that the, the original, my
original designs were too naïve. They had, I had to go, I had to retool
them and, and so the, as I was doing it, I, I came across the idea of
literate programming at the same time. But I have to postpone that a
second first because meanwhile, with the old systems, with the TeX 78
and Metafont 79, I also wanted to tune up my fonts, you know, so that I
could, so that I could really match the, the letter forms, and I'd been
reading a lot of, I'd been reading a lot of things about font design and
learning more about history of letters, and so I, I was also talking
with and meeting people at, in industry, like Linotype and you know,
Mergenthaler, and so they introduced me to, to type designers and
especially I got to know Chuck Bigelow and Matthew Carter and
eventually, Hermann Zapf, who were the three of the four leaders in the
field. The fourth man Frutiger, I never did meet. He was, he was ill,
but, but I could have if he had been, if he had been healthy. But, but,
so, so they helped me get the worst glitches out of, out of my, my
designs and when I had the, the Metafont 79 I could, I had the benefit
of their comments to, to revise my designs. I also, I also had to buy a
typesetting machine, and so I took trips around the world to different
places, to where they had different experimental type typesetting
machines, and when we finally got one, it was called an Alphatype, which
had the highest resolution of any machine that I, that I could get in.
It was also, we could afford it. It cost us, I don't know, \$40 or
\$50,000. I think Addison-Wesley paid for \$20,000 of that. We were
running on a shoestring, but, but it turned out later we got, we got
good Research Funding. I was supported by the National Science
Foundation to do research in algorithms, and that was including, as far
as they knew, that was what I was going to do during my sabbatical year
and I said, you know, my research on algorithms is going to, going to
have the following seven parts to it. And the first six parts are, I'm
going to study algorithms for, you know, various combinatorial problems,
and the seventh part was, I'm also going to work on algorithms for, for
typesetting so that I can explain the research that I'd done on the
first six parts of my grant. And really, I spent all my time on, on part
seven, pretty much because I had the other research basically done
already, so I could, I could still have the other thing on my, on my
progress report, but I was working full-time on, on this typesetting.
And, so they were supporting it through the backdoor. And later on,
System Development Foundation learnt that. Well, I met with NSF people
and we were open about the need for it, and they said, why don't I make
a proposal to another division of NSF for, to work on digital typography
after the, my Gibbs Lecture and so on. Then people knew about it, and so
I got, so I got NSF support also from another division, but then the
System Development Foundation was a special thing that they were trying
to spend a lot of money wisely that had been left over from the System
Development Corporation's Sage Project in Southern California, and they,
they supported things like Stanford's center for Music and Acoustics,
and they gave me \$1 million to, in the words of their Chief Financial
Officer, who, what you call him, he said he wanted to give me this money
so that I could finish the work on TeX and get back to ``The Art of
Computer Programming''. So, so I had good funding eventually, but in the
early days it was, it was, we were not flush with money by any means.
So, I get advice from the top- the world's top designers, they look at
the letters and they mark it up with red pen and, and then I fix it
until it satisfies them. And I get this Alphatype Machine, and in order
to run the Alphatype Machine I had to write my own software for the
insides of the machine. It had a little, tiny 8-bit computer in it with
six levels of interrupts and I had to write my own assembly program for
it, and, and it had very little memory in. We had to transmit from our
mainframe to this computer with little Zilog machines. So, it was a heck
of a lot of fun, but also very frustrating too, because this, this
machine is deaf and blind, and it took a lot of work to get it going.
Finally I was able to typeset on the Alphatype, and I had pages that
looked like, that looked like, you know, high resolution and much better
than the ones that I did. So, I, I, I, after all this work, I started in
1977 and, and continuing on much longer than the year that I thought it
was going to take, I finally had Volume Two all ready and printed on the
Alpha Type and I could send it to Addison-Wesley for them to photograph
and, and print by offset.)

\subsection{\texorpdfstring{\href{http://webofstories.com/play/17124}{The
second attempt to print Volume Two of The Art of Computer
Programming}}{The second attempt to print Volume Two of The Art of Computer Programming}}\label{the-second-attempt-to-print-volume-two-of-the-art-of-computer-programming}

I was expecting to be, the greatest day of my life when the book arrived
in January of 1961, '81, 1981, I was expecting that would be the
greatest time to, to open this book and, you know, celebrate. I finally
had the project done. Well, it was one of the worst days of my life. I
opened the book, and I didn't like it at all, what I see, I mean I had,
you know, the, it was the old binding, and, but when I looked at, and I,
and I opened it, and it just looked completely different, and I had, you
know, from the proof that I'd seen, I had thought that it was going to
be, it was going to be fine, but I, I felt so, I could feel this huge
rush of heat as I'm looking at it saying- oh no. I had spent all this
time and, and I got this, and I tried to, the people were saying- nice
job Don and so on, but I couldn't believe them, you know, I, and inside
of me I said, my gosh, I'm way far from being done with this and getting
something, you know, something decent. The worst thing was the numbers,
I mean, most of the pages looked okay if you, if you're not too fussy,
but as you're paging through the book, and you're looking for a certain
page or you're looking at the numbers, or your eyes are focussing on
numbers, and I, and I debugged the A, B, C, D and E, but the zero, one,
two, three, four I didn't spend much time on and, and boy, those numbers
just were very ugly. So I had to go through another, first I couldn't
get the letter S, now I couldn't get a two or a five, and so I would,
you know, all the 25 mile hour limit speed limit signs I would see on
the road I would say, how did they do that? And, so, anyway, I'm
depressed by this, knowing that there's still so much to do. The, but
I've got lots of users around the world who don't know that I'm
depressed about it, and so I'm trying to, but my font designers then
said, look Don, the period of apprenticeship for a type designer is
always five years, you've only been working on this for two years, how
can you expect to have, you know, have succeeded after two years. So,
they were very kind to me and, and through the grants that we had now, I
could invite them to Stanford and spend, and, and, and spend time with,
with them and, and learn, learn from them what I should've done. And
this is when I found that the people in the graphic, the graphic artists
are just about the nicest people in the world as far as I've met so far,
anyway. So it was a great, great thing. I couldn't keep up with all my
teaching at Stanford though, I'm not on sabbatical but I found that
doing software was much, was much harder than writing books and doing
research papers. It takes another level of commitment that you have to
have so much in your head at the time when you're doing software, that,
that I had to take leave of absence from Stanford from my, from my
ordinary teaching for several quarters during this period.)

\subsection{\texorpdfstring{\href{http://webofstories.com/play/17125}{Literate
programming}}{Literate programming}}\label{literate-programming}

Meanwhile I was also coming up with the definitive version of TeX and
this idea of literate programming. So now let me go back to the story of
Weave and Tangle that you brought up a minute ago. In retrospect, I, I
think the greatest benefit to me personally out of all this work on
typography was the idea of literate pro- that I decided to call literate
programming, which is a way of treating computer programs as literature,
where, where the pr- a computer program is something that human beings
are, are supposed to read and you write that for people to read, rather
than for a computer to read, and so as I'm writing programs, I'm being a
teacher. I love to be a teacher. I'm, I'm not just teaching a computer,
I'm teaching a, the reader of my program and I'm, you know, I'm hoping
someday there will be a Pulitzer prize for the most literate program,
etcetera. I started out experimentally doing this because a friend of
mine, Professor Hoare, at Oxford said, Don, people don't ever read
computer programs. Oxford University Press was interested in maybe
publishing some computer programs, examples of how it should be done and
so that, instead of just having, just having this sort of, this hidden
documents that the computer programs would be out there for people to
study and comment on, the way other kinds of literature are, you know,
musical notation; people, you know, people publish scores of symphonies,
they don't just listen to the symphonies, so why can't we publish, you
know, computer programs? And he planted this idea but, but it scared me
because really, computer programs in the real world are so full of
compromises. A professor of computer science couldn't admit to having
written such a thing. We can write short programs that are really, that
are, that really look like little gems and we can pretend that when we
write a large program, it's all, it's all this jewel-like character,
but, but in fact, I don't think any large program would, you know, the
authors wouldn't have been proud to have other people to read it as it
was done at the time. So, I'm also saying, how can I write a program so
that people could enjoy reading it? So, eventually this idea of literate
programming grew where I could see a format, by which a program could be
presented to other people and the great thing that came out of this was
that also, I as the writer of the program would be able to understand it
better. So, eventually I had this system where I would typeset, where
I'd write a program in this web language and then I would, then I had
two computer programs going from it, one of it goes from the web
language to, to a computer code in Pascal language at that time, and the
other goes from the web language into a document that I can read with
things nicely typeset in it, cross indexes and expository developments.
This, this system is what I, it provides weekly delights for me to this
day, I'm writing programs in, in CWEB, which is the descendant of the
original web system, and I, it's a continual source of pleasure to me,
to be able to program in this way that I feel is so right.)

\subsection{\texorpdfstring{\href{http://webofstories.com/play/17126}{Re
writing TeX using the feedback I
recieved}}{Re writing TeX using the feedback I recieved}}\label{re-writing-tex-using-the-feedback-i-recieved}

I rewrote TeX from scratch using literate programming and added, you
know, changed a lot of the things that users had found awkward, added,
added the features that they needed, and, and the system TeX 82 came out
as the definitive one and it's still almost identical to what we have
today except for more features for, for European languages. The, so I
was finally getting to a system that I could believe would be stable,
and had some reason to- During this time I wasn't working alone, by any
means, I had not only my, my research assistants but also I had lots of
volunteers from the Stanford community and we would meet every week on
Fridays, well, at lunchtime we'd bring bagged lunches but we often
stayed two, three hours, and as I'm developing the new system TeX 82,
there are 22 chapters to the manual, and as I'm writing each part of, of
the manual, I'm discussing with the group, what should it really be, I
mean, what, all the issues that, you know, it's like I get sentiment
from, from lots and lots of people and, and it would not only be
Stanford people, but there was always visitors to campus and so we had
people from around the world participating in these sessions, font
designers as well as mathematicians and these sessions continued as we
were refining the Metafont language in later years. So, here I had a
great exposure to, to a lot of varieties of opinion, but I had to make
the final decision. I, I knew that we had this, I talked about creeping
featurism before, that every user wants to have their own thing, and I
also know about design by committee, where if you have a committee who's
responsible for something, everybody on the committee has to feel that
there's some part of the language they can point to, and the language
you get as a result is usually pretty disunified. So, I was going to
insist that I be the filter for all the ideas and I, and I implement
everything and, but I did allow a lot of people to express their opinion
and finally I reserved to myself, the decision. The decision of the
judge will be final, basically, yes. I wrote the program so that it
would last forever, and not, not be always improving, because of my
experience that I had with these improving systems, so I wanted to, I
wanted it so that if somebody creates a document with TeX 82 in 1982, he
can run it also in 2002 and get the same output on every computer in the
world, and pretty much that ideal has been realized. Now, I also
published the program in a book. The entire code is, is, you know, a 600
page book, ``TeX, The Program'', it's, I think more people in the world
understand that program than any other program of its size, because it
was written in this literate manner, and it was, it was, you know, in
all countries, people were looking at this and finding out how it works
and seeing exactly how they could use similar techniques in their own
program. This was done without royalties and, and the only royalties
that would come in for people buy the Users' Manual, and, and those
royalties are shared with the TeX Users Group, this international group
of users for their projects on promoting the use of the language.)

\subsection{\texorpdfstring{\href{http://webofstories.com/play/17127}{The
importance of stability for
TeX}}{The importance of stability for TeX}}\label{the-importance-of-stability-for-tex}

This was, this was kind of the birth of, in some ways, of open source
software movement, which, which has gotten very popular recently, but
it's different from the so-called GNU public license that is a very
successful today for most of the open source thing. It's different in
one way, that the, one way that's very important to me, and that is that
this open source GNU public license software is, comes with the saying
that anybody can change it, not only can anybody use it, they can modify
it to their hearts' content. I don't allow people to change TeX, unless
they call it something else. So, if you, if you make, if you give it any
other name whatsoever, then you're free to change it, but if you call it
TeX, then it's got to be exactly the same as everybody else's TeX and
the one that I personally have, have given, I have special programs that
are used to validate whether or not you've implemented TeX correctly.
And so, that's different from the open source. In fact, somebody once
looked at one of my fonts and said, oh I can improve that, I'll change
the spacing a little bit, and as a result one of my books, all of a
sudden, didn't typeset the same, and, and, but the, one of the early
distributions of Linux included his so-called improved font, and, and I
all of a sudden found out from my co-author that, that, you know, that
he was getting different paragraph behavior, and we traced it down to
somebody, you know, making what looked to him, like an improvement.
That's what happens when you have complete, you know, when anybody can
modify and they, like the Wikipedia nowadays, you know, it's, you can go
to Wikipedia and you can look at any article, and there's a little
button you can press saying edit, and you can change anywhere in that
article and it's there, it's in Wikipedia. They don't even ask you who
you are, making the change. I think that Wikipedia's enormously
successful, but it's so brittle, you know, if I was, if I spent a lot of
time writing an article for the Wikipedia, and I wanted to make sure
nobody screwed it up, I would have to check that article every day to
make sure that it was still okay, and you know, after I've done that I
want to move on and go on to other, other things in my life. With TeX, I
wanted stability especially urgently because people are depending on it
to be a fixed point that they can build on, so in that respect, I differ
from the GNU public license.)

\subsection{\texorpdfstring{\href{http://webofstories.com/play/17128}{LaTeX
and ConTeXt}}{LaTeX and ConTeXt}}\label{latex-and-context}

I developed TeX so that it would support many different formats that
would be at a higher level, so TeX is a fixed point that people can
build on, and one of the main ways that people have built on it, is
LaTeX and different versions of LaTeX improving over the years. Another
very important system goes way beyond LaTeX, called ConTeXt, I don't
know how they pronounce it, it comes in a Dutch, in Holland they have
very interesting pronunciation in Holland, and, but certainly, way most
popular is LaTeX, LaTeX, the author never has decided how to pronounce
it, and, and I don't use it myself, but I know that lots of people
really think it's exactly the language that's most convenient, you know,
for them to use. So, so I developed the basic system so that it would
support many different styles for special projects. If somebody wants to
write a dictionary between Chinese and Icelandic, they can make their
own format that would make this really convenient for them, and there's
many, many scholarly versions and for all kinds of versions, so I, I
tried to make it easy to build different structures upon. When I have
special formats that my wife and I use for family history reports and
for our recipes in the kitchen and things like this, each one looks
different when you type the computer file, but TeX is the engine
underneath in all cases.)

\subsection{\texorpdfstring{\href{http://webofstories.com/play/17129}{A
summary of the TeX
project}}{A summary of the TeX project}}\label{a-summary-of-the-tex-project}

The whole thing started in 1977 when I was working all by myself and
then in '78, we started having other users, '79 we had Metafont, you
know, in '81 was the publication of Volume Two in the new format. Many
users coming through this time and I've also contacted the good font
designers. Then I have to redo all the fonts and I have to have a better
language for fonts too, so in 1983 and '84 I had a major revision of
Metafont language and, and that's when Hermann Zapf came out to spend
some weeks here and we, and he taught me a lot about, you know, things I
needed to know as well as help me refine the new version of Metafont
language. Richard Southall and Chuck Bigelow joined our faculty for
several years. We had a generous donation from the people in Silicon
Valley. I was hoping that we might have a permanent project between
Stanford Art Department and Stanford Computer Science Department. This
turned out to be too difficult to, because we needed more than one
faculty person to do it, and Chuck would have been good, but there was
just too much for one person to handle, but we had more than a dozen
students who have a masters degree in typography, basically, and
they're, they've been extremely important in the, in the industry since
then. We had this class in font design at Stanford in 1984, co-taught by
myself, Southall and Bigelow. Unfortunately, the video tapes seem to, it
was all on video, but I think the tapes were erased, but these guys,
Bigelow and Southall gave brilliant lectures and I thought they were,
you know, they were archived, but somebody seems to have lost them or
re, re used them. But that was a big point in the development because I
was implementing the features of Metafont one week before they were
introduced by the class, and, and this was when the Sun Microsystems-
the Sun workstations were just new and brand new software for this, and
we were also teaching the students how to use the Sun workstations in
order to prepare their Metafont programs and so we had a very exciting
time where they were working day and night to make this course happen
and it was a huge success. That font language finally stabilized in '84,
that manual for Metafont was written and then I had to do my final
revision of the, of the fonts, of the, to make them look acceptable, and
for this, anyway, let's say I finished that in 1985 and, and so finally,
after eight years I was able to bring my typography project to a
conclusion. It was supposed to be a one year project for my sabbatical
year.)

\subsection{\texorpdfstring{\href{http://webofstories.com/play/17130}{A
year in Boston}}{A year in Boston}}\label{a-year-in-boston}

Now it was time for my real sabbatical. My real sabbatical year was
quite special and it started in the Fall of 1985 and, and I went to
Boston. Jill and I went to Boston, and, and the reason it's quite
special is because I had promised Jill that this would be her
sabbatical, not my sabbatical. It was, some people think it was generous
of me, but actually it was just less selfish than I could have been
because this was after 24 years of marriage, I was giving her one year
out of 25, and so, so this year in Boston, I did the cooking, I did the
shopping, I did the, the cleaning and she wrote books, and, and you
know, she'd had lots of projects that she'd been saving up and I had my
chance to learn how to do these things that she had been providing for
me patiently for all the years. It was easier to do it this year than
any other year because I, both of our kids were now in college and so I
didn't have to, I only had to, you know, take care of her, I didn't have
to take care of John and Jenny as well. Although, the kids did come to
visit a couple of times, you know, so I learnt a little bit about
quantity cooking, but mostly I just learnt how to cook for two. And we
lived in downtown Boston in Back Bay and it was very, very pleasant to,
I had to walk six blocks to grocery stores and I had just a little
backpack so I could carry a few groceries every day, but, and the
apartment wasn't too hard to clean. Anyway, it was a very interesting
year and I was incognito. I went over to MIT one day only and just to
look up a book in the library, but I didn't tell anybody that I was
there, I met Nick Trefethen once, who was walking on the Charles River,
but anyway, it was a year when I, when I did completely different
things. The only computer-related thing I did, we had an internet
connection in our apartment so that I could communicate to Stanford to
finish the typesetting of my books on typography because the five volume
series of books was published by my publisher to contain the results of
all of the work that had been done on typography, so the first volume is
the User Manual for TeX, the second volume is the program for TeX, the
third volume a User Manual for Metafont, fourth volume, program for
Metafont, fifth volume, all the programs for the computer modern fonts,
the, the letter forms in my books, and, and so during that year I also
saw those five books through the press located conveniently right there
in Boston.)

\subsection{\texorpdfstring{\href{http://webofstories.com/play/17131}{Writing
a book about the Bible: 3:16 (Part
1)}}{Writing a book about the Bible: 3:16 (Part 1)}}\label{writing-a-book-about-the-bible-316-part-1}

The other thing I started working on at that time was a completely
off-the-wall project that, that became another book later on called
``3:16''. ``3:16'' is a, is a book that's different from any other book
that's ever been written and there are either two reasons why this
happened. One is that it was a stupid idea to write such a book, you
know. Anyway, well, what's the other reason that it could be different?
Anyway, that, you know, that the time was finally ripe to write such a
book, I suppose. Anyway, I, ``3:16'' is, is a study of the Bible from a,
from a different perspective than has been used before and it's a, and
so, I have to explain it a little bit. The computer scientists find it
very natural, when they're studying a complicated thing, to look at
random parts of that thing and then probe those in-depth and try to
learn something about the whole by having a few parts that they
understand well. It's like the Gallup Poll where the Gallup Poll would
come and interview several people, you know, and after interviewing a
thousand people they'll speak for, you know, for what they think
millions of people are like and Nielsen ratings and so on, are based on
sampling. Well, I would use sampling when I was grading papers. If
somebody gives me a term paper, I don't have time to read all 50 pages
of the term paper, I'll choose a random page or so, and then I'll see
what that leads me to. The student doesn't know in advance which page,
which page I'm going to do. So, anyway, sampling is something that comes
naturally to a computer scientist. One day in the '70s I said, well, you
know, what would happen if I tried this on the Bible? I would, that's a
complicated thing. What if I looked at random parts of the Bible and
just to see what people- go to the library and see what people have said
about those, about those parts, I, instead of somebody telling me what
part of the Bible to look at, I would just take an ordinary part, you
know, one, one that was, wasn't pre-planned and I would see whether or
not I could learn something that way. And I found it was really
interesting to do this. In fact, I did it with a group of people at our,
at our local church, and we decided that we would go through and we
would study chapter three, verse 16 of every Book of the Bible, so we
started with Genesis 3:16, and then Exodus 3:16 and we continued through
until we got to Revelation 3:16 and, and I told you that I had gone to
Lutheran schools as a child, so I had been exposed to the Bible, but I
never really felt that there was any part of it that I knew well. This
way I would, I would have some 60 verses of the Bible that, that I would
really have nailed. I could say I knew Genesis 3:16 rather thoroughly
because I could go to Stanford library and I could check out all the
commentaries on the Book of Genesis and there was only a few pages to
read that would tell me what, what's important about Genesis 3:16. So,
the sampling idea was not, was a big, was a big win, not so much for
understanding the Bible so much, but for understanding the secondary
literature, for understanding the commentaries about the Bible. You
could, you could pretty well see how opinions changed during, during the
centuries about, about these different verses and, and you could see
that people, a lot of people had different insights and other people
hadn't done their homework very well, and so on, but you could, you
could get a pretty good idea on these thousands and thousands and
thousands of books written about the Bible as to, as to what their slant
was if you just happened to know a, a few of the, a few parts to probe
into. Anyway, this, this was a surprisingly effective way to learn
something about the Bible and, and so, it's one of these moments again,
where I wake up in the middle of the night, and I say, hey, this would
be, this would be a, a good book to, to, where I could take these things
that I had learned in the library about the 3:16 verses and explain them
to other people, that, you know, it was, it was a story that I think
other people would enjoy learning about as well.)

\subsection{\texorpdfstring{\href{http://webofstories.com/play/17132}{Writing
a book about the Bible: 3:16 (Part
2)}}{Writing a book about the Bible: 3:16 (Part 2)}}\label{writing-a-book-about-the-bible-316-part-2}

So I wrote to Hermann Zapf, who by then was a good friend, and I said,
Hermann, I got this idea for a crazy book called ``3:16'', can you do
the cover for me. I want you to make the most beautiful three that's
ever been designed in the history of the world, and the most beautiful
colon, the most beautiful one, the most beautiful six and then I can
have my book and it it will look great, saying 3:16. And, so Hermann
wrote back enthusiastically saying, you know, and he'd also taken a
couple of the verses of the Bible, that he'd found in his German Bible
and he, he did it with his calligraphy, and he's one of the greatest
calligraphers of all time, and then he said, you know, Don, I know
hundreds of calligraphers around the world, I could, I could get those
people to, to contribute a page to your book, and each one, each one
could do these, these verses. Well, as I said, graphic designers turn
out to be the nicest people in the world, and he was going to introduce
me to even more of them this way. So, he, while I'm in Boston, and
cleaning house and things like this, I also go to the Boston Public
Library and to the Christian Science Bible Museum and over Harvard
Andover Theological Library and, and study, and copy down all the
different translations of these verses that I can find and, and read as,
and you know, start reading the commentaries on the verses, so that I
can really research the, the subject much better than I had done
cursorily in the '70s. And, Hermann made, made the calligraphy for the,
the verse John 3:16, which is the one that, the reason we chose 3:16,
it's the most famous verse in the Bible by its number. People in the
Super Bowl always hold up signs saying John 3:16 and it's sort of a
capsule review of the gospel and, and I chose Hermann to be the
calligrapher for that verse, and he, he provided me with a sample page,
showing his, his rendition of John 3:16. So, while I'm in Boston, I get
the sample page printed at a good printer, and, and then Hermann and I
drafted a letter to these 60 calligraphers around the world, the best in
their, in their respective countries, inviting them to contribute their
pages for the other, for the other verses. So, this was a project that I
had time to do while I was in Boston, and, and I spent a lot of time in
the Boston public libraries and these other libraries, reading what I
could, about what people, what people's comments were on these, on these
verses, that I, this, I was, by the time the end of my sabbatical, the
end of Jill's sabbatical occurred, which I was in the middle of the Old
Testament, let me think, I was just, yes, I was just getting into Song
of Solomon at that time and then we drove across country to get back to
California and, and I stopped off in Yale for a couple of days to use
the, the divinity school library there, and did the book of Isaiah and
Jeremiah and I stopped off in Pennsylvania, you know, Pittsburgh
Theological Seminary, different places coming across country, working on
this. And then I got back to Stanford and I still had a, a few more
years to go through the New Testament and we got great libraries out
here, Stanford Library, Berkeley Graduate Theological Union Library, and
so it was a weekend project for several more years after that, to see
what the story of these 3:16s would, would be.)

\subsection{\texorpdfstring{\href{http://webofstories.com/play/17133}{Writing
a book about the Bible: 3:16 (Part
3)}}{Writing a book about the Bible: 3:16 (Part 3)}}\label{writing-a-book-about-the-bible-316-part-3}

I got the artwork though, from the world's calligraphers. That all
arrived while we were in Boston, and it was coming in, it was, it was
like Christmas every, every day, because we would get another beautiful
letter from somebody with, with their contributions to the book. And so,
the artwork was so great, I tried to make the quality of my accompanying
texts somewhere that would approach the quality of the artwork I was
receiving. I spent some time then, on weekends for the next several
years, and then it occurred to me, I had better digitize this art,
because the world was getting a little more digital all the time, and a
company called Adobe Systems had been founded and I called up, I called
up John Warnock, the cofounder of Adobe and by a serendipity, he
answered the phone. I tried it many times since then, it never worked.
But this day, you know, he answered the phone. I said- John, I've got
this, I've got some great artwork and, and I think I'd like to capture
it in digital form and we want to make a poster which puts the artwork
in a different size. Is it possible to do this with any computer
software? And he said- John, he said Don, I've got just the thing for
you, we have a program we're working on called Photoshop, and, and we've
got, you know, Streamline, you know, it turned out to be a kind of a
dud, but that's what he, you know, he liked it at the time and, and he
said, so, come on out and use our equipment, bring your artwork with
you. So, during the summer of 1989 I went over and I was, I was the
night watchman at Adobe Systems. I worked in their art department and I
had about 50 Macintoshes in a room every night when most people were
home. I could, I could be running my, Photoshop was really not anywhere
near beta test stage yet at that point, but, and I got to work with the
man, Tom, I just called him Tom, from Michigan, who, who was the main
implementer and, so we got some of the bugs out of Photoshop at that
period, but anyway, I could use the artwork from my ``3:16'' book at
Adobe and after, after several, a couple of months, I had, I had it all
in form that I like, ready to, ready to do the, Macintosh were very slow
at the time, and dealing with a file of two or three megabytes was a
heroic effort. You could start Photoshop going on a, on a filtering
operation on one machine, and then five minutes later, it would be ready
for the next operation, so that's why I could go to another Macintosh,
it's something like a chess master playing a simultaneous tournament.
You make a move on one board and then, you go to, you know, you go on to
the next thing. So, but I'm getting this artwork slowly through the
press this way and it takes also a long time to transmit these files to
the, to the printer, but they had, they had what they called a RIP in
those days, a raster image processor, and you could, you could print the
films that I could use for the, for the book, and I got a, I went to
Singapore to see the book through the press, and, and I was expecting
that, that the, that I might have one or two readers who, who were
touched by the book and liked it, and that would make my, make the whole
project worthwhile and my wishes were fulfilled within a week of the
publication. It was amazing. I was expecting, you know, to have a lot of
negative reactions to the book, actually, because why, why is a computer
scientist doing something that he has no right to be doing, writing
about the Bible, you know, would I, would I read a book by a theologian
about programming, you know, and the only thing I could, I could do that
a theologian couldn't do, is I can, I can in my book, commend
theologians for having done their work well, in many cases. You know, I
can, I can give a testimonial while they can't do that because they
obviously have an axe to grind, but, but as an outsider, I can at least
direct people, saying, you know, here's something really cool that, and
so that's one thing that I could do. But anyway, I, I found this weekend
project was kind of an enriching thing and I am so glad it turned out so
well as it, as well as it has.)

\subsection{\texorpdfstring{\href{http://webofstories.com/play/17134}{Giving
a lecture series on science and religion at
MIT}}{Giving a lecture series on science and religion at MIT}}\label{giving-a-lecture-series-on-science-and-religion-at-mit}

On the subject of religion I wanted to, I might as well go ahead a
little bit and say that, you know, not only did I, okay, so I'm
publishing this book and I'm thinking that people are mostly going to
say that Knuth is off his rocker because he's publishing a book about
the Bible and what he does well is computer science, are we going to be
able to trust him anymore and so on. But the reaction was, to my face
anyway, was quite the opposite and a lot of people seemed to like the
book. And the most surprising thing was at the, in 1999, or actually it
was a year earlier when I got a letter from MIT inviting me to give a
series of lectures about science and religion, and, well, this is
something I'm obviously unqualified to do but it was hard to just turn
down the letter and say forget it because, well, it was sort of
something that I thought about a lot during my life and I thought maybe
some of the experiences I had could be relevant to other people who are
wrestling with the same things in their minds and so if I was ever going
to talk about these things in my life that was one thing or would I feel
incomplete if I didn't do something about it. On the other hand I
certainly didn't want this to be my career where I'm, you know, not
going out on a religious circuit or something. But on balance I figured,
okay, let's think of it as a win-win situation, I'll go and hope that
the Force is with me and I can say something of interest to people and
if I was ever going to do it one off and once in my life where could I,
where better than in Boston where I had spent most of the time in the
library studying and looking at these theological works and where I got
an audience at MIT, you know, ideal place to have an, to make an
influence if there's any such place. But they had asked me to give a
series of six public lectures, an hour and a half each, and that's, you
know, I have some standards where I didn't know if I had that much to
say obviously. So I decided okay, what I'll do is I will prepare only
half of these lectures and the other half will be Q and A, will be just
improvised responses to what the audience says. And so if it doesn't go
well so what, I tried, I did it and it's over with. And if people think
it's valuable then that's so much the better. But I could, I wouldn't
mind spending a few months in Boston again, being a nice place to be and
I had some things I wanted to look up also in the libraries while I was
there, so I said yes and I went then in the fall for three months at the
end of 1999 to relearn the Boston experience. This time I didn't live in
Back Bay, I lived near Harvard Square, across the street from the
Observatory, and I took my bike every morning along the Charles River to
MIT and got a lot of, made a lot of friends there, you know, younger
people that I hadn't known before, doing computer science. And then I
gave my six public lectures and a couple of others too about a computer
I designed and things like that, other projects that were going on. And
I got to, you know, to also to go to New York City a few times and I
visited Will Shortz and was able to make use of being in the East. But
here I am giving these public lectures, the night that, let's see the
day, I think my first lecture I was competing with Jesse Ventura who was
also giving a lecture in Boston that same day but I still had standing
room only in this hall, about 400 people, it's not that big a hall but
it was amazing to me, and they came back the second week as well. And we
had, the lectures were sort of in the late afternoon, about three or
four times, and then a pause and then another three or four times at
weekly intervals. So in these lectures I didn't present the answers to
any of the deep problems, I just said well, look, I think computer
science is wonderful but it's not everything in life and so there's
other things that deserve attention. And some of the paradoxes that I
was confronted with, how did they affect me and sort of asking the
audience maybe next time it's their turn to give the lecture and, you
know, so it was more of like a focus group or something, no, it was more
just like saying let's for once in our life talk about things that
aren't our subject of expertise but just how we balance these issues.
And so I had great comments from the audience and great, it was very
stimulating to do these question/answer sessions and I also learned a
lot from the people that I met there because there were people from many
different backgrounds. It was, again, kind of a plus thing but certainly
nothing that I wanted to do for the rest of my life, I've got to focus
on ``The Art of Computer Programming'', that's where I can do something
unique.)

\subsection{\texorpdfstring{\href{http://webofstories.com/play/17135}{Back
to work at Stanford and taking early
retirement}}{Back to work at Stanford and taking early retirement}}\label{back-to-work-at-stanford-and-taking-early-retirement}

At the end of 1986 my sabbatical was over and I was still working on
this book, ``3:16'', on weekends but it was back to my normal professor
life. Of course I got to Stanford and I plunged in as always and resumed
teaching my classes and I had great graduate students and- however I was
having almost no time to work on ``The Art of Computer Programming'' and
I don't know if any of the people watching this also go on their
sabbaticals but one thing that happens when you get back home is people
keep asking you- oh, aren't you glad to be back home? And they kept
asking me this for the next year and the next year after that and I, you
know, I would always smile and say, you know, yes but really I didn't
feel quite so good to be back home because I wasn't getting anything
done on ``The Art of Computer Programming''. And time is ticking here,
I'm 50 years old, I've got lots of work to do yet on ``The Art of
Computer Programming''. So at this time I'm 50 years old in 1988, so,
like I go through a whole year and I'm making two days of progress on
``The Art of Computer Programming'', at that rate how long, you know,
I'll have to be 150 years old before I'm done writing the book. So I
decided really this wasn't a sustainable situation anymore, what I would
have to do is figure out how to spend the rest of my life and finish
``The Art of Computer Programming''. So I came to a reluctant decision
in the summer of 1988 that I should retire very early from Stanford and
I should devote the rest of my time to finishing what I really do best,
``The Art of Computer Programming''. You know, since 1962 I've been
gathering tens of thousands of papers, read them, made notes on them,
organized this, all the things I think are really important to have in a
book and I wasn't having any opportunity to really get that stuff ready.
So I wrote a letter to our chair of our department and I said Nils I
would like to retire as of January 1 1990, a year and some months from
now, and I don't feel right about, I don't like the idea of a professor
who's going to just write books and not do the rest of the job that you
do at a university including not only teaching and advising students and
serving on committees but also fundraising and many other,
correspondence, answering people's queries. And I said, so I go and see
why, but still I have to write ``The Art of Computer Programming'' or
I'll never really be happy and it's going to take me many years of work,
so I'm asking that you find a replacement for me at Stanford, I wouldn't
want to take a salary for something, for not doing the work of- And
then, so we met with the Dean and the Provost and they first tried to
convince me to just carry on as I was and I said no, really, this book
is a special thing that I think makes it different from all the other
hundreds of professors that you've got, you know, this book is a little
bit more important than all the other professors' books, I'm telling
them. Well, the Dean checked around with some people and he said well,
the books are pretty good anyway so he thought maybe he could get
somebody to endow a chair by which I would, I have only the
responsibility to write the books. I said no, rather get somebody else
to do what I was doing instead and then let me take early retirement.
Well, early retirement at age 52 was not possible, what was possible was
that I could take a leave of absence for three years until I was 55 and
then I could retire. But that's what happened, January 1, I didn't
anymore have to serve on committees or raise money. On the other hand I
deeply regretted not being able to teach anymore and similarly advising
graduate students, that was something that has always been a special
joy. In order to account for the teaching we decided that I would give
semi-regular lectures called Computer Musings, which would be open to
the public and anybody could come, no credit would be issued for these
lectures but I would talk about whatever I thought was cool to talk
about and something that wasn't covered in Stanford's curriculum, and
periodically, we were going to shoot for once a month, I would do these
lectures. And I could have a title and that's great, I don't know if I'm
the only Stanford professor who was able to choose his own title but I
became Professor of The Art of Computer Programming, and that's my
official title. And that's The Art with a capital T. And that didn't
mean I had a salary but it meant that I had a nice title and very nice
stationery and I became Professor Emeritus of The Art of Computer
Programming, officially then when I was 55 years old. This gave me
medical insurance and also I have, in fact, a secretary and an office
and all the library access and everything I needed to do the books
right. But I'm not in anybody's inner loop where they're depending on me
to do something, like I can take a day like today and never show up and
nobody will be the worse.)

\subsection{\texorpdfstring{\href{http://webofstories.com/play/17136}{Taking
up swimming to help me cope with
stress}}{Taking up swimming to help me cope with stress}}\label{taking-up-swimming-to-help-me-cope-with-stress}

I felt very stressed about this time and I was having all kinds of
symptoms of flaky health and didn't know what was on and I think it was
because of the anxiety of this big change in my life since I'm going
from one phase to another. And so I even talked to a psychiatrist a
couple of times at my doctor's recommendation because the doctor thought
I was having too many symptoms. And the psychiatrist said well, you get
so many stress points for this and so many stress points for this and
you've got way too many stress points so why don't you try swimming
regularly. And I had never really done much exercise up to that point, I
was always sitting and writing or, you know, in Boston I could do a lot
of walking because we didn't have a car there but most of the time, you
know, I never understood why physical education was a required subject
in college, I just, you know, I was the score-keeper but I didn't really
ever play basketball myself. But it was a required subject so I had to
do some of these callisthenics or whatever. Well now I could see that
really the way the human body is designed it's okay to have your heart
moving a little bit every once in a while and I started swimming three
or four times a week in 1990 and all of a sudden my back was feeling
good, I was feeling good, other problems were clearing up and, well- I
was also discouraged though that Stanford hadn't found a replacement for
me and they still haven't. I mean they did some searches but the
professors they identified decided to stay where they were and so really
that part of the bargain Stanford never lived up to.)

\subsection{\texorpdfstring{\href{http://webofstories.com/play/17137}{My
graduate students and my 64th
birthday}}{My graduate students and my 64th birthday}}\label{my-graduate-students-and-my-64th-birthday}

At this time in 1990 my final two or three students finished their PhD
and so that gave a me a grand total of 28 students, mathematicians say
that 28 is a perfect number. It's just a bit of a joke but if you
consider all the divisors of 28 it's 14, 7, 4, 2, 1, and that adds up to
28 so it's called a perfect number. And remarkable 28 students, 28
different personalities from many different parts of the world, about
half of them went into academic careers and half of them are in
industry. For example, the people who worked with me on TeX, Michael
Plass became one of the key programmers for Xerox in their DocuTech
systems and many other of their desktop publishing system. Frank Liang
went on to be one of the key people in Microsoft Word and many of the
students that I had in the typography program, again, were very, have
leadership roles in the software industry for printing. Bob Sedgewick
was chair of Princeton Computer Science Department, Jeff Vitter chair of
the Purdue Computer Science Department, now Dean there, so I have a lot
of academic students. Others are back in their native countries, in
Spain, Brazil and, so all in all I'm really grateful that I had so many
great students who pretty much taught me as much as I taught them. While
I could take credit for their successes, it's really their, they were
great and are great. Only one of my students was a sad case and he seems
to have committed suicide after an unhappy life but 1 out of 28 I think
is a pretty good score card. And I still see at least half of them
because they're in this area now. It reminds me, I might as well mention
a huge surprise on my 64th birthday when people came from all over the
world for a two-day celebration and Jill had kept it totally secret from
me and I just couldn't believe, every time I'd look another way I'd see
yet another amazing person who came. Why 64 years birthday? Well, to a
computer scientist the number 64, the powers of 2 are the most important
numbers, so after 32 then the next interesting number is 64 and after
that 128 is the next interesting number. So I doubt if I'll have another
more appropriate birthday time and so we call it my millionth birthday
since in the binary number system you write 64 as one followed by six
zeroes. That's when I got to see lots and lots of my students, so the
thought came up now.)

\subsection{\texorpdfstring{\href{http://webofstories.com/play/17138}{My
class on Concrete
Mathematics}}{My class on Concrete Mathematics}}\label{my-class-on-concrete-mathematics}

Now I'm retired but then I still can't go full time to work on ``Art of
Computer Programming'' because there's other projects that I really have
to finish, for example, I told you about the ``3:16'' book which I was
doing on weekends, I wanted to get that done. And so I was doing that,
at this time it was just at the end of the '80s, it was published I
think beginning of 1990. And then I had introduced a class at Stanford
called Concrete Mathematics, I started it out in 1970 I think, maybe
1971, probably, well, it might have been in the fall '70 or spring of
'71. Anyway a brand new course in the curriculum because there was no-
Students at Stanford couldn't learn the kind of mathematics that I found
necessary for computer science work. And I'd been thinking a lot about
what it was that I had needed as writing the first parts of ``Art of
Computer Programming'' that I hadn't been taught myself in school. And
there was also a big to-do at that time about, some people were
complaining mathematics was getting too abstract, that it was too
detached from reality so I could make a little joke about that by
calling it Concrete Mathematics, although I said that the word concrete
doesn't really mean the opposite of abstract, it's a combination of the
word continuous and discrete. We had this course Concrete Mathematics,
the students were enthusiastic and I gave it over and over again.
Sometimes we had guest lectures when I'm on sabbatical or when I'm away
somebody else would come and teach it on the model that was established.
And one of the ways I always did teaching at Stanford, I was inspired by
George Pólya where I let the students do the talking in class instead
of me telling them what's in the book. I assumed they all know how to
read and so when they're in class we're going to do things that aren't
in the book. So I would always run my classes by, almost as if it were a
language class instead of a computer or math class where I would present
a question to the students and the students were supposed to figure out
how to solve the problem or at least to make the next step towards
solving a problem, or to try and fail and then we would learn how to
recover from failure. You know, you don't find books saying how to
recover from bad guesses and so we could learn that in class. In order
to, so that the students didn't have to take notes in these classes I
always had my teaching assistants take notes of what the students said
and then they would quickly publish these notes afterwards and as a sort
of a transcript of the session.)

\subsection{\texorpdfstring{\href{http://webofstories.com/play/17139}{Writing
a book on my Concrete Mathematics
class}}{Writing a book on my Concrete Mathematics class}}\label{writing-a-book-on-my-concrete-mathematics-class}

So by the time 1990 rolled around I had almost 20 years of these
transcripts of class discussions that had taken place in this Concrete
Mathematics course and it was natural and almost sort of mandatory that
this legacy be passed on to other generations, so I wanted to write the
book, ``Concrete Mathematics'', describing how the course had evolved
and what had been learned. One of the most conscientious teaching
assistants was Oren Patashnik who had not only transcribed the class
sessions but then he had drafted, I think, almost all the chapters of
the book as handouts to the students so that they could use it in one of
the years that I wasn't teaching. So I could go from his notes and
rewrite them but using his insights into pedagogy as I did this book.
And Ron Graham who was the most successful teacher of the course when I
was gone, he'd come in two or three times and the classes, the people
loved him so much when they took it that the Concrete Maths class for
those years would have annual reunions afterwards. And he also then
started teaching it at Princeton as well. So I had then this Concrete
Maths book with two co-authors, Oren Patashnik and Ron Graham. Usually
when I write a book I finish about one page a day, I mean if the book is
365 pages long it takes me a year, if it's 500 pages long it takes me a
year and a half. In this case I overdid it and I got the 600 page book
done in little less than a year and it was not a sustainable working
style but I really wanted so much to get back to ``The Art of Computer
Programming'' that Jill gave me a year where I didn't have to be quite
as, what you'd call responsible a husband as, you know, I would be
allowed to be, you know, less of a human being for a year so that I
could get this project done and then I could make up for it some other
time. So anyway that was a crash project, which was done under some
strain but I had the stimulation of the beautiful mathematical material
that I was writing up and of Oren Patashnik's draft which was in a
refreshing style that we could make this book a sort of California book.
One of the most fun things about the Concrete Math book is a feature
that I had copied from Stanford's, well, early '70s Stanford University
had a really innovative thing for prospective students. They put out a
book called ``Approaching Stanford'' that was written by former students
and in the main part of the pages of ``Approaching Stanford'' it would
have this, what you would expect to find from a university, telling how
great the place is and the weather and all this- faculty being great and
the students and everything. And then in the margins there's this
graffiti where the students are saying what it really is, you know, like
Stanford dorms are like a zoo without a keeper or something like this,
but they had the true clues about what it's like written by students.
And Stanford put this out and sent this out with its literature to
prospective students. And so I thought, okay, let's do this with the
math book too, let's have graffiti in the math book, let's let the
students comment on what they think about the subject and, you know, it
turned out that this idea certainly has caught on because the book
``Concrete Math'', let me say first of all, if anybody finds an error in
any of my books I pay them for it, you know, if they're the first to
find the error I send them a check. And any errors in this graffiti were
caught immediately, it was clear that all the readers were reading the
graffiti, the errors in the rest of the book, you know, maybe it would
take three or four years before somebody finds it but if there's
anything at all that they could comment on in the graffiti section that
was done. And also the book has been translated into about ten other
languages and in almost all the cases the translators also translated
the graffiti, they thought this was a vital part of the book. So the
book doesn't take itself too seriously but it's still, it turned out to
be a way to contribute also to the content of the book because one of
the comments could be, could say, you know, you'd better skim this next
few pages because you don't want to bother so much with it until you've
been through it once before. And then there were other, some of them
were pretty corny jokes but we censored them a little bit and so it was,
anyway, on balance that was- So we have a California book about
mathematics, a book that shows the informal style of Stanford classes as
well as what I think is a personal manifesto of the way to do
mathematics, the way I think that I love to do, work with the kind of
maths that I needed to study computer programs at any rate.)

\subsection{\texorpdfstring{\href{http://webofstories.com/play/17140}{The
Stanford GraphBase and updating Volumes One to Three of The Art of
Computer
Programming}}{The Stanford GraphBase and updating Volumes One to Three of The Art of Computer Programming}}\label{the-stanford-graphbase-and-updating-volumes-one-to-three-of-the-art-of-computer-programming}

I finished the ``3:16'' book, I finished ``Concrete Mathematics'', I
still can't get going back on ``The Art of Computer Programming''
because there's one more thing I have to do and it's called the Stanford
GraphBase. And this is a collection of literate programs that are to be
used for standard examples that would be used in Volume Four of ``The
Art of Computer Programming''. So that took another couple of years to
get that done right, it was necessary to have this out first before the
book was there so that the, I'd be able to finish other projects with
other people around the world contributing since they could use the
Stanford GraphBase to help. Finally then, about 1995 I'm able to open
the door of the room where I had been throwing all the new material for
``The Art of Computer Programming'' for 15 years. While I'm working on
TeX I really had no time, much to think about it so when I get something
in the mail relevant to Volume Four, Volume Five or so, I would just
throw it into a pile and then I had boxes and boxes, so finally it had
accumulated to about 17 linear feet of material. And besides I had all
of the material I had collected in previous years, so I decided I just
had to reorganise it all and that took another year, more than a year to
go through everything, every scrap of paper that had come in and put it
in a good place, make a computer index so that I could find everything,
make, put it into thousands of little file folders and also correct all
the errors known in Volumes One, Two and Three. People had been writing
to me, I had letters from 1981, 1982 that I hadn't answered yet,
reporting on errors in Volumes One, Two and Three, so I wrote checks to
all these people with interest from the day of the, you know, I had a
little computer program that computed these interest rates and I had a
mistake in it so I think I paid a bit too much interest there. But
anyway I sent out hundreds of checks and had a bigger, errata list for
``The Art of Computer Programming'' for Volumes One, Two, and Three and
this I could typeset with TeX and have it typeset correctly. All these
years people were still buying the books at a steady rate in the book
stores, they're buying like Volume One, Volume Three, they're buying the
edition that was done in the '70s but, you know, like the 30th printing,
the 40th printing. And it's still, each of the books are still selling
at a rate of about 4 000 books a year, something like that. And Volume
Two was the edition of 1981, the one that I couldn't stand the numerals
in but the mathematical material was okay. When my ``Computers and
Typesetting'' series of books, those five volumes came out, that was all
done with the new fonts, with the correct typography and so on. With my
``Concrete Mathematics'' book I was able to use a new font designed by
Herman Zapf and for the ``3:16'' book I had another font that I'd worked
on so I had- So it was only ``The Art of Computer Programming'' that had
ugly typography by this time and still I didn't want to, you know, it's
been taking me so long to get back to writing Volume Four, how could I
stop and redo Volumes One, Two and Three without waiting still longer.
So, to the rescue came Silvio Levy who lives in Berkeley, who was very
active in many projects of Mathematics and he's now the librarian at the
Math Sciences Research Institute. But he was a big fan of TeX and he and
I created the CWEB system of literate programming using C as a
programming language instead of Pascal. So Sylvio, some- the errata
list, finds One, Two and Three and he decided, pro bono, that he would
typeset Volumes One, Two and Three in TeX and he would ask the
publishers to pay him a nominal amount for this to which they, of
course, readily agreed. And he and his wife, Sheila, did the
proofreading and they did a marvellous job going through all three
volumes and incorporating all the errata in my errata list. And then it
didn't take me long to, I mean it took me three or four months, but not
really anything near what it would have taken for me to do all the work
myself, so that we could have, finally, as 1997/1998, we could finally
have ``The Art of Computer Programming'' brought up to date with decent
typesetting and with all the 20 years of improvements that had been in
my files, now incorporated into the text.)

\subsection{\texorpdfstring{\href{http://webofstories.com/play/17141}{Getting
started on Volume Four of The Art of Computer
Programming}}{Getting started on Volume Four of The Art of Computer Programming}}\label{getting-started-on-volume-four-of-the-art-of-computer-programming}

I'm ready then to start finally the work on Volume Four but it still
isn't out, right, so what's happening? Well, in fact, about 400 pages of
it are out now in paperback and it takes me more than a day a page, I'm
not able to convince myself to settle for just reproducing things that I
find in my files but every time I look at something I see that there's a
chance to improve it, and that takes me another few days. So I'm adding
a lot of material that doesn't appear in the literature as I'm writing
Volume Four. I can't sustain that either I know but I tell myself that
the parts of the book I'm writing now are so fundamental that they're
different from the parts that will follow and that next year I will just
be putting together things that are not very highly original. But, in
fact, like tonight I'll be finishing the index for a new section and
this is about 60 pages and out of the 60 pages I think they're, you
know, well, 15 pages of it is stuff that hasn't appeared in the
literature and quite a few different ideas are there, and the same in
these 400 pages of Volume Four that have come out, I can't resist going
beyond the sources that I see when it's such a fundamental part of the
subject. So I'm not the, at the rate I'm going it's pretty clear that I
won't I finish the whole project until I'm 90 years old. So, even though
I'm not putting on, you know, I'm not going beyond the table of contents
that I jotted down on that day in January 1962 when I was asked to write
a book on complilers, I still have sort of that table of contents, the
field has grown dramatically and there's so much good stuff I can't
resist putting in. The, let me see- So that's my life right now, pretty
much going through my files for Volume Four. All the things I'm reading,
I keep on reading more things in the library but then that suggests
ideas that I put together and I contact people around the world for
their help in trying to debug the ideas and to check that I got things
straight. And hopefully I'll be able to continue this for a long time.
It's fun because I get to put together the work of authors that didn't
know about each other's work, so then I get a chance to make a natural
connection between those things and I love to find a way to present
material. I know that I'm getting to be an old man now and my style
isn't as sprightly as it used to be when I'm young but I still have the
3000 pages of manuscript I wrote in the 60s that I can use those
sentences to make, to keep a little bit of youth in there. Still I know
that the material is not an easy read, people will maybe keep it in the
bathroom or something like that or they'll have the book on their shelf
just to prove that, just to imply to somebody that they could understand
it. The subject is not, there's no Royal Road to some of this material,
some of it is inherently difficult but pretty much I'm enthusiastic
about the way things are going. When I, like this morning I was making
the index and I see parts that I had to struggle with when I wrote but
now when they come out it looks like a good story that I didn't- and the
struggle, the fact that I was struggling with it doesn't seem too
obvious to me now.)

\subsection{\texorpdfstring{\href{http://webofstories.com/play/17142}{Two
final major research
projects}}{Two final major research projects}}\label{two-final-major-research-projects}

Now I'm still able to do small research problems that I can solve in a
day or so but I knew that I'd never be able to work on a major difficult
problem, you know, as I'm working on ``The Art of Computer
Programming'', it's hard for me to give that up. So I found myself
working on two big projects, one of them at the end of the 80s and one
at the beginning of the 90s, which were sort of monographs. And the
first one, a project called ``Axioms and Hulls'' where I was studying
interesting problems in geometry, in convex geometry, and I got to work
on that while I was in Singapore getting the book, ``3:16'', printed,
this turned out to be a little book of 100 and some pages. And then I
started looking at the problem of random graphs, a very fascinating
phenomenon where sort of a big bang occurs when you start out with
points that are totally disconnected and then you choose two points at
random and add a connection between them, and you keep doing this. And
as soon as you pass the number of connections, about half the number of
points, suddenly almost everything almost always becomes connected, a
large part of it becomes connected that's called the Giant Component.
And I had been studying this problem with Boris Pittel, a visitor to my
research project at the end of the '80s and we got some preliminary
results that seemed kind of exciting mathematically, and so I was
writing up the paper and I realized that there were even more results
around the corner, in fact, the more I looked at the problem yet another
beautiful pattern seemed to emerge. So, as it turned out, it was it
finally developed into not only was I studying the Giant Component but
we wrote a research paper which was a giant paper, it filled the entire
issue of a journal. And I had three co-authors, Boris Pittel, who I
already mentioned and Svante Janson, Tomasz ?uczac from Poland. We all
kept finding more things about this subject and challenging each other
to do yet another amazing step in the exploration of this phenomenon
that was coming through. So, in a way it was like having written two
more PhD theses at the end of my research career, these two monographs,
the one on Axioms and Hulls and the other one about the Giant Component.
And this was then my swan song, you know, after which I said, okay, now
I've been there and done that and all the fun problems now I'll leave to
other people if I can't solve them in a day. Or I try to give myself an
hour but after an hour sometimes I think well, maybe another hour I'll
be able to do it. So from now on if I think of a new problem and I can't
see how to solve it I will pass it on to somebody else and say, don't
you think this is a cool thing and let them have the fun of solving it.)

\subsection{\texorpdfstring{\href{http://webofstories.com/play/17143}{My
love of writing and a lucky
life}}{My love of writing and a lucky life}}\label{my-love-of-writing-and-a-lucky-life}

As you can see from my comments that I'm addicted to writing, I love the
idea of communicating ideas to other people. I think, in every case, the
books that I've written were things where I had learned about some
phenomenon that I thought was just too good to keep to myself and so I
wanted other people to share in the joy of reading it. So it turns out
then that I have more than 20 books in print now and that's, you know,
so many that I doubt that there's anybody in the world who's read more
than half of them. And I sometimes think what tragedy it would be if
there were ten people in the world like me because we wouldn't have time
to read each other's books, you know, it doesn't scale up. Still, you
know, there's a story behind every book, there's a story behind every
paper that I wrote and it was not a, you know, I don't understand this
idea of publish or perish because I never wrote any of these papers
because I felt that I needed it for my career or something like this. It
was always because I thought there was a cool idea that was just waiting
to be communicated. A guy asked me two or three weeks ago, it was a
project that he's doing for his grade school teacher, he was supposed to
take videos of some people that he knew just a little bit and ask them
the following question; if you could do one thing over in your life and
do it differently, what would it be? And, you know, I'd never been asked
that question before and after five minutes I gave up. I mean I'm
probably getting forgetful at my age but I just couldn't think of a
single thing in my life that I regretted, it was, I mean I just feel too
absolutely too lucky. I came along riding the crests of different waves,
I mean I was born at just the right time when computers were beginning
to arrive in the world and I just happened to have this background, this
combination of skills that made my peculiar way of thinking in harmony
with what you need in order to make a machine do tricks. So if I had
been born ten years earlier, ten years later the whole thing would be so
much different. I was early enough in the computer field that the
problems were easier then and I could solve them. We creamed off of the
easy ones, now those people who are younger have only harder ones left
to go or have to look at problems in completely new areas which of
course are very exciting, with things like robotics and other types of
applications that are now opened up because the fundamental, easier
problems have provided now another level of interesting problems that
are attackable. But I just can't imagine being luckier or having had a,
even though many of the things I did throughout my life weren't popular
and not everything turned to gold, I just can't think of anything I
would rather have done differently, and so it was very hard for me to
answer this boy's question.)

\subsection{\texorpdfstring{\href{http://webofstories.com/play/17144}{Coping
with cancer}}{Coping with cancer}}\label{coping-with-cancer}

Last year I learned that I had cancer and hopefully now we found it at a
very early stage so that it won't be, it won't prevent me from getting
too much more of ``The Art of Computer Programming'' written but I
didn't know at the end of last year how bad the cancer was. And so I
just want to mention a little bit about this story because it's
something that happens to- everybody's going to die but you never know
what is going to finally do for you and everyone's going to have to face
this kind of thing once in a while so I just want to mention a little
bit about how it affected me last year. I'm now at the age where I
attend a lot of funerals of people who I've known and my colleague, Bob
Floyd, died a couple of years ago and I have, every once in a while,
wondered about how to face death. It's certainly inevitable but I always
found that I was ready to die if I had finished a book but not a week
before finishing a book. You know, that would be too, too terrible if I
couldn't finish the project. Then when it was done then, okay, it's all
right again for a little while. After my father died in the '70s I was
confronted with this issue and had to sort of resolve it in my mind as
to how I should take it and I remember it took about a month before I
could go an hour at a time without thinking about death and the
questions, but I thought it through then and got a little more
comfortable with it. Now in recent times I've known of a lot of cases
where someone died and we said, oh, but it was a happy ending because
just a week before they were able to meet with their children or their
friends and sort of get their affairs in order, you know, they had time
for kind of a nice summing up and everything was as good as one could
hope for. So, well, the reason I'm saying this is because last year
after I had the diagnosis of this prostate cancer we scheduled surgery
for December, but I had meanwhile been invited to give important
speeches in Europe, especially at the 150th anniversary of the founding
of the ETH in Zurich, the main Swiss Federal Institute of Technology, I
think it's like the MIT of Europe. And so I was a featured speaker there
and interviewed by all the media and so on and I didn't, of course I
didn't tell my hosts anything about this surgery that was going to be
taking place, but it was just like, you know, my talk went well, the
people were all enthusiastic about the things I'd been doing, I didn't
have an enemy in the world, I was sort of at a summit. So if I was going
to die in December in this surgery it would be one of these happy
endings that I had known about in the other, even more than that because
here I was, I mean there were profiles coming out about my work and
following my talk, at Zurich I got an honorary doctorate, it was like my
31st honorary doctors degree, more than Ronald Reagan has received, not
as many as Bob Hope but still it was, you know, amazing for computer
science. So I'd achieved all the, everything except finishing ``The Art
of Computer Programming'' that I could have hoped to achieve in life.
And then after Switzerland I flew back and had a week with my children
and grandchildren where we all were, had a great time, and so if I was
going to die this again would be just the perfect way for it to end. And
so the way I looked at it was really that I've had every possible
advantage and gift, so if I was going to die in the surgery that would
be okay, that would be just, you know, a sort of a- I wasn't expecting
that I should be upset if it didn't go well, but if it did go well, then
I would just say okay, this means that I have a chance to do some more
as a gift but not as a right, not as something that I'm entitled to if
you can understand what I'm saying. And so, and it's not just something
that I was saying but, you know, I really, when I went into the hospital
I didn't have any plans as to what I would do when I would get home
again afterward, I just thought okay, if I get home I'll think about it
then. And so I'm in this surgery for four or five hours and unconscious
and then all of sudden I hear happy noises and people saying, oh, you
did fine. Okay so, well great, I'm still alive, maybe I can go a little
further. It's no fun being recovering from surgery but still everything
went as well as could be expected and here I am today and I'll be
starting a little bit of radiation therapy next month as a precaution
but the prognosis looks pretty good. So every day from now on where my
health is good I consider as something that, as a special gift, gives me
a chance to do some more of what I do best, is trying to take the things
that people in computer science have discovered and put them together
into a story that can be understood by some of those who can't
understand the original papers.)

\subsection{\texorpdfstring{\href{http://webofstories.com/play/17145}{Honorary
doctorates}}{Honorary doctorates}}\label{honorary-doctorates}

I used to say that I accepted these doctor's degrees in order to make my
mother happy. And I think it did but after she died I still accepted
them so I knew that it meant something to me as well. But I do think I
don't have time to travel very much now but I have travelled for the
purpose of these- I guess I should write a book about commencement
addresses because I've sat through so many college degree ceremonies,
but it's important to the field of computer science because we're a new
kid on the block, computer science hasn't existed for more than 40 years
while other subjects have been going on for a long, long time. And so
when a university decides to award a doctorate in the field of computer
science this is good news for the field, that places are giving more
credence to the idea that we have a field that stands on its own with
respect to fundamental problems that it is working on. So for me this,
since I was one of the earliest, you know, from the generation where
somebody can give me an award where I can sort of represent the field of
computer science, so I consider these degrees are more primarily for the
field rather than for me personally. My friend Bob Floyd always said
that it was getting the first award was the hardest, after that, people
are always hesitant to give an award to somebody who's never received an
award before but after the first one then it's all coasting. And he said
he was going to, he never got a PhD himself but he was going to get it
by the green stamp method, if he had enough letters addressed to him as
Doctor Floyd then he could submit that to some place and they would give
him an honorary doctorate.)

\subsection{\texorpdfstring{\href{http://webofstories.com/play/17146}{The
importance of awards and the Kyoto
Prize}}{The importance of awards and the Kyoto Prize}}\label{the-importance-of-awards-and-the-kyoto-prize}

I think awards are important in a person's life to validate that other
people appreciate the work that goes into, you know, the things that I
do are fun but there's also parts of them that aren't fun and it's
pleasing to know that it's appreciated, so getting an award is certainly
a fine tradition. The first one of importance to me was to get the Medal
of Science from President Carter and this was totally unexpected but it
was my great privilege to be sitting next to Richard Feynman when he got
the Medal of Science from President Carter. And when he, when, just
before I went up to get it he went like this with his elbow and said,
okay Don, here's your big moment. Well, he's one of my huge heroes, I
knew him at Caltech and so this was a big day for me for sure. Then some
other prizes where I could represent computer science, there was a prize
called the Harvey Prize in Israel, again, these are prizes that are open
not only to computer scientists but also to chemists, physicists,
biologists, people of all scientific disciplines. Some other prizes also
were open to people from humanities, I'm glad to say one of my
doctorates is a doctor of letters, I mean after I worked on Metafont I
ought to have a doctor of letters I figure. So this has been something
that, I guess deep down, gives me some satisfaction and encouragement to
keep going. The biggest prize of all of course was the Kyoto Prize which
was about ten years ago and that's a prize that is probably the best a
computer scientist can hope for, it recognises a lifetime achievement in
the field and it is offered to somebody in technology every three or
four years. I, at that time, I was able to bring my family and my wife's
family and my sister could come and my mother and father-in-law and
sister-in-law with my kids and spent several weeks in Japan so it was
also a good thing for the whole family. During that time, I was in Japan
for three weeks, I gave 13 lectures on 13 different subjects, eight of
which were prepared and five of which were improvised. I got to meet the
Emperor and Empress of Japan, you know, and wow, she is an extremely
impressive person. I got to meet my hero, Nob the greatest puzzle expert
and we could go to a hot baths together with him and his family and so
we could experience many, many parts of Japan and this was another
important highlight of my life.

And there's a nice, if I'm not mistaken, a nice tie-in to the beginning
of your career as it were in grade school in Milwaukee. You donated some
of the prize to your grade school.

That's right, the Kyoto Prize also comes with money, it's not as rich
quite as the Nobel Prize but it's enough to convince the world that they
thought twice before they gave the prize. And so it amounted to about
\$400,000, and Jill and I didn't want this to ruin our life because we
were happy without the money so we didn't know what we, if having this
would be best. So we used 100,000 to pay for the trips of our family and
100,000 went to my school where I had started the first grade through
eighth grade and 100,000 to Stanford and 100,000 to pay for a new pipe
organ at the church where I go here in Palo Alto.)

\subsection{\texorpdfstring{\href{http://webofstories.com/play/17147}{The
pipe organ (Part
1)}}{The pipe organ (Part 1)}}\label{the-pipe-organ-part-1}

My father was a church organist and he was, in fact the day I was born
he was called to the hospital from the church service. And part of the
training of a Lutheran schoolteacher is also with music and he's quite
good at music. He played in the Chicago Worlds Fair, he gave a recital
in 1934, whatever the year was, he gave several recitals I have programs
of and I inherited some of his organ music. When I was young I studied
piano but my piano teacher also knew something about the organ so when I
was 12 years old I had a year of organ lessons instead of piano lessons,
and so I learned a little bit about the instrument at that time. And I
met E. Power Biggs, he came to Milwaukee to give a concert that year. He
sat in, I rode in the car with him to the recital actually, I didn't
know how famous a man he was, but he was the man who brought organ music
to recording in America. So that was it though, I went back to piano, in
high school I was playing the piano with the, accompanying the band. In
fraternity at all the parties I was playing piano and we would sing a
lot of songs, mostly Broadway shows, tunes. And I enjoyed playing the
piano, you know, it took me, for example when we went to Norway for my
first time away, instead of renting a TV set we rented a piano and Jill
and I played four hands music with each other every night. So I've
always liked piano but in the 1960s when I'm at Caltech, somebody, we
were singing in the choir and somebody called me on Saturday and said
Don, our organist and choir director has just come down with a detached
retina and he's in the hospital, is it true that you took organ lessons
once when you were a teenager and can you fill in at tomorrow's service?
So I went to the church and I tried to remember how to play and on
Sunday I played the service and then, at that time, in order to recover
from a detached retina, you had to sit still for some months with your
head in, sort of rigid while the thing was healing, it was the way they
did it. So for the next six months I was the organist of a church there.
In Pasadena there was a great tradition of fine organ playing, a man
named Clarence Mader and many of his students were in the churches
around there so they had some of the best in the country actually at the
time. And so I started attending recitals of organists, they also
invited the best organists from the east to this chapter of the American
Guild of Organists and I began to realize that there's a lot of really
great literature for the organ. So I met a few people who had organs in
their home and I thought that was a cool thing. So when I had my year in
Princeton a large number of the best organists in the country were
headquartered in Princeton and I had heard some of them in Pasadena
while I was there so I took classes at Westminster Choir College where
these people were associated in Princeton. I audited the classes, I
found it was kind of strange as a full professor going back to taking
lessons again where somebody would tell me what to do, this was quite a
come down for somebody who'd been used to being king of the- but anyway
I had an excellent teacher in Princeton and she gave me lots of good
music that I could begin to learn. So the idea was planted in me that
pipe organ music is one of the great pleasures of life.)

\subsection{\texorpdfstring{\href{http://webofstories.com/play/17148}{The
pipe organ (Part
2)}}{The pipe organ (Part 2)}}\label{the-pipe-organ-part-2}

Since then I decided, okay, I got royalties from my books so I can use
this to acquire an organ in my home and so when we built this house we
told the architect of this dream that I would like to have a music room
in the house where we could have a pipe organ. And so even though we
didn't have the money to pay for it at that the time we put it into the
plans so that it would support three tons of weight in the front room.
And I went round talking to organ builders, the finest organ builders
that I knew at the time were in Denmark and a friend of mine, Peter Naur
at the University of Copenhagen, I visited him during the year we were
in Norway and spent a week talking to different organ builders in
Denmark. And boy, the gorgeous instruments that they built, very few in
America. If you're interested in anecdotes this is the only time in my
life I drank a cup of coffee and maybe that was the wrong country in
which to drink a cup of coffee, but that was the only time, but after
that I didn't want to try again. But the story is this, I was at this
organ builder and he didn't speak English very well and I didn't speak
German very well, we tried to communicate in German and he served me
coffee and I didn't know how to refuse so I took the coffee. And every
time he, but it tasted awful to me so every time he would be looking at
me though I either took a sip a little bit or I'd at least rattle it in
my cup so that he would know that I was paying attention to the coffee.
But then I did pour it into a plant at the end when he wasn't looking,
but that was my experience anyway. But he's a great organ builder in
Hillerød near Copenhagen. Well, I didn't wind up though with a Danish
organ because although I got a proposal from the man that I liked the
best it turned out that the way the labor laws are set that you can't
set a fixed price for anything, it depends if inflation occurs then you
have to pay more. And it was going to be two years in advance so I would
have to, you know, saying well, no matter what happens to the economy
I'm obliged to pay the current salary in Danish currency. And there was
a big problem about shipping, so it turns out that the Danish solution
wasn't the best. I did though meet up with Larry Abbott and Pete Sieker,
who have a shop near UCLA, on one of my trips to southern California,
and they built four organs a year very fine and I had known, in fact
they had built one of the ones that I had seen in Pasadena when I was
previously. And so I worked with them and came back to Stanford and
meanwhile here I had worked with our own church and our church in Menlo
Park had gotten a new organ and so I was on that committee and visited a
lot of American organ builders at the time and learned something about
the process of specifying an organ. I got lots of books from the library
and saw what the specifications were of the organs that Bach had played
on and many organs from the Netherlands and different parts of the
world. That was a good background and at Stanford we have four organs
and I could play on each one and listen carefully to the different
notes, get a feel for how they balanced and then communicate as well as
I could to Abbott and Sieker down in Los Angeles. So finally they
finished the design and their shop isn't tall enough to hold the organ,
it was in three pieces, they brought it in the truck up here and
installed it. It's only lived here in my home but it's quite a nice,
quite a fine organ for in a person's home. Usually a home organ has to
be so small that you can't get very many different kinds of tone quality
on it but they were able to make it so that I have dozens and dozens of
different combinations that work well.)

\subsection{\texorpdfstring{\href{http://webofstories.com/play/17149}{The
pipe organ (Part
3)}}{The pipe organ (Part 3)}}\label{the-pipe-organ-part-3}

Many Stanford students have given recitals here including Walter Hewlett
who's now one of the big people in the Hewlett Foundation and director
of Hewlett Packard, he was the son of the original Hewlett. So it's been
nice to have this also as now considered one of the Stanford organs. I'm
no great shakes as an organist but as a computer scientist I'm okay, so
this gives me an intro where I can go many places in the world and
people will show me their pipe organ and allow me to play. And I don't
have to be that good because I'm just a computer scientist, I'm not
supposed to be a musician. So I've played on the most, in fact in Zurich
last November I had an hour to play the fantastic organ at the
Fraumunster, the best organ in Zurich. I could play the organ at
Wanamaker's Department Store in Philadelphia, the largest in the world
and organs in Netherlands, the famous one at Haarlem. And, well, in
Kyoto there was a very fine organ with four ranks of pipes that were
based on Japanese instruments, quite intriguing. I guess the, I hardly
ever play in public, the only time really was in Waterloo, Canada, where
the professor of organ there and I put on a program of organ duets and
so some pretty interesting music was written for two players, usually at
one organ but in one case at two organs. And that was, I mean I
practiced pretty hard before that occasion so that was kind of the high
point of my organ playing of my life.

How frequently do you play these days at home?

I play in spurts, so sometimes I'll go for more than a month without
touching it and then I'll sit down and I'll play for three hours or
something like that. You know, it's a, I should really, you know, I
shouldn't penalize myself, I usually think oh, Don, you're behind on
this project, you can't afford to take time off and play. So I shouldn't
argue that to myself, I should allow myself to do it a bit more. But I
enjoy music very much and the pieces that, there are some pieces written
for organ that are so good that you never get tired of them no matter
how often you play them.)

\subsection{\texorpdfstring{\href{http://webofstories.com/play/17150}{An
international symposium on algorithms in the Soviet
Union}}{An international symposium on algorithms in the Soviet Union}}\label{an-international-symposium-on-algorithms-in-the-soviet-union}

I said that Analysis of Algorithms was what I wanted to call my life's
work, this is so, the field that I considered that, most of what I'd
done, including in writing ``The Art of Computer Programming'', is to
find a quantitative way to say how good a computer method is,
algorithms. So since I knew that algorithms was my great, you know, of
all the things that I enjoy and do, that this was somehow, had to be
number one, I was delighted to learn that the word algorithm comes from
Arabic, Al-Khwarizmi where Khwarizm is a region of Uzbekistan, now, but
at the time there was, it's the, there's a lake, I mean the Aral Sea
used to be called Lake Khwarizm. And it's a part of the world that's
pretty much forgotten, it's usually to the east, north, south or west of
whatever map you're looking at. If you look at a map of Iran it's north
of there, if you look at a map of Romania or something it's east of
there, if you look at India or Afghanistan it's west, you know, south of
Russia, it just doesn't appear, it's sort of a forgotten part of the
world. But I found out that that's where the word algorithm comes from,
Khwarizm. It means the person from the Khwarizm area, actually there was
a district of Baghdad which was the Khwarizm district where maybe, you
know, like the Armenians lived in one place and they would call it the
Armenian place, but this was the Khwarizm quarter of Baghdad. So I
thought, okay, it would be interesting to go to Khwarizm sometime in my
life and I looked up in the atlas and, oh no, it's in the middle of the
Soviet Union, how am I ever going to get there, there's no roads going
into it either showing from any, you know. So I mentioned this to my
friend Andrei Ershov who was visiting from Russia, from the Soviet
Academy of Sciences. Well, he wasn't my friend at the time, I didn't
know him very well, but he was a friend of John McCarthy and we were at
a party at John McCarthy's house. And I said to him, you know,
algorithms, the word comes from this part of the world and it's in the
Soviet Union, we should celebrate this some time, wouldn't it be nice to
have some kind of a pilgrimage where computer scientists of the world
can got to Khwarizm and celebrate the birth of this subject. And he
said, hey, that sounds like a great idea. So he goes back home and he
does all the work arranging it and getting the Russian Academy of
Science to sponsor an international symposium on algorithms to last two
weeks and to take place in the Khwarizm oblast, the district of
Uzbekistan that everybody has forgotten. So not only did I get to go to
Khwarizm but when I get off the plane I'm greeted by 200 children
carrying flowers, and, you know, and interviewed on local television.
And it was the first time anybody in the world had shown interest in
their part of the world, so we had, you know, the tremendous hospitality
of the people in the Mid East is amazing. In fact, you know, the hosts
were so generous I said jokingly well, how about providing me with a-
not concubine but anyway, you know, I'm sure he would have believed me
and done it unless I had really assured him that I was joking. And we
could visit, right near where we were is a kind of museum city of Khiva,
which is really where the author, al-Khwarizmi was from, well, I'm not
sure but anyway it's a kind of a preserved city which shows all the
great things of that culture. And at this conference half of the people
were Soviet Union and half of them were from the rest of the world and
we could just meditate on the significance of algorithms, so it was
another highlight of my life I guess, to be able to accomplish that
little visit and see that corner of the world. Amazingly we met children
in this village, many different nationalities, blonde, blue-eyed kids,
some of them with Korean ancestry and we visited the cotton farms and
picked cotton. We saw lots of, you know, I got myself a cap like the
people wear in Khwarizm so that when I'm working on algorithms I can be
properly dressed.)

\subsection{\texorpdfstring{\href{http://webofstories.com/play/17151}{The
Knuth-Morris-Pratt
algorithm}}{The Knuth-Morris-Pratt algorithm}}\label{the-knuth-morris-pratt-algorithm}

There was a story really behind every paper that I wrote pretty much and
this one is a kind of interesting story. The algorithm itself has became
famous but not, I'm not sure, well, I haven't had to use it myself for
the last 20 years but it's in all the textbooks and it makes a nice
instructive example of if you're trying to search through a long piece
of text and you're looking for a certain word- suppose I'm looking for
the word, well, the, or something like that, it would be foolish to look
for that word, let's look for Dikran or something. Okay, so I can go
until I, so the obvious way is you start at every place in the text and
you say is it a D? oh yes; then you look at the next letter, is it an I?
is it a K? no, this is the word direction or something. All right well
then go to the next place and start over again, is that a D? Well all we
observed was that you already know that it's an I, it's not a D so you
might as well skip ahead a little further before you make your next
test. Well, there's more to the story than that though, you can have
more complicated words, like you were looking up the word Didymus or
something where there's a couple of Ds and so you'll have- But the
interesting thing in this case was that a professor at Berkeley, Steve
Cook had proved a very amazing theorem. He said that if you could, if
you took a certain kind of computer that's very limited in its
capability, and if you could write a program for that dumb kind of
computer to solve a problem, no matter how slow that program was then
there was a fast way to write a program for a real computer. In other
words if you take a particular limited computer and if it can solve a
problem at all then there's a fast way to do it on a real computer. So
one of the problems that we could solve with this funny computer was to
decide whether a string of letters was a palindrome, whether or not it
read the same backward and forward, no, sorry, was a concatenation of
palindromes, so a palindrome followed by another palindrome and so on.
It was just a curiosity, I mean nobody really cares about concatenations
of palindromes but there it was, and so according to Cook's theorem,
since there was a way for this funny machine to recognise these
palindromes, then there must be a fast way to recognise these
concatenations of palindromes on a regular computer. But I couldn't
think of any good way to do it on a regular computer, it just seemed to
me that that was going to be a much harder problem. So, I thought I was
a pretty good programmer but here is this theorem saying that there's a
way to do it and I can't think of the way to do it. That was the first
time that I was sort of stumped this way by somebody saying that he had
a good way and I couldn't think of it. And so I took an afternoon, or
maybe an afternoon/evening and I worked out in great detail on a
blackboard the way Cook's construction would have finally given me a way
on my regular computer to do it fast. And all of a sudden, ah ha, of
course, that's the trick. And so I was smoking out the idea as to how
this general theorem would apply to a real programming problem and it
occurred that this would also solve the problem of searching in the
text. So I mentioned this on a trip to Berkeley where I met Vaughan
Pratt and he was the one that made most of the crucial insights, I was
just the one that wrote it up afterwards and then we found out that Jim
Morris had discovered the same idea a few months earlier and used it in
software and that other people had looked at this program and didn't
understand it so they took it out. But, anyway, it's a nice method for
efficient searching text but also good, instructive, to teach principles
of computer science so it's become famous and associated with my name.
But I've got 160 papers and each one of them has a, some kind of a
little twist to it that made it interesting to work on.)

\subsection{\texorpdfstring{\href{http://webofstories.com/play/17152}{My
advice to young
people}}{My advice to young people}}\label{my-advice-to-young-people}

If somebody said what advice would I give to a young person, they always
ask that funny kind of a question. And I think one of the things that I
would, that would sort of come first to me is this idea of, don't just
believe that because something is trendy, that it's good. I'd probably
go the other extreme where if something, if I find too many people
adopting a certain idea I'd probably think it's wrong or if, you know,
if my work had become too popular I'd probably think I'd have to change.
That's of course ridiculous but I see the other side of it too often
where people will do something against their own gut instincts because
they think the community wants them to do it that way, so people will
work on a certain subject even though they aren't terribly interested in
it because they think that they'll get more prestige by working on it. I
think you get more prestige by doing good science than by doing popular
science because if you go with what you really think is important then
it's a higher chance that it really is important in the long run and
it's the long run which has the most benefit to the world. So usually
when I'm writing a book or publishing a book it's different from books
that have been done before because I feel there's a need for such a
book, not because there was somebody saying please write such a book,
you know, or that other people have already done that kind of thing. So
follow your own instincts it seems to me is better than follow the herd.
My friend Peter Wegner told me in the '60s that I should, for ``The Art
of Computer Programming'' I shouldn't write the whole series first, I
should first write a reader's digest of it and then expand on the parts
afterwards. That would probably work for him better than me, much
better, but I work in a completely different way. I have to see
something to the point where I've surrounded it and totally understood
it before I can write about it with any confidence and so that's the way
I work, I don't want to write about a high level thing unless I've fully
understood a low level thing. Other people have completely different
strengths I know but for me, you know, I wrote a book about a few verses
of the Bible, once I understood those verses and sort of everything I
could find in the library about a small part of the Bible, all of a
sudden I had firm pegs on which I could hang other knowledge about it.
But if I went through my whole life only on, without any in depth
knowledge of any part then it all seems to be flimsy and to me doesn't
given me some satisfaction. The classic phrase is that liberal education
is to learn something about everything and everything about something
and I like this idea about learning everything about an area before you
feel, if you don't know something real solid then you never have enough
confidence. A lot of times I'll have to read through a lot of material
just in order to write one sentence somehow because my sentence will
then have, I'll choose words that make it more convincing than if I, if
I really don't have the knowledge it'll somehow come out implicitly in
my writing. These are little sort-of-vague thoughts that I have when
reflecting over some of the directions that distinguish what I've done
from what I've seen other people doing.)

\subsection{\texorpdfstring{\href{http://webofstories.com/play/17153}{My
children: John}}{My children: John}}\label{my-children-john}

Since the beginning I've tried to be a good father and not just an
absentee or something. I did spend a lot of time on writing my books but
I also spent a lot of time with them. One of my favorite papers was
inspired by them on a trip that we took in California, it's kind of a
joke paper called the ``Complexity of Songs'' and what we tried to find
was the longest song that you could have without knowing much. And, you
know, we would start out with songs like, ``99 Bottles of Beer on the
Wall'' and that didn't take much to memorize enough of it to sing for a
long time and then we went on to other, you know, There's a Partridge in
a Pear Tree. Anyway I could study each of these songs from a
mathematical standpoint to see how much, how long a computer program it
would take to generate the song to last a long time. And my kids were
part of this inspiration because we were on a trip to Pinnacles National
Monument and we were in the car and instead of saying, daddy when are we
going to get there, we were singing songs. And then we ran out of songs
and so we thought of this and it led to an interesting mathematical
problem and finally we came up with the final solution which was yeah,
yeah, I like it, yeah, yeah, I like it repeated endlessly and that was
the best solution to that problem. I'm happy that I have some
grandchildren now as well, four boys. It seems that something in my
mother- if she were alive today she would have, I think it is ten
grandchildren, great grandchildren, all boys. So there's something funny
going on here, it doesn't seem quite random. But my son has, amazingly
enough, became an athlete, his dad is a complete slouch but John, when
he was at Stanford was a world class Ultimate player and was, you know,
the most valuable player at nation-wide tournaments. And then he became
part of a semi professional team and they came in second in the world
several times, and he being left-handed was apparently, that was an
advantage of somebody playing Ultimate, but a really difficult game,
lots of running and strategy going on. So people would come to me and
I'd meet a lot of people in other countries that know my son and never
heard of me, and this is very interesting because of his athletic
ability. But he also turned out to be a terrific teacher working with
younger students and he's been a high school math teacher now for more
than ten years. He went through Stanford's program called STEP, Stanford
Teacher Education Program and then taught local schools here, now he's
teaching in Colorado, and won award as the best math teacher in Colorado
at a private school or something like that. So he's in his element, this
is what my father was and, you know, I've always felt that teaching is
the thing that I, is my role. I teach through books because this way I
reach a lot more people, John doesn't like to write, we talk to him
instead of getting letters home. There's such a need in this country now
for people who are dedicated high school teaching and it's such a
thankless job because of modern high schools. He was teaching kids who
were on drugs and pregnant with huge family problems and he didn't have
the situation that a lot of teachers have where people have guns in the
school. But considering the challenges and the low salaries that
teachers get it seems we need a real revolution in society so that the
people, that people who have the motivation to be good teachers are also
properly rewarded and get some credit for what they're doing. I've met a
lot of heroic teachers from Richmond, California, for example, who teach
computer science there and it's a hard life. They have to be supported
by their spouses and so does my son. His wife works for Sun
Microsystems, telecommuting and that means that they can make ends
meet.)

\subsection{\texorpdfstring{\href{http://webofstories.com/play/17154}{My
children: Jenny}}{My children: Jenny}}\label{my-children-jenny}

I'm very proud of my daughter who, right now, is not in a profession,
she's raising her two kids and she's quite active in La Leche League.
She and her husband lived at Oxford for several years and she was- put
on a lot of programs there while raising her kids and always has many
projects going. She also did things like coordinating the exhibit of the
artwork for my book, ``3:16'', which travelled around the world. She's
working on projects of creating museum displays and she's an excellent
proof reader and writer and right now, blogger. So it seems that I've
got these other parts of my life that I wanted to mention too-

What did she study?

She went to Brown University and she found out that if she took a
computer class at Brown that people would expect, her name was Knuth and
people would expect her to really excel and that so she, that scared her
out. I think actually she's very good at sciences but instead, Brown
University is unique in the freedom that it gives to students to choose
their own curriculum. They sort of, roll your own style there which is
well developed, and when she got there she started reading the catalog
at the letter A and in her freshman year she took Art, Anthropology,
Asian Literature, African History, everything she saw in the catalogue
that interested her but she never got to the letter B. But she wound up
as an Anthropology major at Brown and then she was going to do her
thesis in American, well, sort of the Anthropology of America. And she
had some, also some excellent studies of the native houses in Antigua
where she had gone on a dig and she studied the evolution of forms in
the, what do you call it, the models in clothing stores- the-? I'm
trying to think of the word.

Mannequins.

Mannequins, yeah right. So studying the shapes of mannequins, you know,
through the ages and so indicating some kind of interest that she was
going to do. And she made a big study of, also of river systems and how
people dealt with flood control and trying to keep rivers alive, quite a
broad range of interests. She never found the right adviser for her and
when her husband got a position at MIT then she took classes at Harvard
and she could have finished there but then he got a job at Oxford and so
she's never finished her formal education but she's done a lot of
interesting contributions through writing and I think right now she's
got her hands full and doing well. But in high school she showed
strength in science. I always thought that she, well, she was very quick
and we play games a lot and she always beats me and so does John.)

\subsection{\texorpdfstring{\href{http://webofstories.com/play/17155}{Working
on a series of books of my collected
papers}}{Working on a series of books of my collected papers}}\label{working-on-a-series-of-books-of-my-collected-papers}

The fact that I started swimming in the 90s also had a nice other
secondary effect and that's where I met you at the pool one day and you
said, why don't I, why don't we publish a book about literate
programming, this idea that you're so hot on. And I think, in fact, in
Japan such a book had come out already, that someone in Japan had
decided to write such a book. And I was, you know, I had plenty already
on my plate but then you went and got one of my best students John Hobby
to make a plan as to what such a book would contain and so then I was
sold on it. And in fact I got so hooked on the idea that I was thrilled
by the thought that I could have the papers that I'd written, collected
together and published during my lifetime. As a person who uses
libraries a lot I certainly make great use of collected works of
scientists and I know many volumes that I consult frequently of the
papers of mostly mathematicians, but physicists and others. But these
are almost always collected together after the guy died and sometimes
100 years after the person died. So here was something where you were
actually proposing to take several of the papers that I wrote that had
literate programming as a unifying theme and put them together in a
book. That, once I realized how satisfying this would be, you see, I
have these 160 papers and I have reprints of them all and they're all in
different sizes and shapes and colors and I always thought if my wife
wanted to surprise me some day she could have them nicely bound and then
they wouldn't all be in such chaotic form. But here you're saying no,
not only could it be nicely bound it could be, in fact, reprinted and I
could even correct the mistakes that I knew about in these papers. And
you were going to do all the work yourself and you had Beth Bryson
taking care of it so she sent me the cleanest copy. I mean I've never
had a case before where somebody had worked with my papers and presented
me with something that I didn't have to almost totally redo, this was
before I met Silvio Levy. And so I got a book that was amazingly
attractive to me as well as sort of fulfilling one of my dreams that I
didn't know I had. And then it developed then that all of my papers
would appear in subsequent volumes organized by the different topics,
and so far, six of those eight volumes are finished. And I must say how
happy it makes me to have them all in a consistent format and with the
errors that I know about, sucked out of them and put into context and
sort of recast the way I wish I'd written them in the first place. It
gives me a chance to, I mean they're not true to the historical record,
if somebody wants to know what error did I make in a certain year, you
know, they can look it up in the journal, but if they want to know what
excited me and why I wrote the paper and what I think is important about
that paper in the future, they can look at the book, at these collected
works that you've prepared, so there's six of eight are done. The one
that I'm dedicating to Bob Floyd is coming up next and it's about the
design of algorithms and the last one that I'm saving for dessert is the
one about fun and games that I've already referred to. And so, as I was
going through my priorities as to what would I do if it was told that I
only had a certain number of days to live, it occurred to me that I
really would very much like to have those last two volumes of the
collected papers finished. So I hope that that works out. I've got, the
part of Volume Four that I'm working on right now I have to do first,
but then I'm going to do these other two collected papers, get them
finished and continue on Volume Four after that point, is my current
thinking. The one book on typography is a particular pleasure for me
because it has lots of great illustrations in it, but the book on
analysis of algorithms is, you know, that's my main life's research work
and having all those papers put together in one place is something that,
you know, gives a lot of fulfilment.)

\subsection{\texorpdfstring{\href{http://webofstories.com/play/17156}{Why
I chose analysis of algorithms as a
subject}}{Why I chose analysis of algorithms as a subject}}\label{why-i-chose-analysis-of-algorithms-as-a-subject}

Speaking of analysis of algorithms quickly I should say that, you know,
I named the subject and I like it but I didn't mention the thing that
was sort of key in the beginning, why I thought that it would be rich
enough to serve as a life's work. And the reason is that when I was
working on a compiler in 1962 I took a day and I looked at an algorithm
called hashing, more specifically hashing with- it's called linear
probing, a certain kind of hashing. And I'd heard that some students in
Princeton had looked at this algorithm and they couldn't figure out how
fast it was going to run. And I had a few hours to kill so I took a look
at it and I got lucky and I saw how to analyze it and prove exactly how
fast it was going to run and this changed my life. I said, wow, this was
lots of fun working on this problem, figuring out and so it turned out
to be a beautiful mathematical problem although it had come out of a
totally practical computer science domain. And I knew that there were
many other problems in the theory, what's called queuing theory, that
are just special cases of certain kinds of algorithms that had, and
already this was a buzz word, queueing theory, the study of the way
queues behave. And that's just a special case of a certain kind of
algorithm, so if I consider the entire class of all interesting
algorithms then it's bound to be full of problems just as interesting as
queuing and hashing for which I knew that they were there. So that's why
right at that point I said mm, that wouldn't be bad for a life's, to
spend a lifetime on it because you have a huge number of problems, not
only do they have beautiful mathematical structure that ties together,
you know, hangs together in nice patterns, but also there are customers
out there so that when you solve the problem the people say hey, thanks
for solving the problem, Don. So it's a great field to embark in and
that has turned out to be the case.)

\subsection{Code used to originally pull the
above}\label{code-used-to-originally-pull-the-above}

\begin{verbatim}
#!/usr/bin/perl -w
use strict;
use LWP::Simple qw(get);

my @ids = (17060..17156);

for my $id (@ids) {
    my $url = "http://webofstories.com/play/$id";
    my $page = get $url;
    my @lines = split /\n/, $page;
    my ($title, $body);
    for (0..$#lines) {
        if (not defined $title and $lines[$_] =~ /<h3>/) {
            $title = $lines[$_+1];
            for ($title) { s/\s+/ /g; s/^\s*//; s/\s*$// }
        }
        if ($lines[$_] =~ /<div class="transcriptText">/) {
            # Usually it's a single line‚ but in e.g. 
            # <http://webofstories.com/play/17066> it's actually multiple 
            # paragraphs (missing the <p> tags in HTML!)
            for my $endline ($_..$#lines) {
                if ($lines[$endline] =~ m|</div>|) {
                    $body = join "\n", @lines[$_+1..$endline-1];
                    last;
                }
            }
            for ($body) { s/[\t\r ]+/ /g; s/^\s*//; s/\s*$// }
        }
    }
    my $____ = "-" x ((length $title) + (length $url) + 4);
    print <<EOF;
[$title]($url)
$____

$body

EOF
    sleep 1;
}
\end{verbatim}

\end{document}
